Для распространения и установки программного комплекса «COEX» на электронные вычислительные машины был выбран формат двоичного пакета *.DEB. Данный пакет включает в себя все необходимые файлы для работы программы, а также содержит список зависимостей. Он имеет строго типизированную структуру и используется операционными системами Unix. Строго типизированная структура позволяет операционной системе узнавать все нужные элементы для установки и работы с данным пакетам вне зависимости от программного обеспечения (далее ПО), находящейся в данном Deb-пакете, узнавать зависимости программных библиотек, необходимых для запуска программного продукта, содержащегося в двоичном пакете.  Выбор данного формата пакета обусловлен возможностью его гибкой настройки для любого программного обеспечения (возможность использовать встроенные скрипты для настройки процесса установки ПО), поддержкой почти всеми операционными системами Unix. А также широкой распространенностью использования deb-пакетов в семействе операционных систем Unix в связи с высокой популярностью дистрибутивов операционных систем, в которой он распространяется [1] http://www.tecmint.com/

Для пакета *.DEB содержащего программный комплекс «COEX» ранее использовалась ссылка на репозиторий содержащий данный пакет для его дальнейшей установки. При появление новой версии программного комплекса «Coex» создается новый deb пакет, после чего пользователю нужно было зайти на сайт программного комплекса «Coex», и скачать новый deb пакет данного комплекса. Для автоматизации данного процесса было написаны две программы, а также настроена операционная утилита CRON для запуска программ в определенный промежуток времени, определяемый в конфигурационном файле CRON. Первая программа запрашивает у пользователя права на внесение изменений в конфигурационный файл sources.list, затем находит данный файл и вносит изменения. 

\begin{figure}[!ht]
\center{\includegraphics[width=0.7\linewidth]{ser_1}}
\caption{ Программа для внесения изменений в sources.list }
\label{ser_1:ser_1}
\end{figure}

Изменения включают в себя адрес до репозитория   в сети Internet, где хранится deb пакет программного комплекса COEX, а также запись ключа, которым подписывают скаченные файлы с репозитория, вносятся также команды по установки и обновлению пакета при скачивании через менеджера. Данные действия позволяет менеджеру обновлений операционной системы Linux следить за версиями deb пакета в репозиторие, содержащего программный комплекс COEX, при появление новой версии менеджер сообщит пользователю о возможном обновлении. Первая программа находится на сайте проекта COEX, скачивается и запускается пользователем. 

Вторая программа находится на сервере с репозиториям, и с помощью системной утилиты CRON запускается в определенное время, при наличии изменений из ветки master в системе контроля версий GIT.

\begin{figure}[!ht]
\center{\includegraphics[width=0.7\linewidth]{ser_2}}
\caption{ Программа  для запуска сборки по времени  с помощью демона  «CRON» }
\label{ser_2:ser_2}
\end{figure}

Скрипт-программа на основе сведений из системы контроля версий GIT проекта COEX, создает версию для двоичного пакета формата deb, сам deb пакет создается при помощи скрипт-программы по созданию  deb пакета из тексты программ проекта.

\begin{figure}[!ht]
\center{\includegraphics[width=0.7\linewidth]{ser_3}}
\caption{ Программа для присвоения версии }
\label{ser_3:ser_3}
\end{figure}

Тестирование проводилось, на операционной системе «Debian 8.4», и ее наследниках(Ubuntu(16.04), Mint (17.3)).

\begin{figure}[!ht]
\center{\includegraphics[width=0.7\linewidth]{ser_4}}
\caption{ Установка на Debian }
\label{ser_4:ser_4}
\end{figure}

\begin{figure}[!ht]
\center{\includegraphics[width=0.7\linewidth]{ser_5}}
\caption{ Установка на Mint }
\label{ser_5:ser_5}
\end{figure}

\begin{figure}[!ht]
\center{\includegraphics[width=0.7\linewidth]{ser_6}}
\caption{ Установка на Ubuntu }
\label{ser_6:ser_6}
\end{figure}

В ходе тестирования проверялось корректность работы двоичного пакета после прохождения автоматизированной сборки, также проверялась функционирование программного комплекса «COEX», и корректность удаления проекта с компьютера пользователя. Результатом тестирования были выявлены проблемы с некаторами зависимостями модулей, и связи модулей c программным ядром «COEX». После выявления проблем были перепроверены и исправлены зависимости, в модулях где были найдены ошибки с зависимостями. А также переделана структура пакета, для исправления проблемы взаимодействия ядра «COEX» и его модулей при установке через двоичный пакет.  
