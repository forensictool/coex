Markdown — это язык форматирования (или язык описания форматирования), по принципу работы похожий на HTML, который используется для определения финального вида текста.
Работает это следующим образом — в текстовое поле вводится текст, форматируя его специальными кодами (тегами) Markdown, которые при сохранении формы конвертируются в абсолютно валидный HTML.
Преимущества использования Markdown:

\begin{enumerate}
  \item Основан исключительно на текстовом вводе. Нет необходимости использовать сторонние редакторы и инструменты — только обычное текстовое поле. Таким образом можно быть уверенным, что текст отобразится корректно;
  \item Минимальное количество кода. Всё, что нужно запомнить пользователю — это пара очевидных тегов и несколько правил;
  \item Исходный код максимально читабелен. Редактируя страницу, можно наблюдать код Markdown и сразу понять, где находится заголовок, жирный текст или таблица. Всё очень наглядно;
  \item Является open source языком, использующим BSD лицензию.
\end{enumerate}

Недостатки Markdown:

\begin{enumerate}
  \item Строгие конвенции, которые надо соблюдать;
  \item Определенные символы нужно эскапировать. При необходимости ввода одного из технических символов, используемых в Markdown для размётки, перед ним надо ставить обратный слэш ("\textbackslash").
\end{enumerate}
