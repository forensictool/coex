Для разработки программного комплекса для проведения компьютерной экспертизы было решено использовать Git.

Git --- распределённая система управления версиями файлов. Проект был создан Линусом Торвальдсом для управления разработкой ядра Linux  как противоположность  системе управления версиями Subversion (также известная как «SVN»). \cite{progit}

При работе над одним проектом команде разработчиков необходим инструмент для совместного написания, бэкапирования и тестирования программного обеспечения. Используя Git, мы имеем:
\begin{itemize}
\item возможность удаленной работы с исходными кодами;
\item возможность создавать свои ветки, не мешая при этом другим разработчикам;
\item доступ к последним изменениям в коде, т.к. все исходники хранятся на сервере git.keva.su;
\item исходные коды защищены, доступ к ним можно получить лишь имея RSA-ключ;
\item возможность откатиться к любой стабильной стадии проекта.
\end{itemize}

Основные постулаты работы с кодом в системе Git:

\begin{itemize}
\item каждая задача решается в своей ветке;
\item необходимо делать коммит как только был получен осмысленный результат;
\item ветка master мержится не разработчиком, а вторым человеком, который производит вычитку и тестирование изменения;
\item все коммиты должны быть осмысленно подписаны/прокомментированы.
\end{itemize}

GitHub ---  крупнейший веб-сервис для хостинга IT-проектов и их совместной разработки. Основан на системе контроля версий Git и разработан на Ruby on Rails и Erlang компанией GitHub, Inc (ранее Logical Awesome).

Сервис абсолютно бесплатен для проектов с открытым исходным кодом и предоставляет им все возможности (включая SSL), а для частных проектов предлагаются различные платные тарифные планы.\cite{github} 

Создатели сайта называют GitHub «социальной сетью для разработчиков». Кроме размещения кода, участники могут общаться, комментировать правки друг друга, а также следить за новостями знакомых. С помощью широких возможностей Git программисты могут объединять свои репозитории — GitHub предлагает удобный интерфейс для этого и может отображать вклад каждого участника в виде дерева.

Для проектов есть личные страницы, небольшие Вики и система отслеживания ошибок. Прямо на сайте можно просмотреть файлы проектов с подсветкой синтаксиса для большинства языков программирования. На платных тарифных планах можно создавать приватные репозитории, доступные ограниченному кругу пользователей.

Начиная с 5 декабря 2012 года, на сервисе добавлена возможность прямого добавления новых файлов в свой репозиторий через веб-интерфейс сервиса.

Код проектов можно не только скопировать через Git, но и скачать в виде обычных архивов с сайта. (Для этого достаточно добавить /zipball/master/ в конец адресной строки.)

Кроме Git, сервис поддерживает получение и редактирование кода через SVN и Mercurial.

Для работы над проектом <<COEX>> проектной группой был поднят собственный репозиторий на сервере github.com.

Исходные файлы проекта и файлы для тестирования можно найти здесь:

https://github.com/tusur-coex.

