\newpage
\ESKDthisStyle{empty}
\paragraph{\hfill РЕФЕРАТ \hfill}
Курсовая работа содержит \ESKDtotal{page} страниц, \ESKDtotal{figure} рисунков, \ESKDtotal{bibitem} источников, \ESKDtotal{appendix} приложения.

КОМПЬЮТЕРНАЯ ЭКСПЕРТИЗА, ФОРЕНЗИКА, ЛОГИ, QT, XML, GIT, GITHUB, LATEX, MOZILLA FIREFOX, WINDOWS, HTML5, CSS3, БИБЛИОТЕКИ, РЕПОЗИТОРИЙ, C++, GUI, BASH, APACHE SOLR, SQL, SIGNALS, SLOTS, QTHREADS, PDFCHEKER, NOSQL, VIBER, МЕССЕНДЖЕР, MARKDOWN, PYTHON, JAVASCRIPT, PDF, JAVA. 

Цель работы -- создание программного комплекса, предназначенного для проведения компьютерной экспертизы.

Среди задач, поставленных на данный семестр, было: 
\begin{itemize}
  \item определение индивидуальных задач для каждого участника проектной группы;
  \item исследование предметных областей в рамках индивидуальных задач; 
  \item переезд репозитория на веб-сервере GitHub~\cite{github};
  \item наполнение контентом сайта проекта;
  \item разработка плагина для сбора информации из мессенджера Viber;
  \item рефакторинг и расширение функционала графического интерфейса <<COEX>> и добавление функционала для удобства его эксплуатации;
  \item анализ данных, хранимых в NoSql БД Apache Solr;
  \item рефакторинг и расширение функционала плагина для сбора сохраненных логинов и паролей браузера Mozilla Firefox;
  \item сборка программного пакета проекта;  
  \item разработка плагина для нахождения PDF-документов со встроенными JavaScript-сценариями;
  \item рефакторинг и расширение функционала архитектуры проекта для возможности выполнения плагин в отдельных параллельных потоках и повышения скорости работы программы;
  \item размещение информации об имеющихся плагинах на официальном сайте.
\end{itemize}

Результаты работы в данном семестре:

\begin{itemize}
  \item разработан плагин PDFChecker;
  \item произведена установка и изучение принципов работы Apache Solr; 
  \item произведен рефакторинг графического интерфейса пользователя системы;
  \item разработан плагин для нахождения PDF-документов со встроенными JavaScript-сценариями;
  \item разработан плагин, осуществляющий сбор информации из мессенджера Viber;  
  \item произведен рефакторинг плагина для сбора сохраненных логинов и паролей браузера Mozilla Firefox;  
  \item собран установочный .deb-пакет системы компьютерной экспертизы;  
  \item начало работы над системой параллельного запуска плагинов; 
  \item созданы удаленный репозиторий на GitHub и сайт проекта;  
  \item проведено размещение информации об имеющихся плагинах на официальном сайте;
  \item исправлен ряд ошибок в исходном коде проекта.
\end{itemize}

Пояснительная записка выполнена при помощи системы компьютерной вёрстки \LaTeX\ согласно ОС ТУСУР 01-2013.~\cite{ostusur}
