\documentclass[russian,utf8,14pt,simple]{eskdtext}
\usepackage[numbertop, numbercenter]{eskdplain}

% - Полуторный интервал
\renewcommand{\baselinestretch}{1.50}
% - Отступ красной строки
\setlength{\parindent}{1.25cm}	
% - Шрифт Times New Roman
\renewcommand{\rmdefault}{ftm}

% - Наименование документа
\ESKDtitle{ }
% - Обозначение документа
\ESKDsignature{ФВС КР. Х.ХХХХХХХ 001 ПЗ}
% - Наименование предприятия
\ESKDcolumnIX{ТУСУР, ФВС, КИБЭВС-1208}
% - Проверил
\ESKDchecker{Давыдова Е.М.}	
% - Литера 
\ESKDletter{У}{}{}
% - Разработал
\ESKDauthor{КИБЭВС-1208}			

% - Убирает точку в списке литературы
\makeatletter
\def\@biblabel#1{#1 }

% - ГОСТ списка литературы
\bibliographystyle{gost2008}

% - Верикальные отступы заголовков 
\ESKDsectSkip{section}{1em}{1em}
\ESKDsectSkip{subsection}{1em}{1em}
\ESKDsectSkip{subsubsection}{1em}{1em}

% - Изменение заголовков
\usepackage{titlesec}
\titleformat{\section}{\centering\normalfont\normalsize}{\thesection}{1.0em}{}
\titleformat{\subsection}{\centering\normalfont\normalsize}{\thesubsection}{1.0em}{}
\titleformat{\subsubsection}{\centering\normalfont\normalsize}{\thesubsubsection}{1.0em}{}
\titleformat{\paragraph}{\normalsize}{\theparagraph}{1.0em}{}

% - Оставим место под ТЗ 
%\setcounter{page}{4}

% - Для больших таблиц
\usepackage{longtable}
\usepackage{tabularx}
%\renewcommand{\thetable}{\thesection.\arabic{table}}

% - Используем графику в документе
\usepackage{graphicx}
\graphicspath{{images/}}
\renewcommand{\thefigure}{\thesection.\arabic{figure}}

% - Для вставки гиперссылок
\usepackage[colorlinks]{hyperref}

% - Счётчики
\usepackage{eskdtotal}

% - Для переопределения списков
\makeatletter
\renewcommand{\theenumi}{\arabic{enumi}}
\renewcommand{\labelenumi}{\theenumi)}

\renewcommand{\labelenumii}{\arabic{enumi}.\arabic{enumii}.}
\renewcommand{\labelenumiii}{\arabic{enumi}.\arabic{enumii}.\arabic{enumiii}.}
%\renewcommand{\labelenumiiii}{\arabic{enumi}.\arabic{enumii}.\arabic{enumiii}.\arabic{enumiiii}.}
\setcounter{secnumdepth}{5}

\sloppy

\begin{document}
 \section{DA}
 \newpage

\chapter*{Введение}

Одной из задач на данный семестр стала задача поиска и переработки журнальных файлов Windows. Сложность данной задачи состоит в том, что все системные события хаписываются в журналы с особой структурой. Существует множество софта, позволяющего просматривать записи из данных журналов, но ни один из рассмотренных проектов не предоставлял открытый исходный код своего приложения. Но обо всем по порядку \\
Для решения данной задачи необходимо было разобратся с такими проблеамами как: \\

\begin{enumerate}
\item определить места хранения журнальных файлов операционной системы
\item изучить какие события записываются в журналы и отбросить ненужные журналы
\item разобраться со структурой журнальных файлов
\item выделить важную информацию их каждой записи журнала
\item автоматизировать процесс поиска журнальных файлов
\item реализовать конвертр журнальных файлов в формат XML
\end{enumerate}

Рассмотрим подробнее каждый этап. \\

\chapter*{Журнльные файлы операционной системы }

Во всех операционных системах Windows начиная с XP есть папка config, в данной папке помимо всего прочего находятся бинарные файлы без расширений из которых формируется реестр системы, а так же файлы с расширением .log и .evt. Как раз эти файлы и являются журналами в которые система записывает некоторые произошедшие события. Какие именно события пишутся зависит от настройки самой системы. \\

В файлы с расширением .log пишется системная информация, размер этих файлов всегда равен 1КБ. А вот файлы с расширением .evt сожержат информацию о подключении/отключении устройств, запуске/остановке программ, ошибок при работе программ, существует так же журнал загрузки операционной системы, журнал обновления системы. Так же опционально можно включить такие журналы как например журналы безопасности и обнаруденных угроз. \\

\charter*{структур .evt файлов}

Файл представляет из себя строки данныйх переменной длинны. Из сторонних источников стало извиестно что некоторые поля данных имеют определенное значение. А именно: \\
Первые 4 байта содержат длинну события в файтах, после длинны идет 4 байтный системный код сообщения. Затем 4байтовы номер записи, после него дата создания записи, время созания, идентификатор события, тип события и так далее. \\

Ниже приведен список полей записи, преднозначенной для счтиывания одного события из журнального файла. \\

	quint32 Length; \\
	quint32 Reserved; \\
	quint32 RecordNumber; \\
	quint32 TimeGenerated; \\
	quint32 TimeWritten; \\
	quint32 EventID; \\
	quint16 EventType; \\
	quint16 NumStrings; \\
	quint16 EventCategory; \\
	quint16 ReservedFlags; \\
	quint32 ClosingRecordNumber; \\
	quint32 StringOffset; \\
	quint32 UserSidLength; \\
	quint32 UserSidOffset; \\
	quint32 DataLength; \\
	quint32 DataOffset; \\

Из сторонних источников стало известно о пяти типах событий (поле EventType): \\

значение - тип события \\

\begin{enumerate}
\item 0x0001 - Error event
\item 0x0010 - Failure Audit event
\item 0x0008 - Success Audit event
\item 0x0004 - Information event
\item 0x0002 - Warning event
\end{enumerate}

У поля EventID удалось определить четыре значения: \\

\begin{enumerate}
\item 0x00 - Success
\item 0x01 - Informational
\item 0x02 - Warning
\item 0x03 - Error
\end{enumerate}

Среди множества полей записи события были выделены поля содержащие информацию о типе события, времени возникновения события и создания записи, пользователя от имени которого была сделана запись, а так же поле Data - поле с бинарными данными в которых записана подробная информация о событии. \\

\charter*{Автоматизация процесса поиска .evt файлов}

Так как комплекс работает с оброзом жесткого диска, то программе достаточно указать папку в которой находятся искомы файлы. Для поиска же можно взять список всех файлов папки и отфилтровать их по расширению .evt. Данную операцию можно сделать при помощи инструметов QDirItherator и QFileInfo из набора библиотек QT. \\

Первый инструмент необходим для получения списка файлов в папке config, а второй позволяет просматривать информацию о файлах. При просмотре иноформации будут отобраны пути до файлов с расширением .evt. После чего список путей будет передан передан процедуре обработки файлов. Которая конвертирует каждый .evt файл в XML и сохранит в диррикторию с результатами работы. Для чтения файла используется инструмент QDataStream, а для записи в XML документ QXmlSreamWriter. \\ 


На данный момент полностью реализован комплекс для работы с журнальными файлами windows XP\\

\charter*{описание используемых инструментов}

Для работы с файловыми системами в QT существует несколько библиотек. В данном проекте активно используются две: \\

\begin{enumerate}
\item QDirIterator
\item QDir
\end{enumerate}

QDirIterator — библотека, предназначенная для работы с файловой системой начиная с определенной директории как точки входа. Создав объект данного типа с указанием директории мы получим все пути которые существуют в файловой системе и начинаются с указанной директории. Данный объект поддерживает фильтрацию которая помогает выделять только необходимую информацию, исключая то, что нас не интересует, например можно вывести список только файлов находящихся в данной директории или поддиректориях, или исключить вывод символьных ссылок. Объекты данного типа используются для поиска файлов или папок на образе исследуемого диска. \\

QDir — библиотека позволяющая работать с конкретной директорией. Создав объект данного типа с указанием директории мы получим доступ к этой директории в программе и сможем работать в ней (просматривать содержимое; удалять, создавать или копировать файлы; создавать поддиректории). Данный объект так же поддерживать разные наборы фильтров выходных данных которые могут отсеивать ненужную информацию. \\

Так же данные библиотеки позволяют создавать объект QFile, который позволяет работать с файлом, путь к которому передается как параметр при создании, данный объект позволяет получить базовую информацию о файле, такую как относительный или абсолютный путь до этого файла, размер файла, тип файла или его имя. Так же позволяет перемещать или копировать данный файл.\\

Работа с xml-файлами \\

XML - eXtensible Markup Language или расширяемый язык разметки. XML разрабатывался как язык с простым формальным синтаксисом, удобный для создания и обработки документов программами и одновременно удобный для чтения и создания документов человеком. Задумка языка в том, что он позволяет дополнять данные метаданными, которые разделяют документ на объекты с атрибутами. Это позволяет упростить программную обработку документов, так как структурирует информацию. \\

В QT для работы с xml-документами используется две библиотеки: \\

\begin{enumerate}
\item QxmlStreamReader
\item QxmlStreamWriter
\end{enumerate}

Данные библиотеки позволяют создавать потоки для чтения и записи XML файлов и предоставляют набор функций для разбиения файлов на элементы и создания структур данных по записям в xml-файлах. \\

Команды для записи же позволяют записывать данные в файл автоматически дополняя методанные. Для этого существует набор команд при помощи которых создается xml-файл с заголовком, команды по созданию элемента, команды по добавлению атрибутов в элемент, команды записи конца элемента и конца файла. \\

<-----
ЗАМЕТКИ
Тут думаю ещё запилить описание написанных классов, зачем нужна каждая функция и добавить выдержки из документации о QDir* QFile* и QXmlStrea*
как источники можно указать:
1) http://doc.crossplatform.ru/qt/ - документация по QT
2) http://www.whitehats.ca/main/members/Malik/malik_eventlogs/malik_eventlogs.html - инфа о структуре evt-файлов
3) Николай Николаевич Федотов - ФОРЕНЗИКА компьютерная криминалистика 2-е издание. - как инфа о экспертизе вообще
с инфой по журналам проблема. я её по кусочкам с каких то левых форумов вытягивал. так что хз что про них можно указать.
----->
\end{document}


\begin{document}
 \newpage
\ESKDthisStyle{empty}

\begin{center}
 Министерство образования и науки Российской Федерации\\
 Федеральное государственное бюджетное образовательное учреждение высшего профессионального образования\\
 <<ТОМСКИЙ ГОСУДАРСТВЕННЫЙ УНИВЕРСИТЕТ СИСТЕМ УПРАВЛЕНИЯ И РАДИОЭЛЕКТРОНИКИ>> (ТУСУР)\\
 Кафедра комплексной информационной безопасности электронно-вычислительных систем (КИБЭВС)\\
\end{center}

\vfill

\begin{flushright}
\begin{minipage}{0.45\textwidth}
 \begin{flushleft}
  УТВЕРЖДАЮ\\
  заведующий каф. КИБЭВС
  \underline{\hspace{3cm}}А.А. Шелупанов \\
  <<\underline{\hspace{1cm}}>>\underline{\hspace{3cm}}2014г.\\
 \end{flushleft}
\end{minipage}
\end{flushright}

\vfill

\begin{center}
КОМПЬЮТЕРНАЯ ЭКСПЕРТИЗА

Отчет по групповому проектному обучению

Группа КИБЭВС-1208
\end{center}

\vfill
\begin{flushright}
\begin{minipage}{0.45\textwidth}
 \begin{flushleft}
  Ответственный исполнитель \\
  студент гр. 720-1 \\
  \underline{\hspace{3cm}}Д.С. Никифоров \\
  <<\underline{\hspace{1cm}}>>\underline{\hspace{3cm}}2014г.\\
 \end{flushleft}
\end{minipage}
\end{flushright}

\vfill

\begin{flushright}
\begin{minipage}{0.45\textwidth}
 \begin{flushleft}
  Научный руководитель \\
  аспирант каф. КИБЭВС \\
  \underline{\hspace{3cm}}А.И. Гуляев \\
  <<\underline{\hspace{1cm}}>>\underline{\hspace{3cm}}2014г.\\
 \end{flushleft}
\end{minipage}
\end{flushright}

\vfill

\begin{center}
 2014
\end{center}
 \newpage
\ESKDthisStyle{empty}
\paragraph{\hfill РЕФЕРАТ \hfill}
Курсовая работа содержит \ESKDtotal{page} страниц, \ESKDtotal{figure} рисунка, \ESKDtotal{table} таблицы, \ESKDtotal{bibitem} источников, \ESKDtotal{appendix} приложение.

%допилить ключевые слова
КОМПЬЮТЕРНАЯ ЭКСПЕРТИЗА, ФОРЕНЗИКА, ЛОГИ, QT, XML, GIT, LATEX, MOZILLA THUNDERBIRD, MS OUTLOOK, WINDOWS, PST, MSG, RTF, HTML, БИБЛИОТЕКИ, РЕПОЗИТОРИЙ, ПОЧТОВЫЙ КЛИЕНТ, SQLLITE, РЕЕСТР, МЕТА-ДАННЫЕ, READPST, JPEG, PNG, ID3V1, JFIF, RIFF, CHUNK, DBX, C++.

Цель работы --- создание программного комплекса, предназначенного для проведения компьютерной экспертизы.

Задачей, поставленной на данный семестр, стало написание программного комплекса, имеющего следующие возможности: 
\begin{enumerate}
\item \textbf{ПРАВИТЬ}
\end{enumerate}

Результаты работы в данном семестре:

\begin{itemize}
\item \textbf{ПРАВИТЬ}
\end{itemize}
5 Заключение
В этом семестре было проделано:
Исследование юнит-тестирования в инструментарии Qt;
Имплементация юнит-тестирования для пробной программы;
Имплементация юнит-тестирования для модуля, сканирующего медиа-файлы;

Пояснительная записка выполнена при помощи системы компьютерной вёрстки \LaTeX.

 
 \newpage
 \ESKDthisStyle{empty}
 \section*{Список исполнителей}
 
 \textbf{ПРАВИТЬ!!!}

Боков И.М. -- программист, ответственный за написание части системы для для нахождения медиа-файлов (аудио, видео, изображение) и извлечения мета-данных из них.
 
Кучер М.В. -- программист, ответственный за доработку программного модуля для определения операционной системы и создание репозитория проекта.
 
Лобанов О.В. -- программист, ответственный за разработку сайта проекта. 

Мейта М.В. -- программист, документатор, ответственный за ... верстку необходимой документации в системе \LaTeX.

Серяков А.В. -- программист, ответственный за написание части системы для сбора информации из почтового клиента MS Outlook и создание программного пакета с исходными файлами системы <<COEX>>.

Терещенко Ю.А. -- программист, ответственный за написание части системы для сбора информации из систем мгновенного обмена сообщениями.

Шиповской В.В. -- программист, ответственный за разработку графического интерфейса пользователя системы <<COEX>> и доработку программного модуля для нахождения медиа-файлов.




 
 % - содержание
 \newpage
 \ESKDstyle{plain}
 \tableofcontents

 \newpage
 \ESKDstyle{plain}
 \section*{Введение}
 \addcontentsline{toc}{section}{Введение}
 Компьютерно-техническая экспертиза является классом инженерно-\\технических экспертиз, проводимых в целях поиска криминалистически значимой информации на носителях, её всестороннего исследование, и, как следствие, получения доказательственной информации и установления фактов, имеющих значение для уголовных, гражданских и административных дел, сопряжённых с использованием компьютерных технологий. Для проведения компьютерных экспертиз необходима высокая квалификация экспертов, так как при изучении представленных носителей информации, попытке к ним доступа и сбора информации возможно внесение в информационную среду изменений или полная утрата важных данных.

Компьютерная экспертиза, в отличие от компьютерно-технической экспертизы, затрагивает только информационную составляющую, в то время как аппаратная часть и её связь с программной средой не рассматривается.

На протяжении предыдущих семестров нами были рассмотрены такие направления компьютерной экспертизы, как исследование файловых систем, сетевых протоколов, организация работы серверных систем, механизм журналирования событий. Также нами были изучены основные задачи, которые ставятся перед сотрудниками правоохранительных органов, которые проводят компьютерную экспертизу, и набор чуществующих утилит, способных помочь эксперту в проведении компьютерной экспертизы. Было выявлено, что существует множество разрозненных программ, предназначенных для просмотра лог-файлов системы и таких приложений, как мессенджеры и браузеры, но для каждого вида лог-файлов необходимо искать отдельную программу. Так как ни одна из них не позволяет эксперту собрать воедино и просмотреть все логи системы, браузеров и мессенджеров, было решено создать для этой цели собственный автоматизированный комплекс, которому на данный момент нет аналогов.


 \section{Назначение и область применения}
Разрабатываемый комплекс предназначен для автоматизации процесса сбора информации с исследуемого образа жёсткого диска.

\section{Постановка задачи}
\setcounter{figure}{0}
На данный семестр были поставлены следующие задачи:

\begin{itemize}
\item определить средства для разработки;
\item разработать архитектуру проекта;
\item реализовать структуру проекта;
\item определить единый формат выходных данных;
\item реализовать несколько модулей.
\end{itemize}

Задачи по проектированию модулей:

\begin{enumerate}
\item сбор и анализ истории переписки пользователей системы Windows из Pidgin и Skype;
\item написание автоматизированных модулей для сбора истории переписки пользователей системы Windows из таких программ как Pidgin и Skype;
\item сбор и анализ событий журнальных файлов Windows;
\item написание автоматизированного модуля для сбора событий журнальных файлов Windows.
\end{enumerate}


\section{Инструменты}
\setcounter{figure}{0}
\subsection{Система контроля версий Git}
Для разработки программного комплекса для проведения компьютерной экспертизы было решено использовать Git.

Git --- распределённая система управления версиями файлов. Проект был создан Линусом Торвальдсом для управления разработкой ядра Linux  как противоположность  системе управления версиями Subversion (также известная как «SVN»). \cite{progit}

При работе над одним проектом команде разработчиков необходим инструмент для совместного написания, бэкапирования и тестирования программного обеспечения. Используя Git, мы имеем:
\begin{itemize}
\item возможность удаленной работы с исходными кодами;
\item возможность создавать свои ветки, не мешая при этом другим разработчикам;
\item доступ к последним изменениям в коде, т.к. все исходники хранятся на сервере git.keva.su;
\item исходные коды защищены, доступ к ним можно получить лишь имея RSA-ключ;
\item возможность откатиться к любой стабильной стадии проекта.
\end{itemize}

Основные постулаты работы с кодом в системе Git:

\begin{itemize}
\item каждая задача решается в своей ветке;
\item необходимо делать коммит как только был получен осмысленный результат;
\item ветка master мержится не разработчиком, а вторым человеком, который производит вычитку и тестирование изменения;
\item все коммиты должны быть осмысленно подписаны/прокомментированы.
\end{itemize}

GitHub ---  крупнейший веб-сервис для хостинга IT-проектов и их совместной разработки. Основан на системе контроля версий Git и разработан на Ruby on Rails и Erlang компанией GitHub, Inc (ранее Logical Awesome).

Сервис абсолютно бесплатен для проектов с открытым исходным кодом и предоставляет им все возможности (включая SSL), а для частных проектов предлагаются различные платные тарифные планы.\cite{github} 

Создатели сайта называют GitHub <<социальной сетью для разработчиков>>. Кроме размещения кода, участники могут общаться, комментировать правки друг друга, а также следить за новостями знакомых. С помощью широких возможностей Git программисты могут объединять свои репозитории — GitHub предлагает удобный интерфейс для этого и может отображать вклад каждого участника в виде дерева.

Для проектов есть личные страницы, небольшие Вики и система отслеживания ошибок. Прямо на сайте можно просмотреть файлы проектов с подсветкой синтаксиса для большинства языков программирования. На платных тарифных планах можно создавать приватные репозитории, доступные ограниченному кругу пользователей.

Код проектов можно не только скопировать через Git, но и скачать в виде обычных архивов с сайта. (Для этого достаточно добавить /zipball/master/ в конец адресной строки.)

Кроме Git, сервис поддерживает получение и редактирование кода через SVN и Mercurial.

Для работы над проектом <<COEX>> проектной группой был поднят собственный репозиторий на сервере github.com.

Исходные файлы проекта и файлы для тестирования можно найти здесь:

https://github.com/tusur-coex.

\subsection{Система компьютерной вёрстки \TeX}
\input{technical_things/tex}
\subsection{Qt - кроссплатформенный инструментарий разработки ПО}
\input{technical_things/qt_doc}
\subsection{GitLab}




Для работы над проектом проектной группой был поднят собственный репозиторий на сервере git.keva.su.
Адреса репозиториев следующие:

\textbf{Исходные файлы проекта можно найти здесь:

http://gitlab2.keva.su/gpo/coex

Репозиторий для тестирования проекта:

git clone git@gitlab2.keva.su:gpo/coex.git}


\section{Технические характеристики}
\section {Технические характеристики}

%я не уверен, но помоему этот раздел надо бы подругому назвать =) 

\subsection {Постановка задачи}

На данный семестр были поставлены следующие задаи:

\begin{enumerate}
\item Определиться при помощи каких средств вести разработку
\item Разработать архитектуру проекта
\item Реализовать структуру проекта 
\item Определиться с форматом выходных данных
\item Реализовать несколько модулей
\end{enumerate}

\subsection {Выбор единого формата выходных файлов}

Для вывода результата был выбран формат XML-документов, так как с данным форматом лего работать при помощи программ, а результат работы данного комплекса в дальнейшем планируется обрабатывать при помощи программ

%и вот тут вброс про XML можно.

\subsubsubsubsection {Требования к аппаратному обеспечению}

Минимальные системные требования:\\
- Процессор 1ГГц Pentium 4;\\
- память 512 Мб;\\
- диск 9 Гб.

\subsubsubsubsection {Требования к программному обеспечению обеспечению}

На компьютерне должна быть установлена операционная система ubuntu 12.04 или выше, данная система должна иметь набор библиотек QT.
\subsection{Выбор единого формата выходных файлов}
Для вывода результата был выбран формат XML-документов, так как с данным форматом лего работать при помощи программ, а результат работы данного комплекса в дальнейшем планируется обрабатывать при помощи программ.

XML - eXtensible Markup Language или расширяемый язык разметки. Язык XML представляет собой простой и гибкий текстовый формат, подходящий в качестве основы для создания новых языков разметки, которые могут использоваться в публикации документов и обмене данными \cite{xml}. Задумка языка в том, что он позволяет дополнять данные метаданными, которые разделяют документ на объекты с атрибутами. Это позволяет упростить программную обработку документов, так как структурирует информацию.

Простейший XML-документ может выглядеть так:

\noindent <?xml version="1.0"?> \\
<list\_of\_items> \\
<item id="1"\textbackslash><first/>Первый</item\textbackslash> \\
<item id="2"\textbackslash>Второй <subsub\_item\textbackslash>подпункт 1</subsub\_item\textbackslash></item\textbackslash> \\
<item id="3"\textbackslash>Третий</item\textbackslash> \\
<item id="4"\textbackslash><last/\textbackslash>Последний</item\textbackslash> \\
</list\_of\_items\textbackslash>

Первая строка - это объявление начала XML-документа, дальше идут элементы документа <list\_of\_items> - тег описывающий начало элемента \\list\_of\_items, </list\_of\_items> - тег конца элемента. Между этими тегами заключается описание элемента, которое может содержать текстовую информацию или другие элементы (как в нашем примере). Внутри тега начала элемента так же могут указывать атрибуты элемента, как например атрибут id элемента item, атрибуту должно быть присвоено определенное значение.


\section{Разработка программного обеспечения}
\setcounter{figure}{0}
 
% \subsection{Архитектура}
% \subsection{Основной алгоритм}
В ходе разарботки был применен видоизменнённый шаблон проектирования Factory method.

%Описание шаблона и его модификации
Данный шаблон относится к классу порождающих шаблонов. Шаблоны данного класса - это шаблоны проектирования, которые абстрагируют процесс инстанцирования (создания экземпляра класса). Они позволяют сделать систему независимой от способа создания, композиции и представления объектов. Шаблон, порождающий классы, использует наследование, чтобы изменять инстанцируемый класс, а шаблон, порождающий объекты, делегирует инстанцирование другому объекту.
Основной алгоритм представлен на рисунке \ref{architech:architech}.

\begin{figure}[h!]
\center{\includegraphics[width=0.9\linewidth]{architech}}
\caption{Основной алгоритм}
\label{architech:architech}
\end{figure}

Использование данного шаблона позволило нам разбить наш проект на независимые модули, что весьма упростило задачу разработки, так как написание алгоритма для конкретного таска не влияло на остальную часть проекта. При разработке был реализован базовый класс для работы с образом диска. Данный клас предназначался для формирования списка настроек, определения операционной системы на смонтированном образе и инстанционировании и накапливание всех необходимых классов-тасков в очереди тасков. После чего каждый таск из очереди отправлялся на выполнение. Блоксхема работы алгоритма (тут картинка alg_main.eps)

Каждый класс-таск порождался путем наследования от базового абстрактного класса который имеет 8 методов и 3 атрибута:

\begin{enumerate}
\item QString manual() - возвращает справку о входных параметрах данного таска;
\item void setOption(QStringList list) - установка флагов для поданных на вход параметров;
\item QString command() - возвращает команду для инициализации такска вручную;
\item bool supportOS(const coex::typeOS \&os) - возвращает флаг, указывающий на возможность использования данного таска для конкретной операционной системы;
\item QString name() - возвращает имя данного таска;
\item QString description() - возвращает краткое описание таска;
\item bool test() - предназначена для теста на доступность таска;
\item bool execute(const coex::config \&config) - запуск таска на выполнение;
\item QString m\_strName - хранит имя таска;
\item QString m\_strDescription - хранит описание таска;
\item bool m\_bDebug - флаг для параметра --debug;
\end{enumerate}

На данный момент в проекте используется восемь классов. UML-диаграмма классов представлена на рисунке (тут картинка UML.eps)

Классы taskSearchSyslogsWin, taskSearchPidginWin и taskSearchSkypeWin - наследники от класса task являются тасками. Класс winEventLog и _EVENTLOGRECORD предназначины для конвертации журнальных файлов операционной системы Windows XP, а класс writerMessages для преобразования истории переписки.

 % --------Отчет Макса---------------%
% \newpage
% \subsection{Плагин SearchProgram}
% \subsubsection{Расположения файлов мессенджера Viber}
 
В зависимости от операционной системы (в дальнейшем ОС) у Viber разные пути установки. Для Windows XP: C:\textbackslash Documents and Settings\textbackslash \%Username\%\textbackslash Application Data\textbackslash ViberPC.
Для Windows 7, 8, 8.1, 10: C:\textbackslash Users\textbackslash \%Username\%\textbackslash AppData\textbackslash Roaming\textbackslash ViberPC (рис.~\ref{kucher_1:kucher_1}).
 
\begin{figure}[h!]
\center{\includegraphics[width=0.8\linewidth]{kucher_1}}
\caption{ Путь к файлам Viber }
\label{kucher_1:kucher_1}
\end{figure}

\subsubsection{Описание содержимого папки «ViberPC»}

При изучении содержимого папки «ViberPC» было обнаружено, что интересующая информация содержится в папках, название которых – номер телефона (рис.~\ref{kucher_2:kucher_2}).
	

\begin{figure}[h!]
\center{\includegraphics[width=0.8\linewidth]{kucher_2}}
\caption{ Содержимое папки с номером телефона }
\label{kucher_2:kucher_2}
\end{figure} 

\begin{itemize}
  \item Папка «Avatars» – содержит изображения пользователей;
  \item Папка «Thumbnails» – содержит все изображения, которые были отправлены и получены в ходе переписки;
	«viber.db» – база данных (далее БД), в которой хранится информация о контактах, переписках, звонках.
	БД «viber.db» – имеет формат SQLite format 3 (рис.~\ref{kucher_3:kucher_3}).
\end{itemize}
 
\begin{figure}[h!]
\center{\includegraphics[width=0.3\linewidth]{kucher_3}}
\caption{ Содержимое БД «viber.db» }
\label{kucher_3:kucher_3}
\end{figure} 

\subsubsection{SQL-запросы для получения информации}

Чтобы получить все контакты и их имена был написан следующий SQL-запрос: 
\textit{Select ContactRelation.Number, Contact.FirstName from СontactRelation, Contact where Contact.ContactID = ContactRelation.ContactID}.

Чтобы получить все контакты и имена, на которые можно позвонить в Viber, нужен следующий SQL-запрос: 
\textit{Select Contact.FirstName, ContactRelation.Number from contact, PhoneNumber, ContactRelation where PhoneNumber.IsViberNumber = 1 and PhoneNumber.Number = ContactRelation.Number and ContactRelation.ContactID = Contact.ContactID}.

Чтобы связать изображение пользователя с номером телефона и именем, нужен следующий SQL-запрос:
\textit{Select Contact.FirstName, ContactRelation.Number, OriginNumberInfo.AvatarPath From OriginNumberInfo, ContactRelation, Contact Where OriginNumberInfo.Number = ContactRelation.Number and ContactRelation.ContactID = Contact.ContactID}.

Для получения информации о звонках которые осуществлялись через Viber, нужен следующий запрос: 
\textit{select Contact.FirstName, Events.Direction, datetime(Events.TimeStamp, 'unixepoch') from Contact, Events, ContactRelation where Events.EventID = (select Calls.EventID from Calls) AND Events.Number = ContactRelation.Number and ContactRelation.ContactID = Contact.ContactID}.

Для получения текста переписки с конкретным пользователем нужно знать его номер чата. Для получения всех номеров чата нужно воспользоваться следующим запросом: 
\textit{Select ChatInfo.ChatID, Contact.FirstName, ChatInfo.TokenFrom ChatInfo, ContactRelation, Contactwhere ChatInfo.Token = ContactRelation.Number and ContactRelation.ContactID = Contact.ContactID}.

Зная номер чата, можно получить текст переписки: 
\textit{select Messages.Body, Contact.FirstName, Events.Direction, Messages.ThumbnailPath, datetime(Events.TimeStamp, 'unixepoch') from messages, Events, Contact, ContactRelation where Messages.EventID = Events.EventID and Events.Number = ContactRelation.Number and ContactRelation.ContactID = Contact.ContactID and Events.ChatID = @nomer\_chata}.

\subsubsection{Описание плагина}

Плагин «TaskViber» получает точку монтирования жесткого диска, с которого, в отличии от ОС, проверяет папку «ViberPC» у всех пользователей в ОС. Если папка «ViberPC» существует, то плагин извлекает информацию из аккаунтов, под которыми авторизовались с данного компьютера. Всю найденную информацию плагин сохраняет по указанному пути программного обеспечения «COEX». В папку «Avatars» (рис.~\ref{kucher_9:kucher_9}) копируются все найденные изображения пользователей. В папку «Thumbnails» (рис.~\ref{kucher_10:kucher_10}) копируются все изображения, которые были отправлены и получены в ходе переписки. В файле «Avatar Path.txt» находятся связи между изображениями пользователей, именами и номерами телефонов. В файле «Calls.txt» находятся описание звонков, которые осуществлялись через «Viber» (с кем был звонок, во сколько и кто кому звонил). В файле «Phone book.txt» находятся все номера телефонов и имена с мобильного телефона, на котором был зарегистрирован аккаунт в «Viber». В файле «Viber book.txt» находятся все номера телефонов и имена, которым можно позвонить через «Viber». В файле «Имя/номер messages.txt» содержится переписка с пользователем «Имя/номер» (с кем велась переписка, кто кому писал, что писал и во сколько писал).

Блок-схема алгоритма работы плагина «TaskViber» представлена на рисунке~\ref{kucher_4:kucher_4}, блок-схема функции WinXP --- на рисунке~\ref{kucher_5:kucher_5}. Ниже также представлены блок-схемы Win\_7\_8\_10 (рис.~\ref{kucher_6:kucher_6}) и Viber\_XP\_7\_8\_10 (рис.~\ref{kucher_7:kucher_7}).

Результат работы плагина представлен на рисунке~\ref{kucher_8:kucher_8}.
 

\begin{figure}[h!]
\center{\includegraphics[width=0.6\linewidth]{kucher_4}}
\caption{ Блок-схема алгоритма работы плагина }
\label{kucher_4:kucher_4}
\end{figure} 
  
\begin{figure}[h!]
\center{\includegraphics[width=0.6\linewidth]{kucher_5}}
\caption{ Блок-схема функции WinXP }
\label{kucher_5:kucher_5}
\end{figure} 

\begin{figure}[ht]
\center{\includegraphics[width=0.6\linewidth]{kucher_6}}
\caption{ Блок-схема Win\_7\_8\_10 }
\label{kucher_6:kucher_6}
\end{figure} 

\begin{figure}[h!]
\center{\includegraphics[width=0.5\linewidth]{kucher_7}}
\caption{ Блок-схема Viber\_XP\_7\_8\_10 }
\label{kucher_7:kucher_7}
\end{figure} 

\clearpage

\begin{figure}[h!]
\center{\includegraphics[width=1\linewidth]{kucher_8}}
\caption{ Результат работы плагина }
\label{kucher_8:kucher_8}
\end{figure} 

\begin{figure}[h!]
\center{\includegraphics[width=1\linewidth]{kucher_9}}
\caption{ Содержимое папки «Avatars» }
\label{kucher_9:kucher_9}
\end{figure} 

\begin{figure}[h!]
\center{\includegraphics[width=0.8\linewidth]{kucher_10}}
\caption{ Содержимое папки «Thumbnails» }
\label{kucher_10:kucher_10}
\end{figure} 

\clearpage


 % ---------Отчет Юры----------------%
% \newpage
% \subsection{ Доработка программного модуля TaskFirefoxWin }
% Цель работы:
\begin{enumerate}
  \item исправить ошибки плагина TaskFirefoxWin;
  \item переделать программный модуль TaskThunderbirdWin в плагин для проекта coex.
\end{enumerate}

\subsubsection{TaskFirefoxWin}

Плагин TaskFirefoxWin предназначен для сбора истории посещений браузера Mozilla Firefox. История находится в файле базы данных places.sqlite, который расположен в C:\textbackslash Users\textbackslash User\textbackslash AppData\textbackslash Roaming\textbackslash Mozilla\textbackslash Firefox\textbackslash Profiles\textbackslash profilename.

Плагин выполняет рекурсивный обход директорий пока не найдет файл places.sqlite. Затем плагин подключается к базе данных, содержащейся в данном файле и выполняет sql-запрос на выборку данных из таблицы.
~\ref{teresh_1:teresh_1}

\begin{figure}[h!]
\center{\includegraphics[width=0.2\linewidth]{teresh_1}}
\caption{ Блок-схема алгоритма TaskFirefoxWin }
\label{teresh_1:teresh_1}
\end{figure}

Программный модуль TaskFirefoxWin был некорректно переделан в плагин проекта coex. Вследствие чего, он не исполнялся. В результате тестирования было выявлено, что причиной некорректной работы плагина были служебные файлы TaskPlugin.pro, build.sh и taskFirefox.h. Были внесены исправления в эти служебные файлы. После этого плагин исполнялся.
~\ref{teresh_2:teresh_2}

\begin{figure}[h!]
\center{\includegraphics[width=0.6\linewidth]{teresh_2}}
\caption{ Изменения в служебных файлах }
\label{teresh_2:teresh_2}
\end{figure}

Были добавлены тестовые данные для проверки работоспособности плагина. Также был добавлен вывод результатов работы плагина в XML. В результате тестирования было выявлено, что плагин не выполняет поставленную задачу.

В одном из обновлений Mozilla Firefox была изменена структура файла places.sqlite. Из-за этого не работал sql-запрос для выборки данных из таблицы с историей посещений. Был составлен новый запрос: select * from moz\_places.
~\ref{teresh_3:teresh_3}

\begin{figure}[h!]
\center{\includegraphics[width=0.7\linewidth]{teresh_3}}
\caption{ Структура базы данных и нужной таблицы }
\label{teresh_3:teresh_3}
\end{figure}

Было произведено тестирование исправленного плагина. Из представленных исходных данных был получен такой отчет в XML-файле.
~\ref{teresh_4:teresh_4}
~\ref{teresh_5:teresh_5} 

\begin{figure}[h!]
\center{\includegraphics[width=0.5\linewidth]{teresh_4}}
\caption{ Тестовые данные }
\label{teresh_4:teresh_4}
\end{figure}

\begin{figure}[h!]
\center{\includegraphics[width=0.6\linewidth]{teresh_5}}
\caption{ Результат тестирования TaskFirefoxWin }
\label{teresh_5:teresh_5}
\end{figure}

\subsubsection{TaskThunderbirdWin}

TaskThunderbirdWin был выполнен в виде отдельного программного модуля. Модуль предназначен для сбора сообщений и представления их в формате XML. Сообщения хранятся в файле Inbox.mbox для входящих сообщений и Sent.mbox для исходящих сообщений. Путь, по которому находятся файлы: C:\textbackslash Users\textbackslash User\textbackslash AppData\textbackslash Roaming\textbackslash Thunderbird\textbackslash Profiles\textbackslash profile\_name\textbackslash Mail\textbackslash server\_name. mbox представляет собой текстовый файл, в котором хранятся все сообщения почтового ящика. Начало почтового сообщения определяется строкой из 5 символов: словом <<From>> с последующим пробелом.

После открытия файл mbox разделяется на отдельные сообщения с помощью регулярного выражения <<(From \textbackslash \textbackslash r\textbackslash \textbackslash n)|(From \textbackslash \textbackslash n\textbackslash \textbackslash r)|From \textbackslash \textbackslash r|From \textbackslash \textbackslash n>>. Затем к каждому сообщению применяются регулярные выражения:

\begin{itemize}
  \item <<\textbackslash \textbackslash nDate: ([\textasciicircum \textbackslash \textbackslash n]*)\textbackslash \textbackslash n>> --- время отправки/приема сообщения;
  
  \item <<\textbackslash \textbackslash nFrom: .*([a-z][\textbackslash \textbackslash w\textbackslash \textbackslash .]*\textbackslash \textbackslash w@\textbackslash \textbackslash w[\textbackslash \textbackslash w\textbackslash \textbackslash .]*\textbackslash \textbackslash .\textbackslash \textbackslash w*).*\textbackslash \textbackslash nUser-Agent:>> --- кто отправил сообщения;
  
  \item <<\textbackslash \textbackslash nTo: .*([a-z][\textbackslash \textbackslash w\textbackslash \textbackslash .]*\textbackslash \textbackslash w@\textbackslash \textbackslash w[\textbackslash \textbackslash w\textbackslash \textbackslash .]*\textbackslash \textbackslash .\textbackslash \textbackslash w*).*\textbackslash \textbackslash nSubject:>> --- кто получил сообщение;
  
  \item <<\textbackslash \textbackslash nContent-Transfer-Encoding: 8bit\textbackslash \textbackslash s*(\textbackslash \textbackslash S.*\textbackslash \textbackslash S)\textbackslash \textbackslash s*[0-3]\textbackslash \textbackslash d\textbackslash \textbackslash .[01]\textbackslash \textbackslash d\textbackslash \textbackslash .\textbackslash \textbackslash d{4} [0-2]\textbackslash \textbackslash d:[0-5]\textbackslash \textbackslash d, [\textasciicircum \textbackslash \textbackslash n]*\textbackslash \textbackslash n>> --- текст сообщения.
\end{itemize}

~\ref{teresh_6:teresh_6}

\begin{figure}[h!]
\center{\includegraphics[width=0.6\linewidth]{teresh_6}}
\caption{ Блок-схема алгоритма TaskThunderbirdWin }
\label{teresh_6:teresh_6}
\end{figure}

Как отдельный модуль, TaskThunderbirdWin выполнял поставленные перед ним задачи.
~\ref{teresh_7:teresh_7}

\begin{figure}[h!]
\center{\includegraphics[width=0.9\linewidth]{teresh_7}}
\caption{ Результат тестирования TaskThunderbirdWin }
\label{teresh_7:teresh_7}
\end{figure}

Но после того, как TaskThunderbirdWin был переписан под архитектуру coex, возникли проблемы в работе данного модуля. Ведется тестирование с целью нахождения ошибок, мешающих корректному выполнению данного модуля.

\clearpage


 % ---------Отчет Влада--------------%
% \newpage
% \subsection{ Написание графического интерфейса пользователя для системы <<COEX>> }
% В графическом интерфейсе необходимо было предусмотреть выбор директорий для работы системы и вывод информации.

Для создания интерфейса использовался Qt Designer --- кроссплатформенная свободная среда для разработки графических интерфейсов (GUI, англ. <<Graphic User Interface>>) программ, использующих библиотеку Qt. Входит в состав Qt framework. При разработке использовался механизм слотов. Сигналы и слоты --- подход, который позволяет реализовать шаблон <<наблюдатель>>, минимизируя написание повторяющегося кода. Концепция заключается в том, что компонент может посылать сигналы, содержащие информацию о событии. В свою очередь другие компоненты могут принимать эти сигналы посредством специальных функций --- слотов. Данный механизм является основной идеей в QT и подходит для описания графического интерфейса пользователя.

Так как система <<COEX>> является консольной утилитой, то графический интерфейс генерирует строку с необходимыми параметрами и запускает с ними консольную утилиту, перехватывая вывод и отображая его в интерфейсе. В качестве параметров выступают директории для работы системы. Главное окно представлено на рисунке~\ref{ship_1:ship_1}.

\begin{figure}[h!]
\center{\includegraphics[width=0.5\linewidth]{ship_1}}
\caption{ Главное окно интерфейса }
\label{ship_1:ship_1}
\end{figure}

Интерфейс состоит из следующих элементов:
\begin{enumerate}
  \item кнопка <<Открыть папку>> --- отвечает за выбор папки, в которой будет производиться поиск;
  \item кнопка <<Сохранить в папку>> --- отвечает за выбор папки для сохранения результатов; 
  \item в данном окне отображается вывод процесса выполнения;
  \item кнопка <<Запуск>> --- отвечает за запуск системы;
  \item окно, в котором находятся текущие плагины для выполнения в виде списка. 
\end{enumerate}

Если не были выбраны директории для работы, то при нажатии на кнопку <<Запуск>> отобразится соответствующее сообщение (рис.~\ref{ship_2:ship_2}). Диалоговое окно выбора директории представлено на рисунке~\ref{ship_3:ship_3}. Процесс выполнения программы представлен на рисунке~\ref{ship_4:ship_4}. По завершении работы системы отобразится соответствующие сообщение (рис.~\ref{ship_5:ship_5}).

\begin{figure}[h!]
\center{\includegraphics[width=0.5\linewidth]{ship_2}}
\caption{ Сообщение об ошибке }
\label{ship_2:ship_2}
\end{figure}

\begin{figure}[h!]
\center{\includegraphics[width=0.5\linewidth]{ship_3}}
\caption{ Выбор директории }
\label{ship_3:ship_3}
\end{figure}

\begin{figure}[h!]
\center{\includegraphics[width=0.5\linewidth]{ship_4}}
\caption{ Процесс выполнения программы }
\label{ship_4:ship_4}
\end{figure}

\begin{figure}[h!]
\center{\includegraphics[width=0.5\linewidth]{ship_5}}
\caption{ Сообщение о выполнении }
\label{ship_5:ship_5}
\end{figure}

\clearpage





% \newpage
% \subsection{ Доработка программного модуля Media Scanner }
% Программный модуль media scanner осуществляет нахождение медиа-файлов (аудио, видео, изображение) и извлечение мета-данных с записью их в XML-формат. Данный программный модуль не соответствовал следующим условиям, при которых система распознает его как плагин:
\begin{itemize}
  \item директория для временных файлов /tmp;
  \item исходный код должен находится в папке /src;
  \item *.pro файл должен именоваться Task*.pro;
  \item файлы *.cpp и *.h должны соответствовать шаблону TaskExample.
\end{itemize }

После выполения данных условий система <<COEX>> успешно распознает, собирает (рис.~\ref{ship_6:ship_6}) и выполняет данный плагин (рис.~\ref{ship_7:ship_7}). Результат выполнения программного модуля представлен на рисунке~\ref{ship_8:ship_8}.

\begin{figure}[h!]
\center{\includegraphics[width=0.5\linewidth]{ship_6}}
\caption{ Успешная сборка плагина }
\label{ship_6:ship_6}
\end{figure}

\begin{figure}[h!]
\center{\includegraphics[width=0.7\linewidth]{ship_7}}
\caption{ Выполнение плагина }
\label{ship_7:ship_7}
\end{figure}

\begin{figure}[h!]
\center{\includegraphics[width=0.9\linewidth]{ship_8}}
\caption{ Результат выполнения плагина }
\label{ship_8:ship_8}
\end{figure}

 
 % ---------Отчет Андрея--------------%
% \newpage
% \subsection{ Cбор информации из почтового клиента MS Outlook }
% В начале семестра был переписан модуль <<Outlook>> под новую архитектуру <<COEX>>.
Добавлены наследуемые функции для описания плагина и функции для контроля выполнения плагина в программном комплексе <<COEX>> из главного класса <<task>>.

Функции для описания плагина и функции для контроля выполнения плагина представлены на рисунке~\ref{Outlook:Outlook}, функция для контроля выполнения плагина в программном комплексе <<COEX>> --- на рисунке~\ref{Outlook2:Outlook2}.

\begin{figure}[h!]
\center{\includegraphics[width=0.6\linewidth]{Outlook}}
\caption{ Функции для описания плагина и функции для контроля выполнения плагина }
\label{Outlook:Outlook}
\end{figure}

\begin{figure}[h!]
\center{\includegraphics[width=0.6\linewidth]{Outlook2}}
\caption{ Функция для контроля выполнения плагина в программном комплексе }
\label{Outlook2:Outlook2}
\end{figure}




% \newpage
% \subsection { Создание <<бинарного>> пакета DEB из исходных файлов программного комплекса <<COEX>> } 
% Формат *.deb --- расширение имён файлов <<бинарных>> пакетов для распространения и установки программного обеспечения в ОС проекта Debian и других, использующих систему управления пакетами dpkg. Пакеты Debian содержат исполняемые и конфигурационные файлы, номер версии формата, информацию об авторских правах, а также другую документацию, необходимую для установки программы из пакета.

Основные составляющие *.deb-пакета, необходимые для его создания:
\begin{itemize}
  \item control;
  \item md5sums;
  \item changelog;
  \item rules;
  \item README;
  \item conffiles;
  \item dirs;
  \item watch.
\end{itemize}

Control --- центральный файл пакета (обязательный), описывающего все основные свойства. Файл --- текстовый, состоящий из пар <<Атрибут: значение>>.

Md5sums --- содержит md5-хеши для всех файлов, кроме файлов, находящихся в каталоге DEBIAN/. Данный файл необязателен для deb-пакета, 
однако программы верификации пакетов считают пакеты, не содержащие этот файл, ошибочными. Может использоваться некоторыми программами администрирования системы для верификации изменений в файловой системе. Осуществляет контроль целостности файлов.

Changelog --- обязательный файл, его специальный формат описан в руководстве по политике Debian, раздел 4.4 <<debian/changelog>>. Этот формат используется программой dpkg и другими для получения информации о номере версии, редакции, разделе и срочности пакета.~\cite{deb_package_howto}

Rules --- используется для управления компиляцией пакета. 

README --- в этот файл записывается любая дополнительная информация, а также различия между программой из пакета Debian и исходной программой.

Conffiles --- Обычно пакеты содержат <<болванки>> конфигурационных файлов, например, размещаемых в /etc. Очевидно, что если конфигурационный файл в пакете обновляется, пользователь потеряет свой отредактированный конфигурационный файл. Эта проблема легко решается использованием папок типа <<config.d>>, содержимое которых включается в основной конфигурационный файл, заменяя собой повторяющиеся опции.~\cite{deb_man}

Dirs --- в этом файле указываются каталоги, которые необходимы для обычной установки (make install DESTDIR=..., вызываемая dh\_auto\_install), но которые автоматически не создаются. Обычно, это указывает на проблему в Makefile.

Watch --- формат файла watch описан в справочной странице uscan(1). Файл watch настраивает программу uscan (предоставляется пакетом devscripts) для слежения за сайтами, с которых вы скачали исходный код. Он также используется службой Debian для слежения за состоянием внешних источников (DEHS).~\cite{deb_package_howto}

Для решения данной  задачи было сделано:

\begin{itemize}
\item изучен *.deb формат, особенности файлов стандартов необходимых для создания данного пакета и установки его в операционные системы семейства Unix; 
\item изучены программные продукты для работы с данным форматом (dpkg, build-essential,  quilt, fakeroot, devscripts, debhelper и dh-make, lintian);
\item изучены особенности написания скриптов bash для операционных систем Unix;
\item ознакомление и изучение программы make, набора инструкцией Makefile и программной утилиты для QT C++ qmake;
\item углубленное изучение системы построения программного комплекса <<COEX>> и зависимых библиотек для компиляции данного программного комплекса;
\item создание двух программ bash-скриптов для формирования *.deb пакета и формирования changelog на основе истории коммитов системы контроля версий GIT;
\item тестирование на <<чистой>> операционной системе Linux Mint 17.2;
\item изучен формат набора макрорасширений системы компьютерной вёрстки LaTeX, личный отчет по групповому проектному обучению был оформлен согласно данному формату, и отдан документатору для редакции и включения в общий отчет.
\end{itemize}

\subsubsection{ Подробное рассмотрение набора bash-скриптов для формирования *.deb пакета и формирования changelog}

В начале программы с помощью скрипта build.sh компилируется проект и осуществляется сборка бинарного файла <<coex>>, а также динамических библиотек, используемых программным комплексом <<COEX>>. Исполняемые файлы плагинов данного проекта имеют формат *.so. 

В корне пакета создается папка <<DEBIAN>>. Эта папка содержит управляющую генерацией пакета информацию и не копируется на диск при установке пакета.
Также корневая папка пакета содержит будущий <<корень диска>>: при установке пакета все файлы (кроме папки <<debian>>) распаковываются в корень /. Поэтому бинарный файл должен лежать по пути относительно корня пакета <<usr/bin/coex>>, что и было сделано в скрипте create\_package.sh (рис.~\ref{cpcoex:cpcoex}). 

\begin{figure}[h!]
\center{\includegraphics[width=0.5\linewidth]{cpcoex}}
\caption{Копирование основного бинарного файла проекта}
\label{cpcoex:cpcoex}
\end{figure}

Исполняемые файлы плагинов и программные библиотеки программного комплекса <<COEX>> представлены на рисунке~\ref{PluginsAndLibs:PluginsAndLibs}.

\begin{figure}[h!]
\center{\includegraphics[width=0.6\linewidth]{PluginsAndLibs}}
\caption{Исполняемые файлы плагинов и программные библиотеки программного комплекса <<COEX>>}
\label{PluginsAndLibs:PluginsAndLibs}
\end{figure}

Установка иконки продукта нужна для графического интерфейса пользователя программного комплекса <<COEX>> (рис.~\ref{image:image}). Файл coex.desktop служит для запуска приложения, содержит основную информацию о программном комплексе <<COEX>> (рис.~\ref{Aplicatio:Aplicatio}).

\begin{figure}[h!]
\center{\includegraphics[width=0.6\linewidth]{image}}
\caption{ Иконка для GUI приложения <<COEX>> }
\label{image:image}
\end{figure}


\begin{figure}[h!]
\center{\includegraphics[width=0.6\linewidth]{Aplicatio}}
\caption{ Файл coex.desktop <<COEX>> }
\label{Aplicatio:Aplicatio}
\end{figure}

Лицензия на программный комплекс <<COEX>> защищает права разработчиков данного программного обеспечения. Руководство по использованию программного комплекса <<COEX>> находится в стадии разработки и будет содержать основные команды для использования программного комплекса <<COEX>>, информацию о плагинах и об особенностях работы с ними (рис.~\ref{LIcenzMan:LIcenzMan}).

\begin{figure}[h!]
\center{\includegraphics[width=0.6\linewidth]{LIcenzMan}}
\caption{ Лицензия и мануал по <<COEX>> }
\label{LIcenzMan:LIcenzMan}
\end{figure}

Файл control (рис.~\ref{control:control}) --- обязательный файл, в котором прописана версия программного комплекса <<COEX>>. Также указаны зависимости от библиотек, нужных для использования <<COEX>>.

\begin{figure}[h!]
\center{\includegraphics[width=0.8\linewidth]{control}}
\caption{ Файл control }
\label{control:control}
\end{figure}

Файл changelog (рис.~\ref{chengelog:chengelog}) --- в данном файле находятся сведения о всех изменениях программного комплекса <<COEX>> и версий пакета. Данный файл генерируется в отдельном скрипте gen\_change\_log.sh рисунок 10

\begin{figure}[h!]
\center{\includegraphics[width=0.8\linewidth]{chengelog}}
\caption{ Файл chengelog }
\label{chengelog:chengelog}
\end{figure}

Файл md5sums (рис.~\ref{md5sums:md5sums}) --- в данном файле записываются хеш-значения всех файлов, находящихся в <<debian/usr>>.

\begin{figure}[h!]
\center{\includegraphics[width=0.6\linewidth]{md5sums}}
\caption{ Файл md5sums }
\label{md5sums:md5sums}
\end{figure}

Далее, чтобы распаковать *.deb-пакет для установки на компьютер пользователя, необходимо всем файлам задать права доступа root, иначе пакет не установится, после чего утилитой dpkg собрать из всех созданных файлов пакет формата *.deb. 

\subsubsection{Тестирование созданного пакета}

После создания *.deb-пакета для его тестирования выбраны были три образа операционных систем семейства Unix, а именно: Linux Mint 17.2, Ubuntu 14.04, Debian 8.2. На каждой из операционных систем был установлен пакет coex\_0.1-43-g59b4cc1\_all.deb. Программное обеспечение было успешно установлено, все плагины работают (рис.~\ref{debian:debian}).

\begin{figure}[h!]
\center{\includegraphics[width=0.8\linewidth]{debian}}
\caption{ Бинарный файл coex на debian}
\label{debian:debian}
\end{figure}

Библиотеки и исполняемые файлы плагинов coex на debian --- рисунок~\ref{debian2:debian2}.

\begin{figure}[h!]
\center{\includegraphics[width=0.9\linewidth]{debian2}}
\caption{ Библиотеки и исполняемый файлы плагинов coex на debian}
\label{debian2:debian2}
\end{figure}

Бинарный файл coex на Linux Mint --- рисунок~\ref{Linux_Mint:Linux_Mint}.

\begin{figure}[h!]
\center{\includegraphics[width=0.6\linewidth]{Linux_Mint}}
\caption{ Бинарный файл coex на Linux Mint}
\label{Linux_Mint:Linux_Mint}
\end{figure}

Плагины на Linux Mint --- рисунок~\ref{Plugins:Plugins}.

\begin{figure}[h!]
\center{\includegraphics[width=0.6\linewidth]{Plugins}}
\caption{ Бинарный файл coex на Linux Mint}
\label{Plugins:Plugins}
\end{figure}

Запуск плагина на Linux Mint --- рисунок~\ref{Zapusk:Zapusk}.

\begin{figure}[h!]
\center{\includegraphics[width=0.6\linewidth]{Zapusk}}
\caption{ Запуск плагина на Linux Mint}
\label{Zapusk:Zapusk}
\end{figure}

Бинарный файл coex на Ubuntu --- рисунок~\ref{ubuntu:ubuntu}.

\begin{figure}[h!]
\center{\includegraphics[width=0.8\linewidth]{ubuntu}}
\caption{ Бинарный файл coex на Ubuntu 15.04 }
\label{ubuntu:ubuntu}
\end{figure}

Запуск coex на Ubuntu --- рисунок~\ref{ubuntu2:ubuntu2}.

\begin{figure}[h!]
\center{\includegraphics[width=0.8\linewidth]{ubuntu2}}
\caption{ Запуск coex на Ubuntu 15.04 }
\label{ubuntu2:ubuntu2}
\end{figure}

\clearpage









%  % ---------Отчет Олега---------------%
% \newpage 
% \subsection{ Разработка веб-сайта проекта <<COEX>> }
% \subsubsection{ Дизайн веб-сайта проекта }

Трудно себе представить в современном мире компанию или программный продукт без персональной страницы в сети Интернет. В связи с этим возникла необходимость создания веб-сайта для проекта <<COEX>>.

Современные тенденции в веб-дизайне, как и в приложениях: минимализм и простота функционала. Крупнейшие корпорации в течение последних лет выпустили гайдлайны для своих продуктов: у Microsoft --- ModerUI, у Google --- Flat Design.
В качестве основного направления для дизайна выбран плоский стиль с минимум сторонней графики. Используемые технологии: HTML5, CSS3, JavaScript.
Не менее важным является структура будущего макета, для этого необходимо выделить основные блоки информации, которые будут на сайте:
\begin{enumerate}
  \item Информация о системе. Здесь необходимо описать, что представляет из себя система <<COEX>> и ее необходимость.
  \item Возможности системы, описание плагинов.
  \item Установка. Как установить систему, ссылки на скачивание.
  \item Архитектура. Информация о структуре системы.
  \item Контакты.
\end{enumerate}

Исходя из вышесказанного был выбран первый вариант макета: сайт-презентация (рис.~\ref{lo1:lo1}).

\begin{figure}[h!]
\center{\includegraphics[width=0.7\linewidth]{lo1}}
\caption{ Первый вариант дизайна }
\label{lo1:lo1}
\end{figure}

Позднее данный дизайн был отклонён из-за нерационального использования пространства веб-страницы и сложности расположение контента в пользу нового варианта (рис.~\ref{lo2:lo2}).

\begin{figure}[h!]
\center{\includegraphics[width=0.8\linewidth]{lo2}}
\caption{ Новый вариант дизайна }
\label{lo2:lo2}
\end{figure}

Был выбран вариант с одним блоком информации на одну страницу <<прокрутки>> --- так можно акцентировать внимание посетителя на необходимых деталях.

\subsubsection{ Вёрстка дизайна } 

Как было сказано ранее, для верстки использовались технологии HTML5 и CSS3, что позволило существенно сократить код стилей и нагляднее писать код страницы. Минусом данного подхода является требование к наличию у посетителей последних версий интернет-браузеров, но, как показывает статистика, количество таких пользователей превалирует (рис.~\ref{lo3:lo3}).

\begin{figure}[h!]
\center{\includegraphics[width=0.8\linewidth]{lo3}}
\caption{ Количество посетителей с различными браузерами }
\label{lo3:lo3}
\end{figure}

Для обеспечения поддержки мобильных устройств и устройств с нестандартным разрешение экрана в вёрстке максимально использованы новые возможности CSS3, такие как flex-box, box-sizing и другие (рис.~\ref{lo4:lo4}).

\begin{figure}[h!]
\center{\includegraphics[width=0.8\linewidth]{lo4}}
\caption{ Часть кода CSS-стилей }
\label{lo4:lo4}
\end{figure}

\subsubsection{ Регистрация домена }

Был подобран и зарегистрирован домен coex.su, параллельно с этим изучены принципы построения DNS серверов и NX записи (рис.~\ref{lo5:lo5}).

\begin{figure}[h!]
\center{\includegraphics[width=0.8\linewidth]{lo5}}
\caption{ Пример NX-записей для домена coex.su }
\label{lo5:lo5}
\end{figure}

\subsubsection{ Настройка WEB-сервера }

Для сайта был запущен и настроен веб-сервер apache в связке с nginx. В дальнейшем apache будет <<отдавать>> посетителям динамический контент, такой как новости, а nginx – статический: исходные коды программы с репозитория, картинки, видео и прочее (рис.~\ref{lo6:lo6}).

\begin{figure}[h!]
\center{\includegraphics[width=0.8\linewidth]{lo6}}
\caption{ Часть конфига Apache }
\label{lo6:lo6}
\end{figure}

\subsubsection{ Контент }

Для наполнения сайта были написаны тексты, описывающие систему, ее части и возможности. Для каждого плагина была подготовлена своя иконка (рис.~\ref{lo7:lo7}) и описание.

\begin{figure}[h!]
\center{\includegraphics[width=0.7\linewidth]{lo7}}
\caption{ Иконки для плагинов }
\label{lo7:lo7}
\end{figure}

\clearpage










 % ---------Отчет Ильи---------------%
% \newpage
% \subsection{ Модульное тестирование } 
% \subsection{Документирование плагинов}

На данный момент основной репозиторий находится на ресурсе GitHub. Данный ресурс использует язык разметки MarkDown (подробнее в разделе~\ref{sec:markdown}) и автоматически добавляет файл <<Readme.md>> к описанию программного модуля, если этот файл присутствует. В связи с этим было решено создать документацию плагинов, используя MarkDown. Документация должна включать в себя:

\begin{enumerate}
  \item Название плагина;
  \item Версию плагина;
  \item Автора плагина;
  \item Описание плагина;
  \item Требуемую операционную систему;
  \item Версию ПО, с которым этот плагин работает;
  \item Основные методы плагина с описанием входов и выходов.
\end{enumerate}

Результат разработанной документации можно наблюдать на странице плагина в репозитории (рис.~\ref{bok_1:bok_1}).

\begin{figure}[!ht]
\center{\includegraphics[width=0.8\linewidth]{bok_1}}
\caption{ Документация в веб-интерфейсе репозитория }
\label{bok_1:bok_1}
\end{figure}

Поскольку проект <<COEX>> имеет свою собственную веб-страницу, данную документацию также необходимо преобразовать в формат HTML, чтобы затем добавить на веб-страницу проекта. Для преобразования был разработан небольшой скрипт на языке Python (приложение ~\ref{apx:mdtohtml}). На рисунке ~\ref{bok_2:bok_2} можно наблюдать ту же документацию, но в формате HTML.

\begin{figure}[!ht]
\center{\includegraphics[width=0.9\linewidth]{bok_2}}
\caption{ Документация в формате HTML }
\label{bok_2:bok_2}
\end{figure}

\clearpage
\subsection{Разработка и внедрение копии жесткого диска}

Поскольку большое количество плагинов <<COEX>> обращается к жесткому диску для поиска тех или иных файлов, что в свою очередь создает серьезную нагрузку на него, то было решено модифицировать архитектуру проекта с целью хранения копии информации о жестком диске. Данную информацию решено было хранить как поле объекта <<config>>, к которому будут обращаться остальные плагины. Поле представляет из себя класс <<Hdd>> с атрибутом типа QList<QDir> (приложение ~\ref{apx:hddclass}). Данный тип был выбран, поскольку он позволяет хранить данные о всех директориях и файлах внутри них, а также предоставляет удобные интерфейсы для доступа к ним. Методы класса <<Hdd>>:

\begin{enumerate}
  \item Hdd::Hdd(QString path);
  \item Hdd::~Hdd();
  \item QFileInfoList getFiles(QStringList wildcardlist);
  \item QFileInfoList getFiles(QString wildcard).
\end{enumerate}

Метод <<Hdd::Hdd(QString path)>> является конструктором класса. Переменная <<path>>, подаваемая на вход метода является путем до начальной папки. Конструктор с помощью экземпляра класса <<QDirIterator>> посещает каждую папку в начальной папке и сохраняет данные о ней в переменную типа <<QDir>>, после чего добавляет эту переменную к массиву <<QList<QDir>>>, и наконец сохраняет полученный массив как поле класса. Алгоритм конструктора можно увидеть на рисунках~\ref{bok_6:bok_6} и~\ref{bok_7:bok_7}.

\begin{figure}[!ht]
\center{\includegraphics[width=0.3\linewidth]{bok_6}}
\caption{ Алгоритм конструктора класса <<Hdd>> }
\label{bok_6:bok_6}
\end{figure}

\begin{figure}[!ht]
\center{\includegraphics[width=0.5\linewidth]{bok_7}}
\caption{ Продолжение алгоритма конструктора класса <<Hdd>> }
\label{bok_7:bok_7}
\end{figure}

Метод <<Hdd::~Hdd()>> является деструктором класса.

Метод <<QFileInfoList getFiles(QStringList wildcardlist)>> возвращает объект <<QFileInfoList>> для всех файлов, которые соответствуют заданному массиву масок <<wildcardlist>>. Алгоритм метода можно увидеть на рисунке ~\ref{bok_9:bok_9}.

\begin{figure}[!ht]
\center{\includegraphics[width=0.3\linewidth]{bok_9}}
\caption{ Алгоритм метода <<getFiles>> класса <<Hdd>> }
\label{bok_9:bok_9}
\end{figure}

Метод <<QFileInfoList getFiles(QString wildcard)>> выполняет ту же функцию, что и прошлый метод. Он является перегрузкой прошлого метода и принимает на вход одну маску вместо массива. Алгоритм метода можно увидеть на рисунке ~\ref{bok_8:bok_8}.

\begin{figure}[!ht]
\center{\includegraphics[width=0.3\linewidth]{bok_8}}
\caption{ Алгоритм перегруженного метода <<getFiles>> класса <<Hdd>> }
\label{bok_8:bok_8}
\end{figure}

После разработки архитектуры класса, он был внедрен в <<скелет>> проекта. Класс конструируется перед работой плагинов, но после определения операционной системы.

\begin{figure}[!ht]
\center{\includegraphics[width=1\linewidth]{bok_3}}
\caption{ Работа класса <<Hdd>> при запуске <<COEX>> }
\label{bok_3:bok_3}
\end{figure}

Далее необходимо было изменить плагины, таким образом, чтобы они обращались к сохраненной копии диска вместо самого диска. Таким образом были изменены два плагина - <<TaskMediaScanner>> и <<TaskChromeWin>>.

Теперь перед нами стояла задача сравнить нагрузку диска до и после внедрения класса <<Hdd>>. Поскольку нами использовался удаленный репозиторий и система контроля версий git, то это не составило проблемы по причине того, что разработка класса велась в отдельной <<ветке>>.

Было решено с помощью утилиты iotop замерить использование жесткого диска (в КБ/с) несколько раз до и после введения нового плагина и отфильтровать полученные результаты, чтобы учитывать исключительно нагрузку, создаваемую программой <<COEX>>. Для этого был разработан мультипоточный скрипт на языке Python, запускающий отдельно утилиту <<iotop>> и <<COEX>> и фильтрующий результаты, сохраняемые утилитой <<iotop>> (приложение ~\ref{apx:diskusage}):

\begin{figure}[!ht]
\center{\includegraphics[width=1\linewidth]{bok_4}}
\caption{ Результат работы скрипта }
\label{bok_4:bok_4}
\end{figure}

Далее результаты были обработаны, и на основании их был построен график, показывающий нагрузку на жесткий диск до и после внедрения класса <<Hdd>>. Было решено оставить по 10 итераций на каждое измерение, поскольку после этого количества разница между итерациями была минимальна и уже прослеживалась значимая разница между измерениями.

\begin{figure}[!ht]
\center{\includegraphics[width=1\linewidth]{bok_5}}
\caption{ Сравнение нагрузки на жесткий диск до и после изменения архитектуры }
\label{bok_5:bok_5}
\end{figure}

Из графика видно, что даже при изменении всего двух плагинов для использования новой архитектуры нагрузка на диск заметно снизилась. Так как на данный момент в проекте <<COEX>> имеется 17 рабочих плагинов, преобразование каждого из них должно сильно сказаться на нагрузке жесткого диска в лучшую сторону.

\clearpage


 % ---------Отчет Марины--------------%
\newpage
\subsection{ Многопоточное программирование } 
Одной из поставленных в данном семестре задач стало изучение возможностей многопоточного программирования с использованием программной библиотеки Qt. Программирование потоков осуществляется с помощью класса QThreads, а также механизма сигналов и слотов.

Все это необходимо для того, чтобы реализовать в системе <<COEX>> параллельное выполнение программных модулей (<<плагинов>>), осуществляющих поиск остаточных данных с образа системы, изображений, различных файлов и т.д. 

\subsubsection{Сигналы и слоты}

Сигналы и слоты используются для связи между объектами. Механизм сигналов и слотов --- это основная особенность Qt и, вероятно, основная часть Qt, которая больше всего отличается по функциональности от других библиотек.

Более старые инструментарии обеспечивают подобную связь с помощью функций обратного вызова. Обратный вызов --- это указатель на функцию. Если необходимо, чтобы функция обработки уведомила о некотором событии, ей передается указатель на другую функцию (отзыв). Функция обработки вызовет функцию обратного вызова, когда это будет уместно. Но данный подход имеет два фундаментальных недостатка: во-первых, он не типобезопасен. Мы некогда не сможем проверить, что функция обработки вызывает отзыв с правильными аргументами. Во-вторых, этот метод жестко связан с функцией обработки, так как она должна знать, какой отзыв вызывать.

В Qt используется техника, альтернативная функциям обратного вызова: механизм сигналов и слотов. Сигнал испускается, когда происходит определенное событие. Слот --- это функция, вызываемая в ответ на определенный сигнал.

Этот механизм типобезопасен: сигнатура сигнала должна соответствовать сигнатуре принимающего слота (фактически, слот может иметь более короткую сигнатуру, чем сигнал, который он получает, поскольку может игнорировать лишние аргументы). Сигналы и слоты связаны нежёстко: класс, испускающий сигналы, не знает и не интересуется, который из слотов получит сигнал. Механизм сигналов и слотов Qt гарантирует, что, если сигнал соединен со слотом, слот будет вызываться с параметрами сигнала в нужный момент. Сигналы и слоты могут иметь любое количество аргументов любых типов. Они полностью типобезопасны.

Все классы, наследуемые от QObject или одного из его подклассов (например, QWidget) могут содержать сигналы и слоты. Сигналы испускаются при изменении объектом своего состояния, если это изменение может быть интересно другим объектам. Все объекты делают это для связи с другими объектами. Их не заботит, получает ли кто-нибудь испускаемые ими сигналы. Это является истинной инкапсуляцией информации, и она гарантирует, что объекты могут использоваться как отдельные компоненты программного обеспечения.

Слоты могут получать сигнал, но они также являются обыкновенными функциями-членами. Также, как объект не знает, получает ли кто-нибудь сигналы, испускаемые им, слоты не знают, существуют ли сигналы, с ними связанные. Это гарантирует, что можно создать полностью независимые Qt-компоненты.

Можно присоединять к одному слоту столько сигналов, сколько необходимо, и один сигнал может быть соединен со столькими слотами, сколько требуется. Также возможно соединение сигнала непосредственно с другим сигналом (второй сигнал будет испускаться немедленно всякий раз, когда испускается первый).

Вместе сигналы и слоты представляют собой мощный механизм компонентного программирования. Графическое представление связи сигналов и слотов различных объектов можно увидеть на рисунке~\ref{signals-slots:signals-slots}.

\textbf{тут ссылка на http://doc.crossplatform.ru/qt/4.3.2/signalsandslots.html}

\begin{figure}[h!]
\center{\includegraphics[width=0.8\linewidth]{signals-slots}}
\caption{ Механизм сигналов и слотов для связи объектов в Qt }
\label{signals-slots:signals-slots}
\end{figure}

\subsubsection{Потоки QThreads}

В многопоточных приложениях, обслуживание интерфейса производится в отдельном потоке, а обработка данных -- в другом (одном или нескольких) потоке. В результате приложение сохраняет возможность откликаться на действия пользователя даже во время интенсивной обработки данных. Еще одно преимущество многопоточности -- на многопроцессорных системах различные потоки могут выполняться на различных процессорах одновременно, что несомненно увеличивает скорость исполнения.

Для реализации потоков Qt предоставляет класс QThread.

Поток — это независимая задача, которая выполняется внутри процесса и разделяет вместе с ним общее адресное пространство, код и глобальные данные.

Процесс, сам по себе, не является исполнительной частью программы, поэтому для исполнения программного кода он должен иметь хотя бы один поток (далее -- основной поток). Конечно, можно создавать и более одного потока. Вновь созданные потоки начинают выполняться сразу же, параллельно с главным потоком, при этом их количество может изменяться — одни создаются, другие завершаются. Завершение основного потока приводит к завершению процесса, независимо от того, существуют другие потоки или нет.Создание нескольких потоков в процессе получило название многопоточность.

\textbf{http://qt-doc.ru/processy-i-potoki-v-qt.html}

Для использования многопоточности нужно унаследовать класс от QThread. Чтобы запустить поток, нужно вызвать метод start().

Каждый поток может иметь собственный цикл обработки событий. Главный поток начинает цикл обработки событий, используя QCoreApplication::exec(); другие потоки могут начать свои циклы обработки событий, используя QThread::exec().

Цикл обработки событий потока делает возможным использование потоком некоторых неграфических классов Qt, которые требуют наличия цикла обработки событий (такие как QTimer, QTcpSocket и QProcess). Это также даёт возможность соединить сигналы из любых потоков со слотами в определённом потоке (рис.~\ref{thread-cycle:thread-cycle}).

\begin{figure}[h!]
\center{\includegraphics[width=0.8\linewidth]{thread-cycle}}
\caption{ Цикл обработки событий потоков в Qt }
\label{thread-cycle:thread-cycle}
\end{figure}

\textbf{http://doc.crossplatform.ru/qt/4.4.3/threads.html}

Для того, чтобы запускать программные модули на выполнение в нескольких потоках и должным образом завершать их выполнение, понадобилось написать контроллер --- объект, который создает потоки для уже существующих объектов (самих плагинов), перемещает эти объекты в созданные потоки. Далее он запускает их выполнение при помощи метода Controller::start\_threads(). После того, как каждый поток завершается, он посылает сигнал контроллеру о завершении finished() и переходит в режим ожидания. Когда все потоки завершаются, контроллер задействует слот stop\_threads(), предназначенный для того, чтобы послать сигнал об успешном завершении работы как всех потоков, так и работы самого контроллера, основной программе. При этом для связи сигналов и слотов используется функция connect(object\_1, SIGNAL(signal\_1), object\_2, SLOT(slot\_2)).  

Поскольку главный поток (основная программа) начинает цикл обработки событий, используя QCoreApplication::exec(), при получении сигнала finished(), сгенерированного контроллером, цикл обработки событий главного потока прерывается и главная программа успешно завершается. 

Блок-схема алгоритма работы основной программы представлена на рисунке~\ref{main-qthreads:main-qthreads}, блок-схема контроллера --- на рисунке~\ref{controller-qthreads:controller-qthreads} 

\begin{figure}[h!]
\center{\includegraphics[width=0.5\linewidth]{main-qthreads}}
\caption{ Блок-схема алгоритма работы основной программы qthreads }
\label{main-qthreads:main-qthreads}
\end{figure}

\clearpage

\begin{figure}[ht]
\center{\includegraphics[width=0.3\linewidth]{controller-qthreads}}
\caption{ Блок-схема алгоритма работы контроллера qthreads }
\label{controller-qthreads:controller-qthreads}
\end{figure}

\clearpage

Результат вывода программы представлен на рисунке~\ref{program-output:program-output}.

\begin{figure}[h!]
\center{\includegraphics[width=0.8\linewidth]{program-output}}
\caption{ Вывод программы qthreads }
\label{program-output:program-output}
\end{figure}

Кроме написания потокового контроллера был доработан плагин ThreadTaskICQ таким образом, чтобы учитывались особенности работы с механизмом сигналов и слотов. Только после этого плагин может быть запущен в потоке с использованием класса QThreads.  

\subsubsection{Итоги работы за семестр}

Таким образом, в течение семестра была написана рабочая программа-реализация многопоточного программирования с использованием программной кроссплатформенной библиотеки Qt. Изучены некоторые особенности работы механизма слотов и сигналов. Доработан модуль ThreadTaskICQ. В дальнейшем планируется имплементировать написанный контроллер потоков под архитектуру системы <<COEX>>, а также дописать должным образом плагины для реализации возможности выполнения программных модулей в потоках с использованием слотов и сигналов. 






\newpage
\section*{Заключение}
\addcontentsline{toc}{section}{Заключение}
В данном семестре нашей группой была выполнена часть работы по созданию автоматизированного программного комплекса для проведения компьютерной экспертизы. Основной целью в данном семестре стала подготовка проекта <<COEX>> к релизу, для чего были разработаны веб-сайт и репозиторий проекта, графический интерфейс пользователя, доработаны некоторые из программных модулей, собран <<бинарный>> пакет для установки и распространения системы <<COEX>>. 
 
 
 \newpage
 \renewcommand{\refname}{Список использованных источников}
 \bibliography{lit}

 \ESKDappendix{Обязательное}{\normalfont Компакт-диск}
 Компакт-диск содержит: 
 \begin{itemize}
 \item электронную версию пояснительной записки в форматах *.tex и *.pdf;
 \item актуальную версию программного комплекса для проведения компьютерной экспертизы;
 \item тестовые данные для работы с программным комплексом.
 \end{itemize}
 
\end{document}
