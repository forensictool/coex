Doxygen --- это кроссплатформенная система документирования исходных текстов, которая поддерживает разлличные языки программирования (в том числе и C++) \cite{doxygen}. 

Doxygen генерирует документацию на основе набора исходных текстов и также может быть настроен для извлечения структуры программы из недокументированных исходных кодов. Возможно составление графов зависимостей программных объектов, диаграмм классов и исходных кодов с гиперссылками.

Doxygen имеет встроенную поддержку генерации документации в формате HTML, \LaTeX\, man, RTF и XML. Также вывод может быть легко сконвертирован в CHM, PostScript, PDF.

Doxygen — консольная программа в духе классической Unix. Она работает подобно компилятору, анализируя исходные тексты и создавая документацию. Параметры создания документации читаются из конфигурационного файла, имеющего простой текстовый формат.

Автором программы является голландец Димитри ван Хееш (Dimitri van Heesch).

QT Creator --- среда разработки, использумая разработчиками coex --- поддерживает формат комментариев, используемого Doxygen при генерации документации. В ходе работы была сгенерирована документация при помощи данного инструмента, представляющая собой набор связанных html-страниц, описывающих различные классы и функции, используемые в coex. В дальнейшем планируется усовершенствовать полученную документацию, добавить описания для всех программных объектов.
