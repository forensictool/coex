Для вывода результата был выбран формат XML-документов, так как с данным форматом лего работать при помощи программ, а результат работы данного комплекса в дальнейшем планируется обрабатывать при помощи программ.

XML - eXtensible Markup Language или расширяемый язык разметки. Язык XML представляет собой простой и гибкий текстовый формат, подходящий в качестве основы для создания новых языков разметки, которые могут использоваться в публикации документов и обмене данными \cite{xml}. Задумка языка в том, что он позволяет дополнять данные метаданными, которые разделяют документ на объекты с атрибутами. Это позволяет упростить программную обработку документов, так как структурирует информацию.

Простейший XML-документ может выглядеть так:


\begin{verbatim}
<?xml version="1.0"?>
<list_of_items>
<item id="1"\><first/>Первый</item\>
<item id="2"\>Второй <subsub_item\>подпункт 1</subsub_item\></item\>
<item id="3"\>Третий</item\>
<item id="4"\><last/\>Последний</item\>
</list_of_items>
\end{verbatim}


Первая строка - это объявление начала XML-документа, дальше идут элементы документа <list\_of\_items> - тег описывающий начало элемента \\list\_of\_items, </list\_of\_items> - тег конца элемента. Между этими тегами заключается описание элемента, которое может содержать текстовую информацию или другие элементы (как в нашем примере). Внутри тега начала элемента так же могут указывать атрибуты элемента, как например атрибут id элемента item, атрибуту должно быть присвоено определенное значение.
