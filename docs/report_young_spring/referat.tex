\newpage
\ESKDthisStyle{empty}
\paragraph{\hfill РЕФЕРАТ \textbf{ПРАВИТЬ!!!} \hfill}
Курсовая работа содержит \ESKDtotal{page} страниц, \ESKDtotal{figure} рисунка, \ESKDtotal{table} таблицы, \ESKDtotal{bibitem} источников, \ESKDtotal{appendix} приложение.

%допилить ключевые слова
КОМПЬЮТЕРНАЯ ЭКСПЕРТИЗА, ФОРЕНЗИКА, ЛОГИ, QT, XML, GIT, LATEX, ICQ, MS OUTLOOK, WINDOWS, PST, MSG, RTF, HTML, БИБЛИОТЕКИ, РЕПОЗИТОРИЙ, МЕССЕНДЖЕР, ПОЧТОВЫЙ КЛИЕНТ, SQLLITE, РЕЕСТР, ИЗОБРАЖЕНИЯ, READPST, JPEG, PNG.

Цель работы --- создание программного комплекса, предназначенного для проведения компьютерной экспертизы.

Задачей, поставленной на данный семестр, стало написание программного комплекса, имеющего следующие возможности: 
\begin{enumerate}
\item сбор и анализ информации из реестра;
\item сбор и анализ информации из журналов истории браузеров;
\item сбор и анализ информации из мессенджеров;
\item сбор и анализ информации из почтовых приложений;
\item идентификации файлов изображений по внутреннему содержимому и их проверка;
\item сбора информации об установленном ПО по остаточным файлам.
\end{enumerate}

Результаты работы в данном семестре:

\begin{itemize}
\item реализован алгоритм извлечения строковых переменных из реестра Windows;
\item реализован алгоритм побитового считывания файла формата PST;
\item реализован импорт истории (посещений, поисковых запросов, загруженных файлов), закладок и 
другой информации (версия приложения, логин аккаунта google) из приложения Google Chrome;
\item реализован алгоритм парсинга контактного листа пользователя, сохраняемого приложением ICQ;
\item реализована проверка конца файла для форматов JPEG и PNG (для идентификации файлов изображений) и проверка заголовков 5 форматов изображений;
\end{itemize}

Пояснительная записка выполнена при помощи системы компьютерной вёрстки \LaTeX.
