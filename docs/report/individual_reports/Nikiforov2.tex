\subsection{Введение}

Анализ журналов операционных систем может помочь при решении многих задач. Примером таких задач может быть попытка восстановления системы после поломок, поиск причин неполадок системы, просмотр журналов с целью выявления активности приложений в определенные периоды и т.д. и т.п. Так же анализ журнальных файлов является неотъемлемой частью компьютерно-технических экспертиз 

Компьютерно-техническая экспертиза – это самостоятельный род судебных экспертизы, относящийся к классу инженерно-технических экспертиз, проводимая в целях: определения статуса объекта как компьютерного средства, выявление и изучение его роли в рассматриваемом деле, а так же получения доступа к информации на электронных носителях с последующим всесторонним её исследованием. [1] 

При исследовании компьютеров задача анализа журнальных файлов ставится на этапе «выявление и изучение его роли в рассматриваемом деле», так как именно в журнальных файлах операционной системы хранится информация о действиях производимых на данном компьютере. 

Для проведения компьютерных экспертиз существует множество специализированных программных средств, как платных так и свободно распространяемых. Примером такого средства является «osforensics». Данный инструмент позволяет проводить множество различных исследований таких как просмотр содержимого оперативной памяти, поиск подозрительных файлов на жестком диске, составления списка программ установленных на исследуемом компьютере. Но бесплатная версия не предоставляет никаких средств по работе с журналами операционной системы. Данный факт подтолкнул к созданию собственного программного средства для проведения компьютерных экспертиз.

В данной работе описывается работа по созданию модуля для данного программного средства, позволяющего в автоматическом режиме находить, читать и конвертировать журнальные файлы операционной системы Windows XP в XML-документы, понятные для человека и легко обрабатываемые при помощи электронно-вычислительных средств.

\subsection{Общие сведения о журнальных файлах}

В операционной системе Windows XP по умолчанию есть четыре журнала:

\begin{enumerate}
\item Журнал приложений
\item Журнал безопасности
\item Журнал установки
\item Журнал системы
\end{enumerate}

Каждый из этих журналов содержит определенный тип информации. Журналы приложений содержат информацию о запуске и остановке процессов приложений, изменении статуса каждого приложения, а так же предупреждения и ошибки связанные с приложениями. 

Журнал безопасности содержит информацию о входах в систему, и события связанные с безопасностью системы, например превышение количества попыток неправильно введенных подряд паролей. 

Журнал установки содержит информацию об установке и обновлении компонентов системы. 

Журнал системы хранит информацию о системных событиях. Например изменение схемы энергопотребления или различного рода предупреждения и ошибок. 

Каждый журнал хранится в соответствующем файле с расширением .evt. Эти файлы хранятся в папке \%windows\%/system32/config. Эти файлы имеют специальную структуру. 

\subsection{Структура журнальных файлов операционной системы Windows XP}

Все журнальные файлы операционной системы Windows XP имеют единую структуру. Файл представляет собой последовательность записей бинарных данных. Каждая запись – это структура имеющая семнадцать полей [2]. 

Первые 4 байта содержат длину события в байтах, после длинны идет 4 байтный системный код сообщения. Затем 4 байтовый номер записи, после него дата создания записи, время создания, идентификатор события, тип события и так далее. Ниже приведен список полей записи, предназначенной для считывания одного события из журнального файла. 

uint32 Length; \\
uint32 Reserved; \\
uint32 RecordNumber; \\
uint32 TimeGenerated; \\
uint32 TimeWritten; \\
uint32 EventID; \\
uint16 EventType; \\
uint16 NumStrings; \\
uint16 EventCategory; \\
uint16 ReservedFlags; \\
uint32 ClosingRecordNumber; \\
uint32 StringOffset; \\
uint32 UserSidLength; \\
uint32 UserSidOffset; \\
uint32 DataLength; \\
uint32 DataOffset; \\
byte[] Data. \\

Из сторонних источников [2] стало известно о пяти типах событий (поле EventType) (значение - тип события):

\begin{enumerate}
\item 0x0001 – Error event;
\item 0x0010 – Failure Audit event;
\item 0x0008 – Success Audit event;
\item 0x0004 – Information event;
\item 0x0002 – Warning event.
\end{enumerate}

У поля EventID удалось определить четыре значения: 

\begin{enumerate}
\item 0x00 – Success; \\
\item 0x01 – Informational;\\
\item 0x02 – Warning; \\
\item 0x03 – Error. \\
\end{enumerate}

Среди множества полей записи события были выделены поля содержащие информацию о типе события, времени возникновения события и создания записи, пользователя от имени которого была сделана запись, а так же поле Data - поле с бинарными данными в которых записана подробная информация о событии.

\subsection{Описание основных функций модуля}

Данный модуль является классом-наследником от некоторого абстрактного класса используемого как основу для всех модулей программы (шаблон проектирования Factory method). Модуль содержит в себе 8 методов и 3 атрибута: 

QString manual() - возвращает справку о входных параметрах данного таска \\
void setOption(QStringList list) - установка флагов для поданных на вход параметров \\
QString command() - возвращает команду для инициализации такска вручную \\
bool supportOS(const coex::typeOS &os) - возврощает флаг указывающий на возможность использования данного таска для конкретной операционной системы \\
QString name() - возвращает имя данного таска \\
QString description() - возвращает краткое описание такска \\
bool test() - предназначена для проверки работоспособности таска \\
bool execute(const coex::config &config) - запуск таска на выполнение \\
QString m_strName - хранит имя таска \\
QString m_strDescription - хранит описание таска \\
bool m_bDebug - флаг для параметра –debug 

\subsection{XML}

XML - eXtensible Markup Language или расширяемый язык разметки. Язык XML представляет собой простой и гибкий текстовый формат, подходящий в качестве основы для создания новых языков разметки, которые могут использоваться в публикации документов и обмене данными. [3] Задумка языка в том, что он позволяет дополнять данные метаданными, которые разделяют документ на объекты с атрибутами. Это позволяет упростить программную обработку документов, так как структурирует информацию. 

Простейший XML-документ может выглядеть так: 

<?xml version="1.0"?> \\
<list_of_items> \\
<item id="1"\><first/>Первый</item\> \\
<item id="2"\>Второй <sub_item\>подпункт 1</sub_item\></item\> \\
<item id="3"\>Третий</item\> \\
<item id="4"\><last/\>Последний</item\> \\
</list_of_items\> 

Первая строка это объявление начала XML-документа, дальше идут элементы документа <list_of_items> - тег описывающий начало элемента list_of_items, </list_of_items> - тег конца элемента. Между этими тегами заключается описание элемента, которое может содержать текстовую информацию или другие элементы (как в нашем примере). Внутри тега начала элемента так же могут указывать атрибуты элемента, как например атрибут id элемента item, атрибуту должно быть присвоено определенное значение. 

\subsection{Алгоритм работы модуля}

Модуль выполняет две задачи: поиск журнальных файлов на образе диска и конвертацию каждого файла в XML-документ. Первая задача выполняется при помощи библиотек QDir и QDirIterator из Qt Framework.

QDir — библиотека позволяющая работать с конкретной директорией. Создав объект данного типа с указанием директории мы получим доступ к этой директории в программе и сможем работать в ней (просматривать содержимое; удалять, создавать или копировать файлы; создавать поддиректории). Данный объект так же поддерживать разные наборы фильтров выходных данных которые могут отсеивать ненужную информацию. 

QDirIterator — библотека, предназначенная для работы с файловой системой начиная с определенной директории как точки входа. Создав объект данного типа с указанием директории мы получим все пути которые существуют в файловой системе и начинаются с указанной директории. Данный объект поддерживает фильтрацию которая помогает выделять только необходимую информацию, исключая то, что нас не интересует, например можно вывести список только файлов находящихся в данной директории или поддиректориях, или исключить вывод символьных ссылок. Объекты данного типа используются для поиска файлов или папок на образе исследуемого диска. 

Так же данные библиотеки позволяют создавать объект QFile, который позволяет работать с файлом, путь к которому передается как параметр при создании, данный объект позволяет получить базовую информацию о файле, такую как относительный или абсолютный путь до этого файла, размер файла, тип файла или его имя. Так же позволяет перемещать или копировать данный файл. Поиск работает по алгоритму представленному на рисунке (тут картинка evtSearch.eps) 

На выходе данного алгоритма мы получаем объект QStringList который содержит пути до всех найденных .evt файлов. Каждый экземпляр коллекции строк данного объекта передается в конструктор объекта winEventLog, который и конвертирует указанный файл в XML-документ. Алгоритм работы конвертора представлен на рисунке (тут картинка evtToXML.eps) 

%Список используемых источников
%1. Николай Николаевич Федотов - ФОРЕНЗИКА компьютерная криминалистика 2-е издание;
%2. http://www.whitehats.ca/main/members/Malik/malik_eventlogs/malik_eventlogs.html;
%3. http://msdn.microsoft.com/ru-ru/library/ms256177(v=vs.110).aspx.