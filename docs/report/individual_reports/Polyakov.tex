\subsection{Введение}

Задачей поставленной на данный семестр стало написание автоматизированного экспертизного? комплекса, имеющего следующие возможности: \\

\begin{enumerate}
\item сбор и анализ событий системных журналов операционной системы;
\item сбор и анализ информации из журналов истории браузеров;
\item сбор и анализ истории переписки мессенджеров;
\item сбор и анализ событий журнальных файлов приложений;
\item обнаружение сетевых параметров системы;
\item поиск файлов по имени;
\end{enumerate}


Моя задача на данный семестр . \\

\begin{enumerate}
\item сбор и анализ истории переписки мессенджеров;
\end{enumerate}

Для упрощения разобьем задачу, на подзадачи
\begin{enumerate}
\item определить места хранения переписки пользователя
\item определить формат хранения переписки
\item разработать парсер для каждого из возможных форматов
\item выделить важную информацию их каждой записи
\item автоматизировать процесс поиска журнальных файлов
\item производить сохранение полученных информации формат XML
\end{enumerate}

\subsection{Определить формат хранения переписки.}

Приложение «skype» хранит переписку локально на машинах пользователей или же возможна синхронизация с машин других пользователей [1]. Формат хранения SQLite. \\

Приложение «pidgin» хранит лог файлы локально на машине пользователя в формате HTML и TXT. По умолчанию лог файли хранятся в .нтмл файле. Настройки программы и подключенных аккаунтов в XML. \\

\subsection{Определить места хранения переписки пользователя.}

Поиском файлов занимается класс QDirIterator. После вызова проиходит поочередный обход по каждому файлу в директории и под директори, проверка полученного файла по суффиксу методом FileInfo.suffix(), если это HTML TXT происходит добавление в список.
\\

\subsection{Парсинг найденых файлов.}

Из полученного ранее спаска, найденые файлы поочередно открываются на чтение (построково). Методами класса QstringList происходит резанье строк и добавление в список. Для парсинга полученного списка используется регулярные вырожения использующие класс QRegExp.\\ 


На данный момент полностью реализован комплекс для работы с журнальными файлами windows 7\\

\subsection{Сохранение полученного в XML.}

Соранение полученных данных происходит в ранее выбраный формат XML. Для этого используется класс QXmlStreamReader и QxmlStreamWriter.
Класс QXmlStreamWriter представляет XML писателя с простым потоковым API.

QXmlStreamWriter работает в связке с QXmlStreamReader для записи XML. Как и связанный класс, он работает с QIODevice, определённым с помощью setDevice ().


Класс QXmlStreamReader представляет собой быстрый синтаксически корректный XML анализатор с простым потоковым API.
QXmlStreamReader является быстрым и более удобным для замены в Qt анализатора SAX (смотрите QXmlSimpleReader), а в некоторых случаях он даже более предпочтителен, чем использование DOM дерева (смотрите QDomDocument). QXmlStreamReader считывает данные с QIODevice (смотрите setDevice()) или с необработанного QByteArray (смотрите addData()). Вместе с QXmlStreamWriter Qt обеспечивает связанный класс для записи XML.\\

Работа с xml-файлами \\

XML - eXtensible Markup Language или расширяемый язык разметки. XML разрабатывался как язык с простым формальным синтаксисом, удобный для создания и обработки документов программами и одновременно удобный для чтения и создания документов человеком. Задумка языка в том, что он позволяет дополнять данные метаданными, которые разделяют документ на объекты с атрибутами. Это позволяет упростить программную обработку документов, так как структурирует информацию. \\

Список использованной литературы:
1 -  http://community.skype.com/t5/Security-Privacy-Trust-and/Is-chat-history-stored-on-Skype-servers/td-p/472379 вопрос где скайп логи хранит
