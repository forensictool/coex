Qt - это кроссплатформенная библиотека C++ классов для создания графических пользовательских интерфейсов (GUI) от фирмы Digia. Эта библиотека полностью объектно-ориентированная, что обеспечивает легкое расширение возможностей и создание новых компонентов. Ко всему прочему, она поддерживает огромнейшее количество платформ.

Qt позволяет запускать написанное с его помощью ПО в большинстве современных операционных систем путём простой компиляции программы для каждой ОС без изменения исходного кода. Включает в себя все основные классы, которые могут потребоваться при разработке прикладного программного обеспечения, начиная от элементов графического интерфейса и заканчивая классами для работы с сетью, базами данных и XML. Qt является полностью объектно-ориентированным, легко расширяемым и поддерживающим технику компонентного программирования.

Список использованных классов фраемворка QT
\begin{itemize}
\item iostream
\item QChar
\item QCryptographicHash
\item QDir
\item QDirIterator
\item QFile
\item QFileInfo
\item QIODevice
\item QList
\item QRegExp
\item QString
\item QTextStream
\item QtSql/QSqlDatabase
\item QVector
\item QXmlStreamReader
\item QXmlStreamWriter
\item Conversations
\end{itemize}

Класс QXmlStreamWriter представляет собой XML писателя с простым потоковым.

Класс QXmlStreamReader представляет собой быстрый синтаксически корректный XML анализатор с простым потоковым API. 

QVector представляет собой класс для создания динамических массивов.

Модуль QtSql/QSqlDatabase помогает обеспечить однородную интеграцию БД в ваши Qt приложения.

Класс QTextStream предоставляет удобный интерфейс для чтения и записи текста.

QTextStream может взаимодействовать с QIODevice, QByteArray или QString. Используя потоковые операторы QTextStream, вы можете легко читать и записывать слова, строки и числа. При формировании текста QTextStream поддерживает параметры форматирования для заполнения и выравнивания полей и форматирования чисел. \cite{qtdoc}

Класс QString предоставляет строку символов Unicode.

QString хранит строку 16-битных QChar, где каждому QChar соответствует один символ Unicode 4.0. (Символы Unicode со значениями кодов больше 65535 хранятся с использованием суррогатных пар, т.е. двух последовательных QChar.)

Unicode - это международный стандарт, который поддерживает большинство использующихся сегодня систем письменности. Это расширение US-ASCII (ANSI X3.4-1986) и Latin-1 (ISO 8859-1), где все символы US-ASCII/Latin-1 доступны на позициях с тем же кодом.

Внутри QString использует неявное совместное использование данных (копирование-при-записи), чтобы уменьшить использование памяти и избежать ненужного копирования данных. Это также позволяет снизить накладные расходы, свойственные хранению 16-битных символов вместо 8-битных.

В дополнение к QString Qt также предоставляет класс QByteArray для хранения сырых байт и традиционных нультерминальных строк. В большинстве случаев QString - необходимый для использования класс. Он используется во всем API Qt, а поддержка Unicode гарантирует, что ваши приложения можно будет легко перевести на другой язык, если в какой-то момент вы захотите увеличить их рынок распространения. Два основных случая, когда уместно использование QByteArray: когда вам необходимо хранить сырые двоичные данные и когда критично использование памяти (например, в Qt для встраиваемых Linux-систем).\cite{qtcross}

Класс QRegExp предоставляет сопоставление с образцом при помощи регулярных выражений.

Регулярное выражение, или ''regexp'', представляет собой образец для поиска соответствующей подстроки в тексте. Это полезно во многих ситуациях, например:

Проверка правильности -- регулярное выражение может проверить, соответствует ли подстрока каким-либо критериям, например, целое ли она число или не содержит ли пробелов.
Поиск -- регулярное выражение предоставляет более мощные шаблоны, чем простое соответствие строки, например, соответствие одному из слов mail, letter или correspondence, но не словам email, mailman, mailer, letterbox и т.д.
Поиск и замена -- регулярное выражение может заменить все вхождения подстроки другой подстрокой, например, заменить все вхождения \& на \&amp;, исключая случаи, когда за \& уже следует amp;.
Разделение строки -- регулярное выражение может быть использовано для определения того, где строка должна быть разделена на части, например, разделяя строку по символам табуляции.

QFileInfo  - Во время поиска возвращает полную информацию о файле.

Класс QDir обеспечивает доступ к структуре каталогов и их содержимого.

QIODevice представляет собой базовый класс всех устройств ввода/вывода в Qt.

Класс QCryptographicHash предоставляет способ генерации криптографических хэшей.
QCryptographicHash могут быть использованы для генерации криптографических хэшей двоичных или текстовых данных.В настоящее время MD4, MD5, и SHA-1 поддерживаются.\cite{qtcross}

QChar обеспечивает поддержку 16-битных символов Unicode.

\subsubsection{Aвтоматизация поиска журнальных файлов}

Для сканирования образа на наличие интересующих лог файлов использовался класс QDirIterator. После вызова происходит поочередный обход по каждому файлу в директории и поддиректории. Проверка полученного полного пути к файлу осуществляется регулярным выражением, если условие выполняется, происходит добавление в список обрабатываемых файлов.

\subsubsection{Реализация сохранения результатов работы программного комплекса в XML}

Cохранение полученных данных происходит в ранее выбранный формат XML(Extensible Markup Language). Для этого используется класс QXmlStreamReader и QxmlStreamWriter.
Класс QXmlStreamWriter представляет XML писателя с простым потоковым API.

QXmlStreamWriter работает в связке с QXmlStreamReader для записи XML. Как и связанный класс, он работает с QIODevice, определённым с помощью setDevice().

Сохранение данных реализованно в классе WriteMessage. В методе WriteMessages, структура которого представлена на UML диаграмме в разделе Архитектура.
