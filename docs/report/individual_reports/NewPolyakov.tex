\newpage

\chapter*{Введение}

Задачей поставленной на данный семестр стало написание автоматизированного экспертизного? комплекса, имеющего следующие возможности: \\

\begin{enumerate}
\item сбор и анализ событий системных журналов операционной системы;
\item сбор и анализ информации из журналов истории браузеров;
\item сбор и анализ истории переписки мессенджеров;
\item сбор и анализ событий журнальных файлов приложений;
\item обнаружение сетевых параметров системы;
\item поиск файлов по имени;
\end{enumerate}


Моя задача на данный семестр . \\

\begin{enumerate}
%\item сбор и анализ истории переписки мессенджеров;
\item преобразование собранной ирформации в читаемый вид;
\item добавление новых ффункций к уже имеющимуся.
\end{enumerate}

Для упрощения разобьем задачу, на подзадачи
\begin{enumerate}
\item Визуализация данных
\item Выбор и описание утилит для преобразования
\item XSTL
\item Показать результат

\item функция чтения контактной книги pidgin
\item функция сбора используемых аккаунтов pidgin
\item Описать как работает. SAP и DOM модель разбора xml
\item функция сбора дополнительных данных skype
\end{enumerate}

\chapter*{Визуализация данных}

Визуализация собранной информации очень важный момент создания подобного рода комплексов, т.к. это является главной задачей нашего продукта. 
%Здесь думаю добавить список того, во что вообще можно сконвертить xml.
Список возможных вариантов конвертации XML
\begin{enumerate}
\item CSV
\item HTML
\item TXT
\item PDF
\item ...
\end{enumerate}

Самым удобным оказался HTML.

\chapter*{Выбор и описание утилит для преобразования}

Единственное что я нашел, XSLT

\chapter*{XSTL}

\\XSLT (eXtensible Stylesheet Language Transformations) — язык преобразования XML-документов. Спецификация XSLT входит в состав XSL и является рекомендацией W3C.

При применении таблицы стилей XSLT, состоящей из набора шаблонов, к XML-документу (исходное дерево) образуется конечное дерево, которое может быть сериализовано в виде XML-документа, XHTML-документа (только для XSLT 2.0), HTML-документа или простого текстового файла. Правила выбора (и, отчасти, преобразования) данных из исходного дерева пишутся на языке запросов XPath.

XSLT имеет множество различных применений, в основном в области веб-программирования и генерации отчётов. Одной из задач, решаемых языком XSLT, является отделение данных от их представления, как часть общей парадигмы MVC (англ. Model-view-controller). Другой стандартной задачей является преобразование XML-документов из одной XML-схемы в другую.

\chapter*{Показать результат}

\\Здесь пара картинок таго чего получилось

Список использованной литературы:
1 -  http://community.skype.com/t5/Security-Privacy-Trust-and/Is-chat-history-stored-on-Skype-servers/td-p/472379 вопрос где скайп логи хранит