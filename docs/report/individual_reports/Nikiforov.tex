\subsection{Введение}

Продолжая разработку данного программного продукта стало ясно что, несмотря на выбранный шаблон проектирования, добавление нового функционала осложняется тем, что при добавлении новой функции приходится пересобирать весь проект. Это неправльиный подход, так как не позволяет обновлять приложение. Для исправления этой ситуации необходимо доработать архитектуру приложения таким образом, что бы добавление новых функций не приводило к пересборке всего проекта. Это так же позволит проще обновлять проект в дальнейшем. 

После доработки архитектуры, необходимо будет переделать проект, переделать реализацию всего фукционала, а так же свести все ветки разработки в одну, так как на данный момент не все изменения применены к проекту. \\
Сформулируем задачи на семестр:

\subsubsection{Задачи на семестр}

\begin{enumerate}
\item переработать архитектуру проекта для более легкого добавления новых функций
\item реализовать данную архитектуру
\item переделать готовые task'и для работы с текущей архитектурой
\item смерджить ветки разработки git
\end{enumerate}

\subsubsection{Новая архитектура}

С увеличением количества модулей возникла проблема с обновлением кода, так как используемый подход предполагал монолитную программу и все модули хоть и были стандартизированы при помощи шаблона разработки factory, всё равно являлись частью основной программы и для добавления модулей или их обновления необходимо постоянно перекомпилировать весь проект. Данная ситуация крайне неудобна при разработки и тестировании, поэтому необходимо было разделить проект на независимые модули.

В языке программирования C++ есть возможность создания библиотек. Библиотеки - это бинарные файлы в которых хранится коллекции классов и функций. Эти классы и функции можно использовать в любой программе в которой данная библиотека подулючается. Подключать библиотеки в программу можно статически(на этапе компиляции программы) или динамически(непосредственно при выполнении программы). У нашего проекта есть \"скелет\" и модули, каждый из которых работает независимо друг от друга и обменивается данными только со скелетом. 

Так как каждая библиотека является самостоятельным файлом с исполняемым кодом, то для его создания и изменения необходим отдельный проект разработки, а значит если мы захотим изменить что-либо внутри библиотеки, то практически наверняка нам не нужно будет перекомпилировать весь проект. Это значительно упрощает операцию обновления программного комплекса. Было принято решение вынести реализацию всех функций скелета в отдельную статическую библиотеку. А все модули оформить в виде динамических библиотек. 

Данное решение повлекло за собой ряд проблем:
\begin{enumerate}
\item Необходимо разработать структуру динамических библотек для модулей программного комплекса;
\item Необходимо изменить механизм работы с модулями;
\item Необходимо изменить механизм разбора и передачи параметров в основной части программного комплекса;
\item Разработать механизм передачи информации в механизм разбора и передачи параметров.
\end{enumerate}

\subsubsection{библиотеки в С++ и их использование}

Библиотеки предоставляют программисту возможность использовать при разработке программы готовые фрагменты кода. В библиотеки могут быть включены подпрограммы, структуры данных, классы, макросы. В QT для создания библиотеки необходимо укахать в pro-файле проекта добавить строчку \\
TEMPLATE = lib\\
Данная строчка указывает компилятору на то, что мы компилируем библиотеку. После чего в файле исходного кода описывается всё необходимое и после компиляции мы получим файл библиотеки. Для создания динамической библиотеки необходимо добавить в pro-файл ещё строчку: \\
CONFIG += dll\\
А так же в заголовочном файле библиотеки поместить описание прототипов функций которые мы хотим подключать динамически в конструкцию\\
extern \"C\"{ \"описание прототипов\" }\\
Это необходимо для того, что бы при сборке библиотеки функции не поменяли имена. Если не добавить данную конструкцию, то компилятор при сборке изменит имена функций и обратиться к ним динамически не получится.

Для использования разработанной статической библиотеки необходимо просто указать в файле исходного кода строчку \\
#include \"abolute/part/to/library/file\" \\
Либо поместить созданную библиотеку в папку с системными библиотеками (или добавить в настройки операционной системы путь до созданной библиотеки), тогда можно подключать библиотеку так же как и стандартные библиотеки (имя библиотеки в треугольных скобках)

Динамические библиотеки используются по-другому. Загрузка функции из динамической библиотеки происходит в три этапа: \\
\begin{enumerate}
\item Загрузка библиотеки;
\item Описание шаблона для вызова функции:
\item Вызов функции.
\end{enumerate}

В качестве примера на рисунке(?????) представлен пример вызова функции foо, которая принимает на вход один аргумент типа int и возвращает его удвоенное значение. функция находится в библиотеке exampleLib. Загрузка библиотеки производится при помощи средств, предоставляемых Qt framework: \\
%вот тут короче рисуночек dinLib из папки images

