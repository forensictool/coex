С развитием разрабатываемого программного комплекса возникла необходимость организовать свой репозиторий. Свой репозиторий нужен для того, чтобы выложить программное обеспечение в открытый доступ в удобном для установки и обновления виде.

\subsubsection{Порядок конфигурирования репозитория}

Для начала следует выбрать дирректорию, в которой мы хотим выкладывать наши deb пакеты. Пускай это будет /var/www/repositories/debian.

\ttfamily prefix=/var/www/repositories/debian\\ mkdir -p \$prefix/main \normalfont

Копируем все необходимые deb пакеты в /var/www/repositories/debian/main, там можно устроить любую структуру каталогов. Сделаем подкаталоги для каждого пакета и положим в них файлы .deb, .dsc, .changes и .tar.gz. Далее переходим в корень нашего репозитория.

\ttfamily cd \$prefix \normalfont

Создаем индексные файлы. Для бинарных пакетов сделать это можно с помощью команды dpkg-scanpackages.

\ttfamily
dir=\$prefix/dists/unstable/main/binary-i386\\
mkdir -p \$dir\\
dpkg-scanpackages --arch i386 main /dev/null > \$dir/Packages\\
gzip -9c <\$dir/Packages >\$dir/Packages.gz\\
bzip2 -9c <\$dir/Packages >\$dir/Packages.bz2
\normalfont

Проделаем это для каждой архитектуры, для которой есть пакеты в main. Затем необходимо создать индексные файлы для исходных текстов

\ttfamily
dir=\$prefix/dists/unstable/main/source\\
mkdir -p \$dir\\
dpkg-scansources main /dev/null > \$dir/Sources\\
gzip -9c <\$dir/Sources >\$dir/Sources.gz\\
bzip2 -9c <\$dir/Sources >\$dir/Sources.bz2
\normalfont

И наконец, необходимо создать файл \$prefix/dists/unstable/main/Release, который имеет следующую структуру:

\ttfamily
Archive: unstable
Suite: unstable
Component: main
Origin: название моей организации
Label: Debian репозиторий моей организации
Architecture: архитектуры
\normalfont

Создадим такой файл и заполним необходимые поля, затем дадим команды

\ttfamily
dir=\$prefix/dists/unstable/main
apt-ftparchive release \$dir >>\$dir/Release
\normalfont

Последняя команда вычисляет MD5Sum, SHA1 и SHA256 для файлов Packages* и Sources*.

Теперь осталось только подписать репозиторий. Сделаем это следующей командой:

\ttfamily
gpg -abs -o $dir/Release.gpg $dir/Release
\normalfont

Для того, чтобы aptitude, apt и др. смогли найти новый репозиторий, добавим необходимые записи в /etc/apt/sources.list.

\ttfamily
deb http://example.com/repository/debian unstable main
deb-src http://example.com/repository/debian unstable main
\normalfont
