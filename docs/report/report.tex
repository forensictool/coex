\documentclass[russian,utf8,14pt,simple]{eskdtext}

% - Полуторный интервал
\renewcommand{\baselinestretch}{1.50}
% - Отступ красной строки
\setlength{\parindent}{1.25cm}	
% - Шрифт Times New Roman
\renewcommand{\rmdefault}{ftm}

% - Наименование документа
\ESKDtitle{ }
% - Обозначение документа
\ESKDsignature{ФВС КР. Х.ХХХХХХХ 001 ПЗ}
% - Наименование предприятия
\ESKDcolumnIX{ТУСУР, ФВС, КИБЭВС-1208}
% - Проверил
\ESKDchecker{Давыдова Е.М.}	
% - Литера 
\ESKDletter{У}{}{}
% - Разработал
\ESKDauthor{КИБЭВС-1208}			

% - Убирает точку в списке литературы
\makeatletter
\def\@biblabel#1{#1 }
% - ГОСТ списка литературы
\bibliographystyle{gost780s}

\ESKDsectSkip{subsection}{1em}{1em}
\ESKDsectSkip{section}{1em}{1em}

% - Изменение заголовков
\usepackage{titlesec}
\titleformat{\section}{\normalsize}{\thesection}{1.0em}{}
\titleformat{\subsection}{\normalsize}{\thesubsection}{1.0em}{}
\titleformat{\paragraph}{\normalsize}{\theparagraph}{1.0em}{}

% - Для больших таблиц
\usepackage{longtable}

% - Используем графику в документе
\usepackage[dvips]{graphicx}
\graphicspath{{images\}}

% - Для вставки гиперссылок
\usepackage[colorlinks]{hyperref}

% - Счётчики
\usepackage{eskdtotal}

\begin{document}

\newpage
\ESKDthisStyle{empty}

\begin{center}
Министерство образования и науки Российской Федерации\\
ФЕДЕРАЛЬНОЕ ГОСУДАРСТВЕННОЕ БЮДЖЕТНОЕ ОБРАЗОВАТЕЛЬНОЕ\\
УЧРЕЖДЕНИЕ ВЫСШЕГО ПРОФЕССИОНАЛЬНОГО ОБРАЗОВАНИЯ\\
ТОМСКИЙ ГОСУДАРСТВЕННЫЙ\\
УНИВЕРСИТЕТ СИСТЕМ УПРАВЛЕНИЯ И РАДИОЭЛЕКТРОНИКИ\\
(ФГБОУ ВПО ТУСУР)\\
Кафедра комплексной информационной безопасности электронно-вычислительных систем (КИБЭВС)\\
\end{center}

\begin{tabbing}
XXXXXXXXXXXXXXXXXXXXXXXXXXX \=
XXXXXXXXXXXXXXXX\kill
\> УТВЕРЖДАЮ\\
\> заведующий каф.КИБЭВС\\
\> \underline{\ \ \ \ \ \ \ \ \ \ \ \ \ \ \ \ \ \ \ \ } А.А. Шелупанов\\
\> "\underline{\ \ \ \ \ }"\underline{\ \ \ \ \ \ \ \ \ \ \ \ \ \ \ \ \ \ \ \ } 2013г.\\
\end{tabbing}

\begin{center}
КОМПЬЮТЕРНАЯ ЭКСПЕРТИЗА\\
Отчет по групповому проектному обучению\\
Группа КИБЭВС-1208\\
\end{center}

\begin{tabbing}
XXXXXXXXXXXXXXXXXXXXXXXXXXX \=
XXXXXXXXXXXXXXXX\kill
\> Ответственный исполнитель\\
\> Студент гр. 520-1\\
\> \underline{\ \ \ \ \ \ \ \ \ \ \ \ \ \ \ \ \ \ \ \ } Никифоров Д. С.\\
\> "\underline{\ \ \ \ \ }"\underline{\ \ \ \ \ \ \ \ \ \ \ \ \ \ \ \ \ \ \ \ } 2013г.\\
\ \\
\> Научный руководитель\\
\> Аспирант каф.КИБЭВС\\
\> \underline{\ \ \ \ \ \ \ \ \ \ \ \ \ \ \ \ \ \ \ \ } Гуляев А. И.\\
\> "\underline{\ \ \ \ \ }"\underline{\ \ \ \ \ \ \ \ \ \ \ \ \ \ \ \ \ \ \ \ } 2013г.\\
\end{tabbing}
\vfill
\begin{center}
2013
\end{center}

\newpage
\ESKDthisStyle{empty}
\paragraph{\hfill РЕФЕРАТ \hfill}
Курсовая работа содержит \ESKDtotal{page} страниц, \ESKDtotal{table} таблиц, \ESKDtotal{bibitem} источников.

Пояснительная записка выполнена в текстовом редакторе nano 2.2.6.

\newpage
\ESKDthisStyle{empty}
\paragraph{\hfill Список исполнителей \hfill}
\begin{itemize}
\item Моргуненко А.В. --
\item Никифоров Д.С. -- 
\item Поляков И.Ю. --
\item Пономарёв А.К. --
\end{itemize}

\newpage
\ESKDstyle{formII}
\renewcommand\contentsname{\hfill Содержание \hfill}
\tableofcontents

\newpage
\ESKDstyle{formIIab}
\section{Введение}

\newpage
\section{}
\subsection{}

\section{Заключение}
В результате проделанной работы были достигнуты следующие результаты:
\begin{enumerate}
\item[1)]
\end{enumerate}

В дальнейшей работе планируется:
\begin{itemize}
\item
\end{itemize}

\newpage
% - Переименование списка литературы
\renewcommand{\refname}{Список использованных источников}
\bibliography{books}

\ESKDappendix{Обязательное}{\normalfont Компакт-диск}
Компакт-диск содержит: 
\begin{itemize}
\item электронную версию пояснительной записки в форматах *.tex и *.pdf;
\item индивидуальные ежемесячные отчеты студентов;
\item групповые ежемесячные отчёты.
\end{itemize}

\end{document}
