Под сертификацией принято понимать независимое подтверждение соответствия тех или иных характеристик оборудования или информационной системы некоторым требованиям. В нашем случае речь идет о программном комплексе для проведения компьютерной экспертизы, во время экспертизы программный комплекс хранит и обрабатывает информацию, которая может относиться к персональным данным, коммерческой или государственной тайне и др. а также на основании результатов работы программы специалист проводящий экспертизу пишет заключение, которое может быть использовано как доказательство в суде. Поэтому к данной системе могут быть применены требования, касающаяся безопасности информации.

Для обеспечения высокого показателя качества необходима сертификация технологии и системы обеспечения качества их проектирования, разработки и сопровождения.

Проведение сертификации системы обычно планируется для достижения одной или нескольких целей:

\begin{itemize} 
\item определения соответствия или несоответствия элементов системы установленным требованиям;
\item определения эффективности внедренной системы  с точки зрения соответствия поставленным целям для обеспечения качества продукции; 
\item обеспечения возможности улучшить свою систему;
\item определения соответствия системы регламентирующим требованиям.
\end{itemize}

Первоначальные затраты на их проведение могут нести инициаторы испытаний, заказчик, а также ее разработчики и поставщик.

\subsection{Схема проведения сертификации}

Особенность любых сертификационных испытаний — это независимость испытательной лаборатории сертифицирующей организации, осуществляющей независимый контроль результатов испытаний. 

Схема проведения сертификации:

\begin{enumerate}
\item Заявитель (в нашем случае разработчик) подает в федеральный орган (ФСБ, ФСТЭК или Минобороны) заявку на проведение сертификационных испытаний.
\item Федеральный орган определяет аккредитованную испытательную лабораторию и орган по сертификации.
\item Испытательная лаборатория совместно с заявителем проводит сертификационные испытания. Если в процессе испытаний выявляются те или иные несоответствия заявленным требованиям, то они могут быть устранены заявителем в рабочем порядке, либо может быть принято решение об изменении требований к продукту, например о снижении класса защищенности. 
\item  Материалы испытаний передаются в орган по сертификации, который проводит их независимую экспертизу. В экспертизе участвуют не менее двух независимых экспертов.
\item Федеральный орган по сертификации на основании заключения органа по сертификации оформляет сертификат соответствия. При каких-либо несоответствиях, федеральный орган может провести дополнительную экспертизу с привлечением экспертов из различных аккредитованных лабораторий и органов.
\end{enumerate}

В случае возникновения инцидентов на объектах информатизации, связанных с утечкой информации, регулирующие органы могут проинспектировать лабораторию, которая проводила испытания, и приостановить лицензию и аттестаты аккредитации.

Деятельность российских систем сертификации в РФ регламентируется Федеральным законом № 184 «О техническом регулировании». Сертификация средств защиты информации может быть добровольной или обязательной — проводимой главным образом в рамках Минобороны, ФСБ и ФСТЭК. К компетенции ФСБ относятся средства защиты информации, применяемые в органах государственной власти. Система сертификации средств защиты информации Минобороны, в свою очередь, ориентирована на программные изделия, применяемые на объектах военного назначения.\cite{mostest}

\subsection{Добровольная сертификация}

Добровольная сертификация применяется с целью повышения конкурентоспособности продукции, расширения сферы ее использования и получения дополнительных экономических преимуществ. Результаты сертификации должны оправдывать затраты на ее проведение вследствие получения пользователями продукции с показателями более высокого качества. Таким сертификационным испытаниям подвергаются компоненты операционных систем и пакеты прикладных программ широкого применения. Добровольные системы сертификации средств защиты информации на сегодняшний день пока еще не получили широкого распространения. Такого рода системой является «АйТи-Сертифика», но более известным средством сертификации является ЦЕРБЕР. К сожалению, несмотря на то что в добровольных системах можно получить сертификат на соответствие любому нормативному документу по защите конфиденциальной информации, при аттестации объектов информатизации такие сертификаты ФСТЭК России не признаются.

\subsection{Обязательная сертификация}

Обязательная сертификация необходима для программных продуктов и их производства, выполняющих особо ответственные функции, в которых недостаточное качество, ошибки или отказы могут нанести большой ущерб или опасны для жизни и здоровья людей. Примером такой сертификации служат программные продукты в авиации, в атомной энергетике, в военных системах и т.д.

Существуют два основных подхода к сертификации (таблица \ref{tab:sertif}):

\begin{table}[h!]
\caption{Основные подходы к сертификации}
\label{tab:sertif}
\begin{tabularx}{\linewidth}{|l|X|}
\hline
Подход & Документы\\ 
\hline
Функциональное тестирование & в соответствии с положениями стандарта ГОСТ Р 15408\\ 
\hline
Структурное тестирование & по ГОСТ Р 51275-99 \\ 
\hline
\end{tabularx}
\end{table}

Функциональное тестирование средств защиты информации, позволяющее убедиться в том, что продукт действительно реализует заявленные функции. Это тестирование проводится в соответствии с определенными нормативными документами Гостехкомиссии России. Если же не существует документа, которому сертифицируемый продукт соответствовал бы в полной мере, то функциональные требования могут быть сформулированы в явном виде – в технических условиях, или в виде задания по безопасности.

Структурное тестирование программного кода на отсутствие недекларированных возможностей. Недекларированными возможностями,например, являются программные закладки, которые при возникновении определенных условий позволяют осуществлять несанкционированные воздействия на информацию. Выявление недекларированных возможностей - проведение серии тестов исходных текстов программ.

В большинстве случаев средство защиты информации должно быть сертифицировано как в части основного функционала, так и на предмет отсутствия недекларированных возможностей. Делается исключение для систем обработки персональных данных второго и третьего класса с целью снижения затрат на защиту информации для небольших частных организаций. Если программное средство не имеет каких-либо механизмов защиты информации, оно может быть сертифицировано только на предмет отсутствия недекларированных возможностей. \cite{eshelon}

\subsection{Список организаций, проводящих сертификацию}

Так как в некотором будущем в нашей архитектуре должен появиться модуль системы защиты данных от НСД рассмотрим несколько организаций, которые могут провести сертификацию нашего программного комплекса (таблица \ref{tab:organizsert}):

\begin{center}
\begin{longtable}[h]{|*3{p{5cm}|}}
\caption{Организации проводящие сертификацию} \label{tab:organizsert}\\
\hline
Организация & Описание & Сертификация продуктов\\ 
\hline
\endfirsthead
Радиофизические тестовые технологии & Данный орган по сертификации програмных средств аккредитован Ростехрегулированием(Госстандартом) в системе сертификации ГОСТ Р. & 1 операционные системы и средства их расширения, 2 системы программирования и утилиты, 3 сетевое программное обеспечение,\\
 & &4 системы управления базами данных (СУБД),\\
 & &5 электронные таблицы,\\
 & &6 базы данных и информационно-справочные системы,\\
 & &7 электронные архивы.\\
 & &8 и другие.\\
\hline
АНО МИЦ & Центр зарегистрировал в Росстандарте систему добровольной сертификации программного обеспечения и аппаратно-программных комплексов & 1  ПО, используемое для моделирования технологических процессов, математического и иного моделирования; 2 ПО передачи, хранения, актуализации, защиты, обеспечения доступа и использования измерительной, вычислительной и иной информации;\\
 & &3 ПО баз данных;\\
 & &4 и другие.\\
\hline
Агенство РСТ & уполномоченный агент Регионального органа по сертификации и тестированию РОСТЕСТ-МОСКВА, одного из крупнейших сертификационных и испытательных центров в Европе. Сертификаты РОСТЕСТ-МОСКВА пользуются особым доверием потребителей и контролирующих организаций. & 1 электронные базы данных и информационные системы; 2 системы управления базами данных; 3 сервисные программы (утилиты); 4 ПО для организации и работы с сетью; 5 и другие.\\
\hline
ЗАО «НПО «Эшелон» & аккредитовано в качестве органа по сертификации ФСТЭК России и испытательных лабораторий в системах обязательной сертификации Минобороны России, ФСБ России, ФСТЭК России и добровольной сертификации «АйТи-Сертифика». & 1 соответствие требованиям по защищенности от несанкционированного доступа к информации; 2 отсутствие недекларированных возможностей (в т.ч. программных закладок), 3 соответствие заданию безопасности (по требованиям ГОСТ Р ИСО/МЭК 15408);\\
 & &4 соответствие реальных и декларируемых в документации функциональных возможностей (в т. числе и на соответствие требованиям Технических Условий);\\
 & &5 и другие.\\
\hline
Центр сертификации «Мостест» & Структурное подразделение органа по сертификации «Гортест», предоставляет полный комплекс услуг, связанных с подготовкой и оформлением различных разрешительных документов. & 1 автоматизированных систем, в т.ч. проектирования (САПР) и управления различными технологическими процессами (АСУ ТП), систем управления отраслями и объединениями, ПО для технологической подготовки производства;\\
 & & 2 систем программирования, утилит, операционных систем;\\
 & & 3 сетевого ПО и приложений мультимедиа-и т.д.;\\ 
 & &4 систем управления базами данных, баз данных, информационных и справочных систем;\\ 
 & &5 электронных изданий, архивов, таблиц;\\ 
 & &6 и другое.\\ 
\hline
\end{longtable}
\end{center}

В данном разделе отчета была исследована информация о сертификации программных средств и др. Также были найдены организации, которые могут произвести сертификацию нашего программного комплекса. Это необходимо для возможности использовать данных комплекс для работы с информацией различного уровня конфиденциальности. Что в свою очередь даст возможность использовать данный программный комплекс в государственных и коммерческих учреждениях для проведения компьютерной экспертизы. Также сертификация обеспечит высокий показатель качества программного комплекса для выхода на рынок.
