\newpage
\ESKDthisStyle{empty}
\paragraph{\hfill РЕФЕРАТ \hfill}
Курсовая работа содержит \ESKDtotal{page} страниц, \ESKDtotal{figure} рисунка, \ESKDtotal{table} таблицы, \ESKDtotal{bibitem} источников, \ESKDtotal{appendix} приложение.

%допилить ключевые слова
КОМПЬЮТЕРНАЯ ЭКСПЕРТИЗА, ФОРЕНЗИКА, ЛОГИ, QT, XML, GIT, LATEX, MOZILLA THUNDERBIRD, MS OUTLOOK, WINDOWS, PST, MSG, RTF, HTML, БИБЛИОТЕКИ, РЕПОЗИТОРИЙ, ПОЧТОВЫЙ КЛИЕНТ, SQLLITE, РЕЕСТР, МЕТА-ДАННЫЕ, READPST, JPEG, PNG, ID3V1, JFIF, RIFF, CHUNK, DBX, C++.

Цель работы --- создание программного комплекса, предназначенного для проведения компьютерной экспертизы.

Задачей, поставленной на данный семестр, стало написание программного комплекса, имеющего следующие возможности: 
\begin{enumerate}
\item \textbf{ПРАВИТЬ}
\end{enumerate}

Результаты работы в данном семестре:

\begin{itemize}
\item \textbf{ПРАВИТЬ}
\end{itemize}
5 Заключение
В этом семестре было проделано:
Исследование юнит-тестирования в инструментарии Qt;
Имплементация юнит-тестирования для пробной программы;
Имплементация юнит-тестирования для модуля, сканирующего медиа-файлы;

Пояснительная записка выполнена при помощи системы компьютерной вёрстки \LaTeX.
