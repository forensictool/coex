\newpage
\ESKDthisStyle{empty}
\paragraph{\hfill РЕФЕРАТ \hfill}
Курсовая работа содержит \ESKDtotal{page} страниц, \ESKDtotal{figure} рисунка, \ESKDtotal{table} таблицы, \ESKDtotal{bibitem} источников, \ESKDtotal{appendix} приложение.

КОМПЬЮТЕРНАЯ ЭКСПЕРТИЗА, ФОРЕНЗИКА, ЛОГИ, QT, XML, GIT, GITLAB, LATEX, MOZILLA THUNDERBIRD, MOZILLA FIREFOX, MS OUTLOOK, WINDOWS, HTML5, CSS3, БИБЛИОТЕКИ, РЕПОЗИТОРИЙ, ПОЧТОВЫЙ КЛИЕНТ, МЕТА-ДАННЫЕ, ID3, JFIF, RIFF, C++, ISSUE, NGINX, GUI, BASH, APACHE, UNIT-ТЕСТИРОВАНИЕ.

Цель работы --- создание программного комплекса, предназначенного для проведения компьютерной экспертизы.

Среди задач, поставленных на данный семестр, было: 
\begin{itemize}
  \item определение индивидуальных задач для каждого участника проектной группы;
  \item исследование предметных областей в рамках индивидуальных задач; 
  \item создание репозитория проекта;
  \item дизайн, верстка и развертывание сайта проекта;
  \item сборка программного пакета проекта;   
  \item доработка программных модулей.
\end{itemize}

Результаты работы в данном семестре:

\begin{itemize}
  \item доработан программный модуль, определяющий ОС;
  \item разработан графический интерфейс пользователя системы;
  \item доработан программный модуль, осуществляющий нахождение медиа-файлов;
  \item доработан программный модуль для сбора истории посещений браузера Mozilla Firefox;    
  \item собран установочный .deb-пакет системы компьютерной экспертизы;
  \item доработан программный модуль для сбора информации из почтового клиента MS Outlook;
  \item созданы удаленный репозиторий и сайт проекта;  
  \item проведено Unit-тестирование в инструментарии Qt на примере модуля, сканирующего медиа-файлы;
  \item внесены поправки, изменения и доработки в исходный код проекта.
\end{itemize}

Пояснительная записка выполнена при помощи системы компьютерной вёрстки \LaTeX.
