Компьютерно-техническая экспертиза – это самостоятельный род судебных экспертиз, относящийся к классу инженерно-технических экспертиз, проводимых в следующих целях: определения статуса объекта как компьютерного средства, выявление и изучение его роли в рассматриваемом деле, а так же получения доступа к информации на электронных носителях с последующим всесторонним её исследованием \cite{fedotovforenzika}.
Компьютерная экспертиза помогает получить доказательственную информацию и установить факты, имеющие значение для уголовных, гражданских и административных дел, сопряжённых с использованием компьютерных технологий. Для проведения компьютерных экспертиз необходима высокая квалификация экспертов, так как при изучении представленных носителей информации, попытке к ним доступа и сбора информации возможно внесение в информационную среду изменений или полная утрата важных данных.

Компьютерная экспертиза, в отличие от компьютерно-технической экспертизы, затрагивает только информационную составляющую, в то время как аппаратная часть и её связь с программной средой не рассматривается.

На протяжении предыдущих семестров нами были рассмотрены такие направления компьютерной экспертизы, как исследование файловых систем, сетевых протоколов, организация работы серверных систем, механизм журналирования событий. Также нами были изучены основные задачи, которые ставятся перед сотрудниками правоохранительных органов, которые проводят компьютерную экспертизу, и набор чуществующих утилит, способных помочь эксперту в проведении компьютерной экспертизы. Было выявлено, что существует множество разрозненных программ, предназначенных для просмотра лог-файлов системы и таких приложений, как мессенджеры и браузеры, но для каждого вида лог-файлов необходимо искать отдельную программу. Так как ни одна из них не позволяет эксперту собрать воедино и просмотреть все логи системы, браузеров и мессенджеров, было решено создать для этой цели собственный автоматизированный комплекс, которому на данный момент нет аналогов.
