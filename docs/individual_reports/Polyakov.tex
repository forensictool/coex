\newpage

\subsection{Введение}

Задачей поставленной на данный семестр стало написание автоматизированного экспертизного? комплекса, имеющего следующие возможности: 

\begin{enumerate}
\item сбор и анализ событий системных журналов операционной системы;
\item сбор и анализ информации из журналов истории браузеров;
\item сбор и анализ истории переписки мессенджеров;
\item сбор и анализ событий журнальных файлов приложений;
\item обнаружение сетевых параметров системы;
\item поиск файлов по имени;
\end{enumerate}

Моя задача на данный семестр . 

\begin{enumerate}
\item сбор и анализ истории переписки мессенджеров;
\item написание модулей для автоматизированного комплекса;
\end{enumerate}

Для упрощения разобьем задачу, на подзадачи
\begin{enumerate}
\item определить места хранения переписки пользователя
\item определить формат хранения переписки
\item разработать парсер для каждого из возможных форматов
\item выделить важную информацию их каждой записи
\item автоматизировать процесс поиска журнальных файлов
\item производить сохранение полученных информации формат XML
\end{enumerate}

\subsection{Определить формат хранения переписки.}

Приложение «skype» хранит переписку локально на машинах пользователей или же возможна синхронизация с машин других пользователей [1]. Формат хранения: реляционная база данных SQLite.\\

По умолчанию файлы распологаются в каталоге: “WINDOWS_DRIVE”/Users/”USER_NAME”/AppData/Roaming/Skype/”USER_SKYPE_NAME”

Основная интересующая нас информация находится в main.db

Приложение «pidgin» хранит лог файлы локально на машине пользователя в формате HTML и TXT. По умолчанию лог файли хранятся в .HTML файле. Настройки программы и подключенных аккаунтов в XML, но особой ценности на данный момент не представляют.
По умолчанию файлы распологаются в каталоге: “WINDOWS_DRIVE”/Users/”USER_WIN_NAME”/AppData/Roaming/.purple/logs/”USER_PIDGIN_NAME”
Основная интересующая нас информация хранится в файлах с такой маской имени YEAR-MONTH-DATE.TIME.html пример:2013-03-02.004915+0700NOVT.htm\\

\charter*{Определить места хранения переписки пользователя.}

Определение месторасположения файлов переписки происходит следующим образом. Для при монтированному образу запускается модуль который сужает область поиска, сканируя только нужные места в образе (к примеру не всю папку %ProrgamFiles, а только %ProrgamFiles/Skype). Сканированием папки занимается класс QDirIterator. После вызова происходит поочередный обход по каждому файлу в директории и под директории. Проверка полученного имени файла  осуществляется по маске, если реакция на маску положительная, происходит добавление в список обрабатываемых файлов. 
\\

\charter*{Разбор найденых файлов.}

В зависимости от обрабатываемых логов, запускается нужный модуль.\\ 
Для Skype.
Структура рассматриваемого файла.
main.db  содержит 18 таблиц. и ещё немного статистики.
"DbMeta"   "Contacts"   "LegacyMessages"
"Calls"     "Accounts"   "Transfers"   
"Voicemails"   "Chats"      "Messages"   
"ContactGroups"   "Videos"   "SMSes"
"CallMembers"   "ChatMembers"   "Alerts"
"Conversations"     "Participants"   "VideoMessages"

Таблицы которые нам интересны, на данный момент:    
 "Contacts"    "Messages"    "Chats" 
Из полученного ранее списка, найденные файлы поочередно обрабатывается. Для начала, файлов может быть несколько т.к для каждого залогинившегося в Скайпе создается свой профиль и main.db и затем разбираются.\\
Для работы с БД создается её дамп. Затем, посредством SQL-запроса извлекаются данные.\\
Из рассматриваемых баз можно извлечь контактную книгу, историю переписки, звонков, общие чаты и прочее.\\

Для Pidgin. 

Структура рассматриваемого файла. У каждого лог файла есть заголовок, в котором записан ID_Chat, Дата беседы, логин пользователя и используемый протокол. Затем идет "тело" в котором записано, время, автор сообщения и сообщение.\\
Пример "Заголовка" <head><meta http-equiv="content-type" content="text/html; charset=UTF-8"><title>Conversation with 0dpkhcz6clufs2kozj82uqif30@public.talk.google.com at Чт. 24 окт. 2013 23:40:21 on user.fox@gmail.com/ (jabber)</title></head>\\

Пример "Тела" <font color="#16569E"><font size="2">(23:44:52)</font> <b>user.fox@gmail.com/95C9F047:</b></font> Hello) <br/>\\
Точно так же из полученного списка, найденные файлы поочередно открываются. Открываем файл на чтение, затем при помощи регулярных выражений извлекаем нужную информацию.\\

Пример регулярного вырожения.

Приложение «pidgin» хранит лог файлы локально на машине пользователя в формате HTML и TXT. По умолчанию лог файли хранятся в .HTML файле. Настройки программы и подключенных аккаунтов в XML, но особой ценности на данный момент не представляют.
По умолчанию файлы распологаются в каталоге: “WINDOWS_DRIVE”/Users/”USER_WIN_NAME”/AppData/Roaming/.purple/logs/”USER_PIDGIN_NAME”
Основная интересующая нас информация хранится в файлах с такой маской имени YEAR-MONTH-DATE.TIME.html пример:2013-03-02.004915+0700NOVT.htm

\subsection{Определить места хранения переписки пользователя.}

Определение месторасположения файлов переписки происходит следующим образом. Для при монтированному образу запускается модуль который сужает область поиска, сканируя только нужные места в образе (к примеру не всю папку %ProrgamFiles, а только %ProrgamFiles/Skype). Сканированием папки занимается класс QDirIterator. После вызова происходит поочередный обход по каждому файлу в директории и под директории. Проверка полученного имени файла  осуществляется по маске, если реакция на маску положительная, происходит добавление в список обрабатываемых файлов. 


\subsection{Парсинг найденых файлов.}

В зависимости от обрабатываемых логов, запускается нужный модуль. Из полученного ранее списка, найденные файлы поочередно открываются и парсятся. Методами класса QstringList происходит резанье строк и добавление в список. Для парсинга полученного списка используется регулярные вырожения использующие класс QRegExp. 

\subsection{Сохранение полученного в XML.}

Соранение полученных данных происходит в ранее выбраный формат XML. Для этого используется класс QXmlStreamReader и QxmlStreamWriter.
Класс QXmlStreamWriter представляет XML писателя с простым потоковым API.\\

QXmlStreamWriter работает в связке с QXmlStreamReader для записи XML. Как и связанный класс, он работает с QIODevice, определённым с помощью setDevice ().\\

Сохранение данных реализованно в классе WriteMessage. В методе WriteMessages. (Только не таблица и поля, а хмл ный аналог)Формат хранения: таблица INFO и таблица MESSAGE. Таблица MESSAGE содержит поля message author, dataTime, message.	Таблица INFO содержит поля chathID, account, data, protocol.\\

Контактная книга сохраняется ..

Список использованной литературы:
1 -  http://community.skype.com/t5/Security-Privacy-Trust-and/Is-chat-history-stored-on-Skype-servers/td-p/472379 вопрос где скайп логи хранит
