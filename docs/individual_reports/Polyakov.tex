\newpage

\subsection{Введение}

Задачей поставленной на данный семестр стало написание автоматизированного экспертизного? комплекса, имеющего следующие возможности: 

\begin{enumerate}
\item сбор и анализ событий системных журналов операционной системы;
\item сбор и анализ информации из журналов истории браузеров;
\item сбор и анализ истории переписки мессенджеров;
\item сбор и анализ событий журнальных файлов приложений;
\item обнаружение сетевых параметров системы;
\item поиск файлов по имени;
\end{enumerate}

Моя задача на данный семестр . 

\begin{enumerate}
\item сбор и анализ истории переписки мессенджеров;
\item написание модулей для автоматизированного комплекса;
\end{enumerate}

Для упрощения разобьем задачу, на подзадачи
\begin{enumerate}
\item определить места хранения переписки пользователя
\item определить формат хранения переписки
\item разбор найденых файлов
\item автоматизировать процесс поиска журнальных файлов
\item производить сохранение полученных информации формат XML
\end{enumerate}

\subsection{Определить формат хранения переписки.}

Приложение «skype» хранит переписку локально на машинах пользователей или же возможна синхронизация с машин других пользователей [1]. Формат хранения: реляционная база данных SQLite.\\
Кроме сохранения данных о действиях пользователей
в основной базе , имеющей имя файла «main.db», Skype сохраняет сведения о работе программы во временных файлах (chatsync). Эти файлы имеют расширение «.dat» и цифро-буквенные имена (например «0172b0a519e2c584»).[2]


Приложение «pidgin» хранит лог файлы локально на машине пользователя в формате HTML и TXT. По умолчанию лог файли хранятся в .HTML файле. Настройки программы и подключенных аккаунтов в XML, но особой ценности на данный момент не представляют.

\subsection{Определить места хранения переписки пользователя.}

Определение месторасположения файлов переписки происходит следующим образом. Для при монтированному образу запускается модуль который сужает область поиска, сканируя только нужные места в образе (к примеру не всю папку \%AppData, а только \%AppData\%\\Roaming\\Skype\\). 
По умолчанию файлы распологаются в каталоге: \"WINDOWS_DRIVE\”\/Users\/\”USER_WIN_NAME\”\/\%AppData\%\/Roaming\/.purple\/logs\/\”USER_PIDGIN_NAME\”

Основная интересующая нас информация хранится в файлах с такой маской имени YEAR-MONTH-DATE.TIME.html пример:2013-03-02.004915+0700NOVT.htm\\
По умолчанию файлы распологаются в каталоге: 
\"\%AppData\%\\Roaming\\Skype\\Профиль\\\"

Основная интересующая нас информация находится в main.db.\\

\subsection{Разбор найденых файлов.}

В зависимости от обрабатываемых логов, запускается нужный модуль.\\ 
\subsubsection Skype.
Структура рассматриваемого файла.
main.db  содержит 18 таблиц. и ещё немного статистики.
\"DbMeta\\"   \"Contacts\"   \"LegacyMessages\"
\"Calls\"     \"Accounts\"   \"Transfers\"   
\"Voicemails\"   \"Chats\"      \"Messages\"   
\"ContactGroups\"   \"Videos\"   \"SMSes\"
\"CallMembers\"   \"ChatMembers\"   \"Alerts\"
\"Conversations\"     \"Participants\"   \"VideoMessages\"

Таблицы которые нам интересны, на данный момент:    
 \begin{enumerate}
\item Contacts
\item Messages
\item Chats
\item Calls
\item CallMembers
\item Conversations
\end{enumerate}

В таблице Contacts находятся все контакты, причем даже те, что были удалены, и уже не показываются в клиенте.

select skypename, 
       fullname, 
       given_displayname, 
       birthday, 
       case gender when 1 then \'Мужской\' when 2 then \'Женский\' else \'Не указан\' end as \\"Пол\", 
       case availability when 0 then \'Удален\' when 8 then \'Не предоставил данные\' else \'\' end as \"Доступность\",
       strftime(\'\%d.\%m.\%Y \%H:\%M:\%S\',lastonline_timestamp, \'unixepoch\', \'localtime\') as \"Последний раз был в сети\"
  from contacts



В таблицах Calls и CallMembers содержатся, соответственно, история звонков и их участников.

select calls.id as \"ID разговора\",
       coalesce(contacts.displayname, accounts.fullname) as \"Инициатор\",
       strftime(\'\%d.\%m.\%Y \%H:\%M:\%S\',calls.begin_timestamp, \'unixepoch\', \'localtime\') as \"Дата начала\",
       time(calls.duration, \'unixepoch\') as \"Длительность\",
       callmembers.dispname as \"Подключенный участник\",
       strftime(\'\%d.\%m.\%Y \%H:\%M:\%S\',callmembers.start_timestamp, \'unixepoch\', \'localtime\') as \"Дата подключения\",
       time(callmembers.call_duration, \'unixepoch\') as \"Длительность подключения\"
  from calls
       inner join callmembers on calls.id = callmembers.call_db_id
       left  join contacts on calls.host_identity = contacts.skypename
       left  join accounts on calls.host_identity = accounts.skypename



И, наконец, в таблицах Conversations и Messages содержатся данные переписки и сами сообщения.

select conversations.id as \"ID переписки\", 
       conversations.displayname as \"Участники переписки\", 
       messages.from_dispname as \"Автор сообщения\",  
       strftime(\'\%d.\%m.\%Y \%H:\%M:\%S\',messages.timestamp, \'unixepoch\', \'localtime\') as \"Время сообщения\", 
       messages.body_xml as \"Текст сообщения\"
  from conversations
       inner join messages on conversations.id = messages.convo_id
order by messages.timestamp



Для доступа ко всему содержимому базы достаточно иметь доступ к самому файлу — содержимое базы никак не шифруется и не защищается, так что любой человек, который сможет получить доступ к вашему профилю Windows, сможет найти список контактов, просмотреть историю звонков и прочитать всю переписку. 


\subsubsection Pidgin. 

Приложение «pidgin» хранит лог файлы локально на машине пользователя в формате .HTML и .TXT. По умолчанию лог файлы хранятся в формате .HTML. Настройки программы и подключенных аккаунтов в файлах формата XML, но особой ценности на данный момент не представляют. 
Примечание: Кроме account.xml - он хранит не шифрованые пароли для всех подключенных чатов.[3]


Структура рассматриваемого файла.\\ 
У каждого лог файла есть заголовок находящийся между тегов title. В котором записан ID_Chat, дата начала переписки, логин пользователя и используемый протокол.\\
Затем идет \"тело\" в котором описывается обмен сообщениями в формате, время, автор сообщения и сообщение.\\

Пример \"Заголовка\" <head><meta http-equiv=\"content-type\" content=\"text/html; charset=UTF-8\"><title>Conversation with 0dpkhcz6clufs2kozj82uqif30@public.talk.google.com at Чт. 24 окт. 2013 23:40:21 on user.fox@gmail.com/ (jabber)</title></head>\\

Пример \"Тела\" <font color=\"#16569E\"><font size=\"2\">(23:44:52)</font> <b>user.fox@gmail.com/95C9F047:</b></font> Hello) <br/>\\

Точно так же из полученного списка, найденные файлы поочередно открываются, на чтение. Разбор открытого файла решено осуществлять при помощи регулярных выражений, описанных в класс QRegExp.\\


Пример регулярного вырожения, для работы с лог-файлами в формате .HTML

QRegExp rxHead(\".\*h3.\*with (.\*) at (.\*) on (.\*)\\/ \\((.*)\\)")

QRegExp rxBody(\".\*(\\d\{2\}:\\d\{2\}:\\d\{2\}).\*b\\\>(.*):\\\<\\\/b.\*font\\>(.\*)\\\<br\")


Пример регулярного вырожения, для работы с лог-файлами в формате .TXT

QRegExp rxHead(\".\*with (.\*) at (.\*) on (.\*)\\\/ \\((.\*)\\)\")

QRegExp rxBody(\"\\((\\d\{2\}:\\d\{2\}:\\d\{2\})\\)[ ]\*(.\*)\:(.\*)\$\")

\subsection{автоматизировать процесс поиска журнальных файлов}
 Ну тут я пока ХЗ ... 
Для автоматизации процесса решено написать программный комплекс с подключаемыми модулями. Основа комплекса написана на языке С++ с использованием фраемворка Qt.
Qt позволяет запускать написанное с его помощью ПО в большинстве современных операционных систем путём простой компиляции программы для каждой ОС без изменения исходного кода.\\
Включает в себя все основные классы, которые могут потребоваться при разработке прикладного программного обеспечения, начиная от элементов графического интерфейса и заканчивая классами для работы с сетью, базами данных и XML. Qt является полностью объектно-ориентированным, легко расширяемым и поддерживающим технику компонентного программирования.

Сканированием папки занимается класс QDirIterator. После вызова происходит поочередный обход по каждому файлу в директории и под директории. Проверка полученного имени файла  осуществляется по маске, если реакция на маску положительная, происходит добавление в список обрабатываемых файлов. 


\subsection{Сохранение полученного в XML.}

Соранение полученных данных происходит в ранее выбраный формат XML(Extensible Markup Language). Для этого используется класс QXmlStreamReader и QxmlStreamWriter.
Класс QXmlStreamWriter представляет XML писателя с простым потоковым API.\\

QXmlStreamWriter работает в связке с QXmlStreamReader для записи XML. Как и связанный класс, он работает с QIODevice, определённым с помощью setDevice().\\

Сохранение данных реализованно в классе WriteMessage. В методе WriteMessages. 

Стуктура сохраняемого документа:
  Пролог XML документа. Все документы XML начинаются с пролога (prolog). Пролог сообщает, что документ написан на XML, а также указывает, какая версия XML при этом использовалась.  
  Элемент Messages с атрибутом Messenger, ему присваетвается имя рассматриваемого приложения.
  Элемент INFO содержищий атрибуты chathID, account, data, protocol котрорым присваевается, соответственно: номер чата??, Полная дата в формате  День.Число Месяц. Год Час:Мин:Сек, аккаунт с которого происходил обмен сообщениями и протокол используемый для передачи сообщений. 
  Элемент MESSAGE содержищий атрибуты author, dataTime, message.


\subsection{Список использованной литературы:}
1 -  http://community.skype.com/t5/Security-Privacy-Trust-and/Is-chat-history-stored-on-Skype-servers/td-p/472379 вопрос где скайп логи хранит
2 - Криминалистическое исследование поврежденных и частично удаленных баз Skype http://computer-forensics-lab.org/pdf/00030.pdf
3 - Ответ на офф сайте, какого мол фига мы должны ваши пароли шифровать .. https://developer.pidgin.im/wiki/PlainTextPasswords