\section{TaskSearchArchive}

\subsection{Структура RAR архива}
RAR — проприетарный формат сжатия данных и условно-бесплатная программа-архиватор. Имеет расширение .rar, .rev, .r00 или .r01

Определение архивного файла RAR
RAR-архив состоит из блоков переменной длины с заголовками по 7 байт каждый. Любой архив содержит как минимум два блока MARK_HEAD и MAIN_HEAD. Первый содержит информацию о том, что перед нами RAR, и выглядит как \"0x 52 61 72 21 1A 07 00\" в HEX\'ах, или 0x 52 45 7E 5E” в старых версиях . Третий байт 0х72 как раз таки указывает на то, что это Marker Header. Слово 00 07 в little-endian содержит длину блока. Как раз таки 7 байт.[1]

Второй блок Main Header начинается сразу же после первого и должен содержать 13 байт и иметь маркировочный байт равным 0x73. После него в файле уже начинаются данные — будь-то сжатый файл (маркет 0х74 в третьем байте заголовка блока), комментарий к архиву, дополнительная информация или, к примеру, recovery-запись.[2]

Формат архивного файла RAR
                
Файл архива состоит из блоков разной длины.
В самом начале архива стоит блок-маркер, после которого идёт блок заголовка архива, за которым в произвольном порядке следуют блоки остальных типов.
                
                Каждый блок начинается со следующих полей:         
%Это таблица                 
HEAD_CRC --2 байта --CRC всего блока или его части
HEAD_TYPE --1 байт --Тип блока 
HEAD_FLAGS --16 бит ( =2 байта)--Флаги(*) блока     
HEAD_SIZE --2 байта --Размер блока 
ADD_SIZE --4 байта --Добавление к размеру блока
%конец таблицы 
            
Во всех блоках следующие биты в HEAD_FLAGS (*) имеют одинаковое значение:         
предпоследний 14-й (смещение 0х4000) — если = true (*), то старые версии RAR будут игнорировать этот блок и удалять его при изменении архива; иначе блок копируется в новый архивный файл при изменении архива;
последний 15-й (смещение 0х8000) — если = true, то поле ADD_SIZE присутствует в блоке, иначе — отсутствует.
                
Заявленные типы блоков (возможные значения HEAD_TYPE):             
\begin{itemize}
\item 0x72 (*) — блок-маркер;
\item 0x73 — заголовок архива;
\item 0x74 — заголовок файла;
\item 0x75 — заголовок комментария старого типа;
\item 0x76 — электронная подпись старого типа;
\item 0x77 — субблок старого типа;
\item 0x78 — информация для восстановления старого типа;
\item 0x79 — электронная подпись старого типа;
\item 0x7A — субблок.
\end{itemize}

Форматы блоков
Блок-маркер (MARK_HEAD)
\begin{itemize}
\item HEAD_CRC = 0x6152;
\item HEAD_TYPE = 0x72;
\item HEAD_FLAGS = 0x1a21;
\item HEAD_SIZE = 0x0007;
\end{itemize}

Заголовок архива (MAIN_HEAD)
\begin{itemize}
\item HEAD_CRC = CRC полей от HEAD_TYPE до RESERVED2;
\item HEAD_TYPE = 0x73;
\item HEAD_FLAGS ( =16 бит):                     
\end{itemize}

0-й бит (*) (смещение 0x0001 (*)) — Атрибут тома (том многотомного архива) (*)
1-й бит (смещение 0x0002) — Присутствует архивный комментарий (RAR 3.x использует отдельный блок комментария и не устанавливает этот флаг)
2-й бит (смещение 0x0004) — если = true, то архив заблокирован для изменений
3-й бит (смещение 0x0008) — если = true, то это — непрерывный (solid) архив
4-й бит (смещение 0x0010) — если = true, то используется новая схема именования томов ('volname.partN.rar'), иначе — старая ('volname.rN')
5-й бит (смещение 0x0020) — Присутствует информация об авторе или электронная подпись (AV) (RAR 3.x не устанавливает этот флаг)
6-й бит (смещение 0x0040) — Присутствует информация для восстановления
7-й бит (смещение 0x0080) — Заголовки блоков зашифрованы
8-й бит (смещение 0x0100) — Первый том (устанавливает только RAR 3.0 и выше)
Остальные биты (с 9 по 15-й) зарезервированы для внутреннего использования

\begin{itemize}
\item HEAD_SIZE (*) = Общий размер архивного заголовка, включая архивные комментарии;
\item RESERVED1 (2 байта) - Зарезервировано;
\item RESERVED2 (4 байта) - Тоже зарезервировано;
\end{itemize}

\subsection{Структура ZIP архива}                

ZIP файл состоит из трех областей:
сжатые/несжатые данные, (последовательность структур Local File Header, сами данные и необязательных Data descriptor)
центральный каталог (последовательность структур Central directory file header)
описание центрального каталога (End of central directory record)
С начала файла идет набор из Local File Header, непосредственно данные и (необязательно) структура Data descriptor. Затем структуры типа Central directory file header для каждого файла и папки в ZIP архиве и завершает все это структура End of central directory record.
Local File Header
Используется для описания метаданных файла (имя файла, контрольная сумма, время и дата модификации, сжатый/несжатый размер). Как правило сразу после этой структуры следует содержимое файла.
struct LocalFileHeader
\begin{itemize}
    \item uint32_t signature;	// Обязательная сигнатура, равна 0x04034b50
    \item uint16_t versionToExtract;	// Минимальная версия для распаковки
    \item uint16_t generalPurposeBitFlag;	// Битовый флаг
    \item uint16_t compressionMethod;	// Метод сжатия (0 - без сжатия, 8 - deflate)
    \item uint16_t modificationTime;	// Время модификации файла
    \item uint16_t modificationDate;	// Дата модификации файла
    \item uint32_t crc32;				// Контрольная сумма
    \item uint32_t compressedSize;	// Сжатый размер
    \item uint32_t uncompressedSize;	// Несжатый размер
    \item uint16_t filenameLength;	// Длина название файла
    \item uint16_t extraFieldLength;	// Длина поля с дополнительными данными
    \item uint8_t *filename;			// Название файла (размером filenameLength)
    \item uint8_t *extraField;		// Дополнительные данные (размером extraFieldLength)
\end{itemize}
Сразу после этой структуры идут данные размером compressedSize при использовании сжатия или размером uncompressedSize в противном случае.
Иногда бывает невозможно вычислить данные на момент записи LocalFileHeader, тогда в crc32, compressedSize и uncompressedSize записываются нули третий бит в generalPurposeBitFlag ставится в единицу и после LocalFileHeader добавляется структура типа Data descriptor.
Data descriptor
Если по какой-то причине содержимое файла невозможно создать одновременно с заголовком типа Local File Header, то сразу после него следует структура Data descriptor, где идет находится дополнение метаданных для Local File Header (контрольная сумма, сжатый/несжатый размер). Откровенно говоря, мне такие файлы не попадались, поэтому больше того, чем написано в википедии сказать не могу.
struct DataDescriptor
\begin{itemize}
\item    uint32_t signature;		// Необязательная сигнатура, равна 0x08074b50
\item    uint32_t crc32;			// Контрольная сумма
\item    uint32_t compressedSize;	// Сжатый размер
\item    uint32_t uncompressedSize;	 // Несжатый размер
\end{itemize}
Central directory file header
Расширенное описание метаданных файла. Содержит дополненную версию LocalFileHeader (добавляются поля номер диска, файловые атрибуты, смещение до Local File Header от начала ZIP файла).
struct CentralDirectoryFileHeader
\begin{itemize}
    // Обязательная сигнатура, равна 0x02014b50 
\item    uint32_t signature;    // Версия для создания
\item    uint16_t versionMadeBy;    // Минимальная версия для распаковки
\item    uint16_t versionToExtract;    // Битовый флаг
\item    uint16_t generalPurposeBitFlag;    // Метод сжатия (0 - без сжатия, 8 - deflate)
\item    uint16_t compressionMethod;    // Время модификации файла
\item    uint16_t modificationTime;    // Дата модификации файла
\item    uint16_t modificationDate;    // Контрольная сумма
\item    uint32_t crc32;    // Сжатый размер
\item    uint32_t compressedSize;    // Несжатый размер
\item    uint32_t uncompressedSize;    // Длина название файла
\item    uint16_t filenameLength;    // Длина поля с дополнительными данными
\item    uint16_t extraFieldLength;    // Длина комментариев к файлу
\item    uint16_t fileCommentLength;    // Номер диска
\item    uint16_t diskNumber;    // Внутренние аттрибуты файла
\item    uint16_t internalFileAttributes;    // Внешние аттрибуты файла
\item    uint32_t externalFileAttributes;    // Смещение до структуры LocalFileHeader
\item    uint32_t localFileHeaderOffset;    // Имя файла (длиной filenameLength)
\item    uint8_t *filename;    // Дополнительные данные (длиной extraFieldLength)
\item    uint8_t *extraField;    // Комментарий к файла (длиной fileCommentLength)
\item   uint8_t *fileComment;
\end{itemize}
End of central directory record (EOCD)
Эта структура записывается в конце файла. Содержит следующие поля: номер текущего диска, количество записей Central directory file header в текущем диске, общее количество записей Central directory file header.

\begin{itemize}
// Обязательная сигнатура, равна 0x06054b50
\item uint32_t signature; // Номер диска
\item uint16_t diskNumber;   // Номер диска, где находится начало Central Directory
\item uint16_t startDiskNumber;    // Количество записей в Central Directory в текущем диске
\item uint16_t numberCentralDirectoryRecord;    // Всего записей в Central Directory
\item uint16_t totalCentralDirectoryRecord;    // Размер Central Directory
\item uint32_t sizeOfCentralDirectory;    // Смещение Central Directory
\item uint32_t centralDirectoryOffset;    // Длина комментария
\item uint16_t commentLength;    // Комментарий (длиной commentLength)
\item uint8_t *comment;
\end{itemize}

Папки в ZIP файле представлены двумя структурами Local File Header и Central directory file header с нулевым размером и контрольной суммой. Название папки заканчивается слешем «/».

\subsection{Рефакторинг старого кода :: COEX}
Рефакторинг — это процесс улучшения написанного ранее кода путем такого изменения его внутренней структуры, которое не влияет на внешнее поведение.
Во многом при рефакторинге лучше полагаться на интуицию, основанную на опыте. Тем не менее имеются некоторые видимые проблемы в коде, требующие рефакторинга:
\begin{itemize}
\item дублирование кода;
\item длинный метод;
\item длинный список параметров;
\item «завистливые» функции — это метод, который чрезмерно обращается к данным другого объекта;
\item избыточные временные переменные;
\item классы данных;
\item несгруппированные данные.
\end{itemize}

\subsubsection{Рефакторинг плагина Pidgin}

\begin{itemize}
\item Удаление дублирующегося кода
\item Вынесение блоков кода отвечающих за разбор xml, html логов из TaskPidginWin::execute в отдельные функции (processingLogPidgin, processingContactPidgin, processingAccountPidgin).
\item Замена алгоритма поиска файлов, на более понятный
\item Использование qDebug вместо std::cout
\end{itemize}

\subsubsection{Рефакторинг плагина Skype}

\begin{itemize}
\item Удаление цикличного подключения к БД.
\item Замена алгоритма поиска файлов, на более понятный.
\item Использование qDebug вместо std::cout.
\end{itemize}

\subsection{Совместимость плагинов под формат NoSql БД Solr}

Для совместимости добавления в БД solr необходимо выполнить следующие требования
[https://wiki.apache.org/solr/UpdateXmlMessages]
Документ добавляющий запись в БД имеет следющий формат: 
Корневой тег add,  затем каждая запись/событие помещается в тег doc, где далее/глубже лежат field с нашими  данными. Атрибуты field name определяются разработчиком, в зависимости от необходимости, однако каждая запись должна иметь уникальный id. Так же добавляется в sceme.xml  на сервере solr, для того чтобы БД знала какие данные в нее импортируются, и как с ними работать.
%Таблица
Старый формат
<?xml version=\"1.0\" encoding=\"UTF-8\"?>
<Messages  Messenger=\"pidgin\">
    <info_account name=\"\" email=\"fox.user.3@gmail.com/\" password=\"kpdroscfozyyvsyk\" protocol=\"prpl-jabber\"/>
    <info_account name=\"Игорь Поляков\" email=\"fox.user.3@gmail.com\" password=\"this_is_real_passowrd\" protocol=\"prpl-vkcom\"/>
    <info_account name=\"Игорь Поляков\" email=\"fox.user.3@gmail.com\" password=\"this_is_real_passowrd\" protocol=\"prpl-vkcom\"/>
</Messages >
Новый формат
<?xml version=\"1.0\" encoding=\"UTF-8\"?>
<add>
    <doc>
        <field name=\"id\">pidgin_24d7a3ebd9f601666a7ba27225e71854</field>
        <field name=\"doc_type\">account</field>
        <field name=\"application\">pidgin</field>
        <field name=\"account_id\"></field>
        <field name=\"account_mail\">fox.user.3@gmail.com/</field>
        <field name=\"account_protocol\">prpl-jabber</field>
        <field name=\"account_password\">kpdroscfozyyvsyk</field>
    </doc>
</add>
%Конец таблици
\section{Ссылки}
1 - http://formatsfiles.narod.ru/rar.html
2 - http://www.forensicswiki.org/wiki/RAR