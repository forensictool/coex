\documentclass[
% - Язык документа
russian,
% - Используемая кодировка cp1251
cp1251,
% - Размер шрифта
14pt,
% - Отключение дополнительных граф рамки
simple
]{eskdtext}

% - Полуторный интервал
\renewcommand{\baselinestretch}{1.50}
% - Отступ красной строки
\setlength{\parindent}{1.25cm}	
% - Шрифт Times New Roman
\renewcommand{\rmdefault}{ftm}

% - Наименование документа
\ESKDtitle{ }
% - Обозначение документа
\ESKDsignature{ФВС КР. Х.ХХХХХХХ 001 ПЗ}
% - Наименование предприятия
\ESKDcolumnIX{ТУСУР, ФВС, КИБЭВС-1205}
% - Проверил
\ESKDchecker{Давыдова Е.М.}	
% - Литера 
\ESKDletter{У}{}{}
% - Разработал
\ESKDauthor{КИБЭВС-1205}			

% - Убирает точку в списке литературы
\makeatletter
\def\@biblabel#1{#1 }
% - ГОСТ списка литературы
\bibliographystyle{gost780s}

\ESKDsectSkip{subsection}{1em}{1em}
\ESKDsectSkip{section}{1em}{1em}

% - Изменение заголовков
\usepackage{titlesec}
\titleformat{\section}{\normalsize}{\thesection}{1.0em}{}
\titleformat{\subsection}{\normalsize}{\thesubsection}{1.0em}{}
\titleformat{\paragraph}{\normalsize}{\theparagraph}{1.0em}{}

% - Изменение содержания: отступы, убран жирный шрифт
%\usepackage{titletoc}
%\dottedcontents{section}[3em]{}{2em}{5pt}
%\dottedcontents{subsection}[3em]{}{2em}{5pt}

% - Изменение заголовка теоремы и доказательства
\usepackage{amsthm}
\theoremstyle{definition}						
\newtheorem{theorem}{\normalfont Теорема}[subsection]
\renewcommand{\thm@indent}{\hspace{\parindent}}
\renewenvironment{proof}{Доказательство:}

% - Для больших таблиц
\usepackage{longtable}

% - Для вставки гиперссылок
\usepackage[colorlinks]{hyperref}

% - Использование математического шрифта
\usepackage{amsfonts}


\begin{document}

\newpage
\ESKDthisStyle{empty}

\begin{center}
Министерство образования и науки Российской Федерации\\
ФЕДЕРАЛЬНОЕ ГОСУДАРСТВЕННОЕ БЮДЖЕТНОЕ ОБРАЗОВАТЕЛЬНОЕ\\
УЧРЕЖДЕНИЕ ВЫСШЕГО ПРОФЕССИОНАЛЬНОГО ОБРАЗОВАНИЯ\\
ТОМСКИЙ ГОСУДАРСТВЕННЫЙ\\
УНИВЕРСИТЕТ СИСТЕМ УПРАВЛЕНИЯ И РАДИОЭЛЕКТРОНИКИ\\
(ФГБОУ ВПО ТУСУР)\\
Кафедра комплексной информационной безопасности электронно-вычислительных систем (КИБЭВС)\\
\end{center}

\begin{tabbing}
XXXXXXXXXXXXXXXXXXXXXXXXXXX \=
XXXXXXXXXXXXXXXX\kill
\> УТВЕРЖДАЮ\\
\> заведующий каф.КИБЭВС\\
\> \underline{\ \ \ \ \ \ \ \ \ \ \ \ \ \ \ \ \ \ \ \ } А.А. Шелупанов\\
\> "\underline{\ \ \ \ \ }"\underline{\ \ \ \ \ \ \ \ \ \ \ \ \ \ \ \ \ \ \ \ } 2013г.\\
\end{tabbing}

\begin{center}
МАТЕМАТИЧЕСКИЕ ОСНОВЫ ЗАЩИТЫ ИНФОРМАЦИИ\\
Отчет по групповому проектному обучению\\
Группа КИБЭВС-1205\\
\end{center}

\begin{tabbing}
XXXXXXXXXXXXXXXXXXXXXXXXXXX \=
XXXXXXXXXXXXXXXX\kill
\> Ответственный исполнитель\\
\> Студент гр. 520-1\\
\> \underline{\ \ \ \ \ \ \ \ \ \ \ \ \ \ \ \ \ \ \ \ } Черепанов С. А.\\
\> "\underline{\ \ \ \ \ }"\underline{\ \ \ \ \ \ \ \ \ \ \ \ \ \ \ \ \ \ \ \ } 2013г.\\
\ \\
\> Научный руководитель\\
\> Аспирант каф.КИБЭВС\\
\> \underline{\ \ \ \ \ \ \ \ \ \ \ \ \ \ \ \ \ \ \ \ } Кручинин Д. В.\\
\> "\underline{\ \ \ \ \ }"\underline{\ \ \ \ \ \ \ \ \ \ \ \ \ \ \ \ \ \ \ \ } 2013г.\\
\end{tabbing}
\vfill
\begin{center}
2013
\end{center}

\newpage
\ESKDthisStyle{empty}
\paragraph{\hfill РЕФЕРАТ \hfill}
Курсовая работа содержит  59 страниц,  1 таблицу,  14 источников.

ПРОСТЫЕ ЧИСЛА, ТЕСТЫ НА ПРОСТОТУ, ПРОИЗВОДЯЩИЕ ФУНКЦИИ, КОМПОЗИЦИЯ, MAXIMA.

Объектом исследования являются тесты на простоту натурального числа.

Цель работы -- изучение известных алгоритмов проверки простоты числа, изучение современных математических пакетов и математических редакторов.

Достигнутые результаты: изучен математический аппарат построения критериев простоты натурального числа, основанный на теории производящих функций; построены  и проанализированы критерии простоты, полученные на основе двадцати различных внутренних производящих функций композиции производящих функций; освоена программная среда Maxima.

Пояснительная записка выполнена в текстовом редакторе Texmaker 3.5.

\newpage
\ESKDthisStyle{empty}
\paragraph{\hfill Список исполнителей \hfill}

Черепанов С. А. -- ответственный исполнитель, ответственный за написание раздела о производящих функциях, ответственный за проведение построения критериев простоты чисел и их анализа.

Шабля Ю. В. -- ответственный за документацию, ответственный за написание раздела о системе компьютерной алгебры Maxima, ответственный за проведение построения критериев простоты чисел и их анализа.

% Техническое задание:
\newpage
\ESKDthisStyle{empty}
\ \\ 
\newpage
\ESKDthisStyle{empty}
\ \\ 
\newpage
\ESKDthisStyle{empty}
\ \\ 
\newpage
\ESKDthisStyle{empty}
\ \\ 

\newpage
\ESKDstyle{formII}
\renewcommand\contentsname{\hfill Содержание \hfill}
\tableofcontents

\newpage
\ESKDstyle{formIIab}
\section{Введение}
C развитием общества и переходом от индустриального типа к постиндустриальному (информационному) ценность такого ресурса, как информация, выходит на первый план. Как следствие, защита информации становится неотъемлемой частью информационных процессов. Существует целый раздел математики, занимающийся данной проблемой, -- криптография.

Математической основой современной криптографии является теория чисел. Основным понятием теории чисел, применяющимся в области защиты информации, является простое число. Простым числом $p$ называется натуральное число большее единицы, имеющее только два различных натуральных делителя: единицу и само себя. Изучение простых чисел и их свойств ведёт своё начало с древнейших времён. Например, хорошо известны труды древнегреческого математика Евклида, который уже тогда смог предложить доказательство того, что множество простых чисел бесконечно \cite{Euclid}.

К сожалению, на сегодняшний день вопрос простых чисел является основным нерешённым вопросом целочисленной арифметики и исследования в данной области имеют высокую научную ценность.

Проблема заключается в определении, является ли заданное натуральное число простым. Решение данной проблемы принято делить на следующие три основные задачи:

\begin{itemize}
\item построение простых чисел (на данный момент актуально при построении огромных простых чисел, применяемых в некоторых криптографических системах);
\item проверка натуральных чисел на простоту (поиск наиболее эффективного и точного метода тестирования числа на простоту);
\item факторизация чисел (разложение натурального числа в произведение простых множителей).
\end{itemize} 

В данном проекте рассматривается решение только второй задачи (проверка на простоту), решение которой имеет наибольшее техническое и экономическое значение среди приведенных выше задач  и  позволит улучшить работу современных криптографических систем.

Существует множество критериев, по которым классифицируют алгоритмы проверки случайного натурального числа на простоту, но основным является критерий достоверности полученного результата. Согласно данному критерию алгоритмы делятся на детерминированные и вероятностные.

Особенность детерминированных тестов заключается в том, что они гарантированно выдают точный ответ: простое или составное заданное натуральное число. Но главный недостаток существующих детерминированных тестов - это огромная вычислительная сложность, и, следовательно, невозможность их применения при больших числах, которые востребованы на практике. Данные тесты рационально применять только на достаточно малых числах.

Вероятностные тесты характеризуются значительно меньшим временем выполнения тестирования числа, поэтому именно такого типа тесты применяются на практике. Но результат, который получается    после выполнения теста, является достоверным лишь с некоторой вероятностью, если он положительный (исследуемое число является простым), или полностью достоверным при отрицательном результате (исследуемое число является составным). 

Разработкой и обзором различных тестов на простоту занимались ранее и занимаются до сих пор следующие учёные: Балабанов~А.~А. \cite{Balabanov}, Василенко~О.~Н. \cite{Vasilenko}, Черемушкин~А.~В. \cite{Cheremyshkin}, Рибенбойм~П. \cite{Ribenboim} и т. д.

На практике различные методы проверки на простоту натурального числа применяются, например, в криптографической системе с открытым ключом RSA \cite{RSA}. Данный криптографический алгоритм основан на том, что факторизация больших натуральных чисел - очень трудная вычислительная операция. Благодаря этому данный алгоритм до сих пор остаётся наиболее эффективным и часто применяемым при шифровании для защиты информации. Но скорость выполнения данного алгоритма шифрования и его криптографическая устойчивость целиком и полностью зависят от выбора пары достаточно больших простых чисел, на основе которых собственно и будет осуществляться процесс шифрования данных. 

Актуальность рассматриваемой темы заключается в потребности разработки более эффективных и точных методов проверки натуральных чисел на простоту с целью снижения потребляемых временных ресурсов и повышения качества криптографических систем при шифровании.

\newpage
\section{Формальные степенные ряды и производящие функции}
\subsection{Производящие функции}

Пусть $a_0, a_1, a_2, \ldots$ – произвольная (бесконечная) последовательность чисел. Производящей функцией (производящим рядом) для этой последовательности будем называть выражение вида
$$
a_0 + a_1 x + a_2 x^2 + \ldots,
$$

или, в сокращенной записи,
$$
\sum\limits_{n = 0}^\infty a_nx^n.
$$

Если все члены последовательности, начиная с некоторого, равны нулю, то производящая функция является производящим многочленом.

Числа, входящие в последовательность, могут иметь различную природу. Производящую функцию, как и обычную функцию, часто обозначают одной буквой, указывая в скобках ее аргумент:
$$
A\left(x\right) = a_0 + a_1x + a_2x^2 + \ldots
$$

Производящая функция представляет последовательность чисел в виде ряда по степеням формальной переменной. Поэтому наряду с термином «производящая функция» можно также пользоваться термином «формальный степенной ряд» \cite{Lando}.

\subsection{Операции над производящими функциями}

Суммой двух производящих функций 
$$
A\left(x\right) = a_0 + a_1 x + a_2 x^2 + \ldots
$$

и
$$
B\left(x\right) = b_0 + b_1 x + b_2 x^2 + \ldots
$$

называется производящая функция
$$
A\left(x\right) + B\left(x\right) = \left(a_0 + b_0\right) + \left(a_1 + b_1\right) x + \left(a_2 + b_2\right) x^2 + \ldots
$$

Произведением двух производящих функций
$$
A\left(x\right) = a_0 + a_1 x + a_2 x^2 + \ldots
$$

и
$$
B\left(x\right) = b_0 + b_1 x + b_2 x^2 + \ldots
$$

называется производящая функция
$$
A\left(x\right) B\left(x\right) = a_0 b_0 + \left(a_0 b_1 + a_1 b_0\right) x + \left(a_0 b_2 + a_1 b_1 + a_2 b_0\right) x^2 + \ldots
$$

Операции сложения и умножения производящих функций, очевидно, коммутативны и ассоциативны.

Пусть
$$
A\left(x\right) = a_0 + a_1 x + a_2 x^2 + \ldots
$$

и
$$
B\left(t\right) = b_0 + b_1 t + b_2 t^2 + \ldots
$$

– две производящие функции, причем $B\left(0\right) = b_0 = 0$.

Подстановкой производящей функции $B\left(t\right)$ в производящую функцию $A\left(x\right)$ называется производящая функция
$$
A\left(B\left(t\right)\right) = a_0 + a_1 b_1 t + \left(a_1 b_2 + a_2 b_1\right) t^2 + \ldots
$$

Операция подстановки функции, отличной от нуля в нуле, не определена. При попытке подставить такую функцию, было бы необходимо суммировать бесконечные числовые ряды. Если обе производящие функции $A$ и $B$ являются многочленами, то определения суммы, произведения и подстановки для них совпадают с обычными определениями этих операций для многочленов.

\begin{theorem}(об обратной функции).
\end{theorem}
Пусть функция
$$
B\left(t\right) = b_0 + b_1 t + b_2 t^2 + b_3 t^3 + \ldots
$$

такова, что $B\left(0\right) = b_0 = 0, a b_1 \not\equiv 0$. Тогда существует такая функция
$$
A\left(x\right) = a_1 x + a_2 x^2 + a_3 x^3 + \ldots, A\left(0\right) = 0,
$$

что
$$
A\left(B\left(t\right)\right) = t  \text{ и } B\left(A\left(x\right)\right) = x.
$$

Функция $A$ единственна, она называется обратной к функции $B$. 

Операция деления не всегда корректно определена. В этом отношении степенные ряды похожи на целые числа: не всегда целое число при делении на другое целое число дает в ответе целое число. Однако, во всяком случае, возможно деление на степенной ряд, значение которого в нуле отлично от нуля.

Предложение. Пусть
$$
A\left(x\right) = a_0 + a_1 x + a_2 x^2 + \ldots
$$
 
– формальный степенной ряд, причем $A\left(0\right) \not\equiv 0$. Тогда существует единственный формальный степенной ряд
$$
B\left(x\right) = b_0 + b_1 x + b_2 x^2 + \ldots
$$

такой, что
$$
A\left(x\right) B\left(x\right) = 1.
$$

\subsection{Элементарные производящие функции}

Всякий раз записывать производящие функции в виде ряда неудобно. Поэтому для некоторых часто встречающихся функций используется сокращенная запись.

\begin{enumerate}

\item  $\left(1 + x\right)^\alpha = 1 + \frac{\alpha }{1!} x + \frac{\alpha\left(\alpha -1\right)}{2!} x^2 + \frac{\alpha\left(\alpha -1\right)\left(\alpha -2\right)}{3!} x^{3} +\ldots$,
где $\alpha $ - произвольное комплексное число.

Коэффициент при $x^n$ в этой производящей функции называется числом сочетаний из $\alpha$ элементов по $n$ и обозначается через
\begin{equation}
{\alpha \choose n}=\frac{\alpha\left(\alpha - 1\right)\ldots\left(\alpha - n + 1\right)}{n!} .    
\end{equation}

\item  $e^{x} = 1 + \frac{1}{1!} x + \frac{1}{2!} x^2 + \frac{1}{3!} x^3 + \ldots$;

\item  $\ln\left(\frac{1}{1 - x}\right) = x + \frac{1}{2} x^2 + \frac{1}{3} x^3 + \ldots$;       

\item  $\sin x = x-\frac{1}{3!} x^3 + \frac{1}{5!} x^5 - \ldots$;       

\item  $\cos x = 1 - \frac{1}{2!} x^2 + \frac{1}{4!} x^4 - \ldots$.

\end{enumerate}

\subsection{Дифференцирование и интегрирование производящих функций}

Пусть
$$
A\left(x\right) = a_0 + a_1 x + a_2 x^2 +\ldots
$$

-- производящая функция. Производной этой функции называется функция
$$
A'\left(x\right) = a_1 + 2a_2 x + 3a_3 x^2 + \ldots + na_n x^{n-1} + \ldots    
$$

Интегралом называется функция
$$
\int A\left(x\right)dx = a_0 x + a_1 \frac{x^2}{2} + a_2 \frac{x^3}{3} + \ldots + a_n \frac{x^{n+1} }{\left(n + 1\right)} + \ldots
$$

Операция дифференцирования обратна операции интегрирования:
$$
\left(\int A\left(s\right)ds\right)' = A\left(s\right).  
$$

Операция же интегрирования производной приводит к функции с нулевым свободным членом, и поэтому результат, отличается от исходной функции.

Интегрирование и дифференцирование позволяют подсчитывать производящие функции для большого числа разнообразных последовательностей \cite{Lando}. 

Например, производящая функция следующей последовательности
$$
f\left(x\right) = \frac{1}{1 \cdot 2} + \frac{1}{2 \cdot 3} x + \ldots + \frac{1}{\left(n + 1\right)\left(n + 2\right)} x^n + \ldots,  
$$

можно найти умножая функцию $f$ на $x^2$ и дифференцируя
$$
\left(x^2 f\left(x\right)\right)' = x + \frac{1}{2} x^2 + \frac{1}{3} x^3 + \ldots = \ln\left(1 - x\right)^{-1}
$$

откуда 
$$
f\left(x\right) = x^{-2} \int \ln \left(1 - x\right)^{-1} = x^{-2} \left(\left(x - 1\right)\ln \left(1 - x\right)^{-1} + x\right).  
$$

\subsection{Производящие функции для известных последовательностей}

В этом подразделе рассматриваются известные последовательности и их производящие функции, представленные в таблице ~\thesubsection.1. Также более подробно рассматриваются геометрическая последовательность и последовательность Фибоначчи.

\newpage
\noindent Таблица \thesubsection.1 -- Простые последовательности и их производящие функции\\
\begin{tabular}{|p{2.5in}|p{2.0in}|p{2.0in}|} \hline 
Последовательность & Формальный степенной ряд & Производящая функция \\ \hline 
\{1,0,0,0,0,0,\dots \} & $\sum _{n\ge 0}[n=0]x^{n}  $ & 1 \\ \hline 
\{0,\dots ,0,1,0,0,\dots \} & $\sum\limits _{n\ge 0}[n=m]x^{n}  $ & $z^{m} $ \\ \hline 
\{1,1,1,1,1,1,\dots \} & $\sum\limits _{n\ge 0}x^{n}  $ & $\frac{1}{1-x} $ \\ \hline 
\{1,-1,1,-1,1,-1,\dots \} & $\sum\limits _{n\ge 0}(-1)^{n} x^{n}  $ & $\frac{1}{1+x} $ \\ \hline 
\{1,0,1,0,1,0,\dots \} & $\sum\limits _{n\ge 0}[2\backslash n]x^{n}  $ & $\frac{1}{1-x^{2} } $ \\ \hline 
\{1,0,\dots ,0,1,0,\dots ,0,1,0,\dots \} & $\sum\limits _{n\ge 0}[m\backslash n]x^{n}  $ & $\frac{1}{1-x^{m} } $ \\ \hline 
\{1,2,3,4,5,6,\dots \} & $\sum\limits _{n\ge 0}(n+1)x^{n}  $ & $\frac{1}{(1-x)^{2} } $ \\ \hline 
\{1,2,4,8,16,32,\dots \} & $\sum\limits _{n\ge 0}2^{n} x^{n}  $ & $\frac{1}{1-2x} $ \\ \hline 
\{1,4,6,4,1,0,0,\dots \} & $\sum\limits _{n\ge 0}\left(\begin{array}{c} {4} \\ {n} \end{array}\right)x^{n}  $ & $(1+x)^{4} $ \\ \hline 
$\{ 1,c,\left(\begin{array}{c} {c} \\ {2} \end{array}\right),\left(\begin{array}{c} {c} \\ {3} \end{array}\right),...\} $ & $\sum\limits _{n\ge 0}\left(\begin{array}{c} {c} \\ {n} \end{array}\right)x^{n}  $ & $(1+x)^{c} $ \\ \hline 
\{1,c,c${}^{2}$,c${}^{3}$,\dots \} & $\sum\limits _{n\ge 0}c^{n} x^{n}  $ & $\frac{1}{1-cx} $ \\ \hline 
\end{tabular}

\subsection{Композиция или сложная функция}

Функция $A(x)=R\left(F\left(x\right)\right)$ называется суперпозицией (композицией) функций $R(x)$ и $F(x)$, или сложной функцией.

Для композиции функций известна следующая теорема \cite{KruchininVV}.

\begin{theorem}
\end{theorem}
Пусть даны функции $f(n)$ и $r(n)$ и их производящие функции $F(x)=\sum\limits _{n\ge 1}f(n)x^{n} ,R(x)=\sum\limits _{n\ge 0}r(n)x^{n}   $. Тогда для функции
$$
a(n)=\sum\limits _{k=1}^{n}\sum\limits _{\pi _{k} \in C_{n} }f(\lambda _{1} )f(\lambda _{2} ) ...f(\lambda _{k} )r(k) ,  
$$

производящая функция $A(x)$ является композицией производящих функций $R(x)$ и $F(x)$: $A(x)=R(F(x))$.

\begin{proof}

Последовательно запишем $F(x),F^{2} (x),...,F^{k} (x)$. Для $F^{2} (x)$ по правилу произведения производящих функций и с учетом, что $f(0)=0$, получим
\[F^{2} (x)=\sum\limits _{n\ge 2}\sum\limits _{\begin{array}{l} {\lambda _{1} ,\lambda _{2} \ge 1} \\ {\lambda _{1} +\lambda _{2} =n} \end{array}}f(\lambda _{1} )f(\lambda _{2} )x^{n}   .\] 

По индукции для $F^{k} (x)$ получим
$$
F^{k} (x)=\sum\limits _{n\ge k}\sum\limits _{\begin{array}{l} {\lambda _{1} ,\lambda _{2} ,...,\lambda _{k} \ge 1} \\ {\lambda _{1} +\lambda _{2} +...+\lambda _{k} =n} \end{array}}f(\lambda _{1} )f(\lambda _{2} )...f(\lambda _{k} )x^{n}   .  
$$

По определению имеем, что
\[A(x)=R(F(x))=r(0)+r(1)F(x)+r(2)F^{2} (x)+...+r(n)F^{n} (x)+...\] 

Подставим полученные выражения для $F^{k} (x)$ и произведем группировку членов во всех рядах $F^{k} (x)$ с одинаковыми степенями $x^{n} $, получим
\[\begin{array}{c} {R(F(x))=r(0)+f(1)r(1)x+[f(2)r(1)+f(1)f(1)r(2)]x^{2} +} \\  {+[f(3)r(1)+[f(2)f(1)+f(1)f(2)]r(2)+f(1)f(1)f(1)r(3)]x^{3} } \\ 
{\ldots} \\
{\sum\limits _{k=1}^{n}[\sum\limits _{\pi _{k} =\{ \lambda _{1} +\lambda _{2} +...+\lambda _{k} =n\} \in C_{n} }f(\lambda _{1} )f(\lambda _{2} )...f(\lambda _{k} )]r(k)x^{n}}\\
{\ldots}\\
 \end{array}\] 

Следовательно, функция $a(n)$ имеет вид
\[
a(n)=\sum\limits _{k=1}^{n}[\sum\limits _{\pi _{k} =\{ \lambda _{1} +\lambda _{2} +...+\lambda _{k} =n\} \in C_{n} }f(\lambda _{1} )f(\lambda _{2} )...f(\lambda _{k} )]r(k)  .  
\] 
\end{proof}

Рассмотрим пример. Предположим, что $f(0)=0,f(n)=1$ для всех $n$>$0$. Эта функция задается производящей функцией $F(x)=\frac{x}{1-x}$. Тогда выражение
\[\sum\limits _{\pi _{k} \in C_{n} }f(\lambda _{1} )f(\lambda _{2} )...f(\lambda _{k} ) \] 
будет подсчитывать число композиций натурального числа $n$, имеющих ровно $k$ частей, которое равно $\left(\begin{array}{c} {n-1} \\ {k-1} \end{array}\right)$. Таким образом
$$
\sum\limits _{\pi _{k} \in C_{n} }f(\lambda _{1} )f(\lambda _{2} )...f(\lambda _{k} ) =\left(\begin{array}{c} {n-1} \\ {k-1} \end{array}\right) 
$$

Отсюда можно сделать вывод, что для любой производящей функции $R(x)=\sum\limits _{n\ge 0}r(n)x^{n}  $ и $A(x)=R(\frac{x}{1-x} )$ справедлива следующая формула:
$$
a(n)=\sum\limits _{k=1}^{n}\left(\begin{array}{c} {n-1} \\ {k-1} \end{array}\right)r(k) .     
$$

Композитой производящей функции $F(x)=\sum\limits _{n>0}f(n)x^{n}  $ называется функция \cite{KruchininVV}
$$
F^{\Delta } (n,k)=\sum\limits _{\pi _{k\in C_{n} } }f(\lambda _{1} )f(\lambda _{2} )...f(\lambda _{k} ) .  
$$

Вычисление $F^{\Delta } (n,k)$ имеет первостепенное значение для получения композиции производящих функций, поскольку для вычисления композиции $A(x)=R(F(x))$ справедлива формула
$$
a(n)=\sum\limits _{k=1}^{n}F^{\Delta } (n,k)r(k) .  
$$

Композита не зависит от функции $R(x)$ и является характеристикой производящей функции $F(x)$. В основе получения композиты лежит вычисление композиции $\pi _{k} $ множества композиций числа $n$. Производящая функция композиты равна
$$
[F(x)]^{k} =\sum\limits _{n\ge k}F^{\Delta } (n,k)x^{n}  .  
$$

Для $F(x)$ выполняется условие $F(0)=0$, отсюда нумерация для композиты начинается со значения $k=1,n=1$. При $k=1$ значение композиты $F^{\Delta } (n,k)=f(n)$. При  $k>n$ композита $F^{\Delta } (n,k)$ равна нулю. Это утверждение основано на том, что не существует композиции числа $n$, в которой число частей больше $n$.

Рассмотренный выше пример показывает, что треугольник Паскаля является композитой для производящей функции $\frac{x}{1-x} $ и для получения композиции $A(x)=R(\frac{x}{1-x} )$ необходимо использовать $F^{\Delta } (n,k)=\left(\begin{array}{c} {n-1} \\ {k-1} \end{array}\right)$ \cite{GrKnPa,KruchininVV}.

\newpage
\section{Метод построения критериев простоты}

Для построения критериев рассмотрим отдельный класс производящих функций. Степенной ряд вида
$$
\sum\limits _{n=1}^{\infty }\frac{a(n) }{n} x^{n}  ,                                 
$$

где $a(n)$ -- целочисленная последовательность, будем называть логарифмической производящей функцией.

Логарифмическая производящая функция отличается от обычной, тем что в качестве коэффициентов степенного ряда берутся элементы $a(n) $, деленные на порядковый номер, то есть $\frac{a(n)}{n} $. Чаще всего элементы $a(n) $ равны 1, и в качестве коэффициентов используются числа вида $\frac{1}{n}$.  Еще одно отличие заключается в том, что отсутствует свободный член \cite{KruchininDV_SPofLGF}.

Операции сложения, умножения, дифференцирования и интегрирования, определенные для всех производящих функций, справедливы и для логарифмических.

Пусть $F(x)=\sum\limits _{n = 1}^{\infty}f(n)x^n$ производящая функция с целыми коэффициентами, $R(x)=\sum\limits _{n = 1}^{\infty}r(n)x^n$ логарифмическая производящая функция, Тогда для композиции $A(x)=R(F(x))$, такой что $A(x)=\sum\limits _{n = 1}^{\infty}a(n)x^n$, выполняются следующие свойства.

Свойство 1. Для любых значений $n \in \mathbb{N} $ значения выражения $na(n)$ являются целыми числами.
 
Свойство 2. Для любого простого $n$ значения выражения $\frac{na(n) - f(1)^n}{n}$ являются целыми числами.

Доказательство этих свойств приведено Кручининым Д.В. \cite{KruchininDV_SPofLGF}. 
Использование данных свойств композиции производящих функций делает возможным построение алгоритмов тестов на простоту натурального числа.

В зависимости от параметров композиции, а именно от самой логарифмической функции, от композиты подставляемой производящей функции, выражение
$$
a(n)=\sum\limits _{k=1}^{n-1}F^{\Delta } (n,k)r(k) =\sum\limits _{k=1}^{n-1}F^{\Delta } (n,k)\frac{a(k)}{k}   
$$

имеет различные числовые и вероятностные характеристики, а также вычислительную сложность.

Рассмотрим применение данного метода на следующих примерах:

\begin{itemize}
\item Тест на основе композиции $\ln(\frac{1}{1-F(x)} )$, где $F(x)=\frac{x}{1-x} =x+x^{2} +...+x^{n}\ldots$
\end{itemize}

Получаем, что коэффициенты композиции будут равны
$$
a(n)=\sum\limits _{k=1}^{n}F^{\Delta } (n,k)r(k) =\sum\limits _{k=1}^{n}\frac{1}{k}  \left(\begin{array}{c} {n-1} \\ {k-1} \end{array}\right)=\frac{2^{n} -1}{n} . 
$$

Выражение 
\[\sum\limits _{k=1}^{n-1}\frac{1}{k}  \left(\begin{array}{c} {n-1} \\ {k-1} \end{array}\right)=\frac{2^{n} -2}{n} \] 
является целым числом, для простых $n$.

После преобразования получим 
\[2^{n-1} \equiv 1(mod\ n),\] 
что является тестом на основе Малой теоремы Ферма по основанию 2.

Преимуществом данного теста является его быстрота и маленькая вычислительная сложность, поскольку ЭВМ основывается на двоичной системе счисления, и ${2}^{n-1}$ является последовательностью единиц до разряда $n$. Поэтому основная трудоемкость заключается в делении на $n$.

Последовательность ошибок теста, полученных псевдопростых $n$, следующая  для $n$ не превосходящих 30~000:

$
\begin{array}{c} {{\rm \{ 341,\; 561,\; 645,\; 1105,\; 1387,\; 1729,\; 1905,\; 2047,\; 2465,\; 2701,\; 2821,\; }} \\ {{\rm 3277,\; 4033,\; 4369,\; 4371,\; 4681,\; 5461,\; 6601,\; 7957,\; 8321,\; 8481,\; 8911,}} \\ {{\rm \; 10261,\; 10585,\; 11305,\; 12801,\; 13741,\; 13747,\; 13981,\; 14491,\; 15709,\; }} \\ {{\rm 15841,\; 16705,\; 18705,\; 18721,\; 19951,\; 23001,\; 23377,\; 25761,\; 29341\} .}} \end{array} 
$

Также эта последовательность \href{http://oeis.org/A001567}{A001567} называется числа Сарруса (Sarrus numbers) \cite{oeis}.

Этот тест эффективно использовать на начальной стадии проверки простоты для больших чисел.  В зависимости от требуемой погрешности проверки чисел на простоту, целесообразно использовать как в криптосистемах, так и для решения прикладных задач.

\begin{itemize}
\item Тест на основе композиции $\ln (\frac{1}{1-F(x)} )$, где $F(x)=\alpha x+\beta x^{2}$
\end{itemize}

Для нахождения композиции, необходимо получить выражение для композиты данной производящей функции. В \cite{GrKnPa} показано, что композита производящей функции $F(x)=\alpha x+\beta x^{2} $  имеет следующий вид:
$$
F^{\Delta } (n,k,\alpha ,\beta )=\left(\begin{array}{c} {k} \\ {n-k} \end{array}\right)\alpha ^{2k-n} \beta ^{n-k} . 
$$

Таким образом, для нахождения композиции $A(x)=\ln (\frac{1}{1-\alpha x-\beta x^{2} } )$ воспользуемся выражением
$$
a(n)=\sum\limits _{k=1}^{n}\left(\begin{array}{c} {k} \\ {n-k} \end{array}\right)\alpha ^{2k-n} \beta ^{n-k} \frac{1}{k} .  
$$

Рассмотрим частный вариант этой последовательности, где $\alpha =1,\beta =1$.  Получим следующий тест основанный на числах Люка: значение выражения \[\frac{L_{n} -1}{n}\] является целым или
\[L_{n} =1(mod\ n)\]
для простых чисел, где $L_{n} $ - числа Люка.
\[L_{n} =\left(\frac{1+\sqrt{5} }{2} \right)^{n} +\left(\frac{1-\sqrt{5} }{2} \right)^{n} \] 
или
\[L_{n} =Fib_{n} +2Fib_{n-1} =Fib_{n+1} +Fib_{n-1} .\] 
\[L_{n} =[1,3,4,7,11,18,29,47,76,123,199,322,521,843,1364,2207,3571].\] 

Последовательность \href{http://oeis.org/А005845}{А005845} ошибок теста, полученных псевдопростых $n$ следующая  для $n$ не превосходящих 150~000 \cite{oeis}.
\[ 
\begin{array}{c} {{\rm \{ 705,\; 2465,\; 2737,\; 3745,\; 4181,\; 5777,\; 6721,\; 10877,\; 13201,\; 15251,}} \\ {{\rm \; 24465,29281,\; 34561,\; 35785,\; 51841,\; 54705,\; 64079,\; 64681,\; 67861,}} \\ {{\rm \; 68251,\; 75077,\; 80189,\; 90061,\; 96049,\; 97921,\; 100065,\; 100127,}} \\ {{\rm \; 105281,\; 113573,\; 118441,\; 146611\} .}} \end{array} 
\]

\newpage
\section{Система компьютерной алгебры Maxima}
\subsection{Введение}
Maxima -- программа для выполнения математических вычислений, символьных преобразований и построения графиков.
Maxima является потомком DOE Macsyma, которая начала своё существование в конце 1960 года в Массачусетском технологическом институте. Macsyma первая создала систему компьютерной алгебры, она проложила путь для таких программ как Maple и Mathematica. Главный вариант Maxima разрабатывался Вильямом Шелтером с 1982 по 2001 год. Благодаря его умению Maxima сумела выжить и сохранить свой оригинальный код в рабочем состоянии. На сегодняшний день пакет достаточно активно развивается, и во многих отношениях не уступает другим развитым системам компьютерной математики.

\subsection{Особенности Maxima}
К основным достоинствам программы Maxima можно отнести следующее:
\begin{itemize}
\item возможность свободного использования (Maxima относится к классу свободных программ и распространяется на основе лицензии GNU);
\item возможность функционирования под управлением различных операционных систем (в
частности Linux и Windows); 
\item небольшой размер программы (дистрибутив занимает порядка 33 мегабайт, в установленном виде со всеми расширениями потребуется около 115 мегабайт);
\item широкий класс решаемых задач; 
\item возможность работы как в консольной версии программы, так и с использованием одного из графических интерфейсов (xMaxima, wxMaxima); 
\item расширение wxMaxima (входящее в комплект поставки) предоставляет пользователю удобный и понятный интерфейс, избавляет от необходимости изучать особенности ввода команд для решения типовых задач; 
\item интерфейс программы на русском языке; 
\item наличие справки и инструкций по работе с программой (русскоязычной версии справки нет, но в сети Интернет присутствует большое количество книг с примерами использования Maxima \cite{Maxima_Stahin, Maxima_Chichkarev}).
\end{itemize}


\newpage
\section{Критерии простоты на основе производящей функции натурального логарифма $R(x)=ln(1+x)$}
В качестве внешней функции $R(x)$ композиции $G(x)=R(F(x))$ рассмотрим производящую функцию следующего вида:
$$
R(x)=ln(1+x)=x-\frac{x^2}{2}+\frac{x^3}{3}-\ldots=\sum^{\infty}_{n=1}{\frac{(-1)^{n-1}}{n}x^n}
$$

Тогда функция коэффициентов $r(n)$ производящей функции $R(x)$:
$$
r(n)=\frac{(-1)^{n-1}}{n}
$$

\subsection{Производящая функция $F(x)=ax+bx^2$}
Рассмотрим в качестве внутренней функции композиции $G(x)=R(F(x))$ следующую производящую функцию:
$$
F(x)=ax+bx^2
$$

Композита для данной производящей функции \cite{KruchininVV}:
$$
F^\Delta(n,k,a,b)=a^{2k-n}b^{n-k}{k \choose n-k}
$$

Функция коэффициентов $g(n)$ композиции $G(x)=R(F(x))$:
$$
g(n)=\sum^{n}_{k=1}{F^\Delta(n,k,a,b)r(k)}=\sum^{n}_{k=1}{a^{2k-n}b^{n-k}{k \choose n-k}\frac{(-1)^{k-1}}{k}}
$$

Рассмотрим частные случаи параметров $a$ и $b$:
\begin{itemize}
\item[1)] При $a=1, b=1$

\begin{math}
ng(n)=[1,1,-2,1,1,-2,1,1,-2,1,1,-2,1,1,-2,1,1,-2,1,1,\ldots]
\end{math}

Это последовательность целых чисел \href{http://oeis.org/A061347}{A061347} \cite{oeis}, откуда получаем:
$$
ng(n)=n(mod\ 3)-(n-1)(mod\ 3)
$$

Критерий простоты числа следующий: значение выражения
$$
\frac{n(mod\ 3)-(n-1)(mod\ 3)-(-1)^{n+1}}{n}
$$

является целым для простых чисел $n$.

Данный тест не является эффективным, так как определяет в качестве составных чисел только числа, делящиеся на $2$ и на $3$.

\item[2)] При $a=2, b=1$

\begin{math}
ng(n)=[2,-2,2,-2,2,-2,2,-2,2,-2,2,-2,2,-2,2,-2,2,-2,2,-2,\ldots]
\end{math}

Это последовательность из знакочередующегося числа 2, откуда получаем:
$$
ng(n)=(-1)^{n+1}2
$$

Критерий простоты числа следующий: значение выражения
$$
(-1)^{n+1}\frac{2-2^n}{n}
$$

является целым для простых чисел $n$.

Данный тест не является эффективным, так как выдает положительный результат теста для большого числа составных чисел (из проверки чисел от $2$ до $100000$ ошибочно определены как простые $78$ составных чисел).

\item[3)] При $a=3, b=1$

\begin{math}
ng(n)=[3,-7,18,-47,123,-322,843,-2207,5778,-15127,39603,\\
-103682,271443,-710647,1860498,-4870847,12752043,-33385282,\\
87403803,-228826127,\ldots]
\end{math}

Это знакочередующаяся последовательность целых чисел \href{http://oeis.org/A005248}{A005248} \cite{oeis} -- числа Люка вида $L(2n)$, откуда получаем:
$$
ng(n)=(-1)^{n+1}\frac{(3-\sqrt{5})^{n}+(3+\sqrt{5})^{n}}{2^{n}}
$$

Критерий простоты числа следующий: значение выражения
$$
(-1)^{n+1}\frac{(3-\sqrt{5})^{n}+(3+\sqrt{5})^{n}-6^{n}}{n 2^{n}}
$$

является целым для простых чисел $n$.

Данный тест трудоемок для вычисления. А также не является эффективным, так как выдает положительный результат теста для большого числа составных чисел (из проверки чисел от $2$ до $50000$ ошибочно определены как простые $23$ составных чисел, но если предварительно убрать числа, кратные $3$, то ошибочно определятся только $6$ чисел).

\item[4)] При $a=1, b=2$

\begin{math}
ng(n)=[1,3,-5,-1,11,-9,-13,31,-5,-57,67,47,-181,87,275,\\
-449,-101,999,-797,-1201,\ldots]
\end{math}

Это знакочередующаяся последовательность целых чисел \href{http://oeis.org/A002249}{A002249} \cite{oeis}, откуда получаем:
$$
ng(n)=(-1)^{n+1}\frac{(1+\sqrt{7} i)^{n}+(1-\sqrt{7} i)^{n}}{2^{n}}
$$

Критерий простоты числа следующий: значение выражения
$$
(-1)^{n+1}\frac{(1+\sqrt{7} i)^{n}+(1-\sqrt{7} i)^{n}-2^{n}}{n 2^{n}}
$$

является целым для простых чисел $n$.

Данный тест трудоемок для вычисления. А также не является эффективным, так как выдает положительный результат теста для чисел, кратных 2. Если предварительно убрать четные числа, то результат значительно улучшается (из проверки чисел от $2$ до $50000$ ошибочно определены как простые $2$ составных числа $3745$ и $41419$).

\ \\
\item[5)] При $a=2, b=2$

\begin{math}
ng(n)=[2,0,-4,8,-8,0,16,-32,32,0,-64,128,-128,0,256,-512,\\
512,0,-1024,2048,\ldots]
\end{math}

Это последовательность целых чисел \href{http://oeis.org/A108520}{A108520} \cite{oeis}, откуда получаем:
$$
ng(n)=-(-1-i)^{n}-(-1+i)^{n}
$$

Критерий простоты числа следующий: значение выражения
$$
\frac{-(-1-i)^{n}-(-1+i)^{n}-{2}^{n}(-1)^{n+1}}{n}
$$

является целым для простых чисел $n$.

Данный тест трудоемок для вычисления. А также не является эффективным, так как выдает положительный результат теста для чисел, кратных 2. Если предварительно убрать четные числа, то результат улучшается (из проверки чисел от $2$ до $50000$ ошибочно определены как простые $26$ составных чисел).

\item[6)] При $a=3, b=2$

\begin{math}
ng(n)=[3,-5,9,-17,33,-65,129,-257,513,-1025,2049,-4097,8193,\\
-16385,32769,-65537,131073,-262145,524289,-1048577512,0,\\
-1024,2048,\ldots]
\end{math}

Это последовательность целых чисел \href{http://oeis.org/A000051}{A000051} \cite{oeis}, откуда получаем:
$$
ng(n)=(-1)^{n+1}(2^{n}+1)
$$

Критерий простоты числа следующий: значение выражения
$$
\frac{(-1)^{n+1}(2^{n}+1-3^{n})}{n}
$$

является целым для простых чисел $n$.

Данный тест не является эффективным, так как выдает положительный результат теста для большого числа составных чисел (из проверки чисел от $2$ до $100000$ ошибочно определены как простые $93$ составных числа).

\item[7)] При $a=1, b=3$

\begin{math}
ng(n)=[1,5,-8,-7,31,-10,-83,113,136,-475,67,1358,-1559,-2515,\\
7192,353,-21929,20870,44917,-107527,\ldots]
\end{math}

Это знакочередующаяся последовательность целых чисел \href{http://oeis.org/A131040}{A131040} \cite{oeis}, откуда получаем:
$$
ng(n)=(-1)^{n+1}\frac{(1+\sqrt{11} i)^{n}+(1-\sqrt{11} i)^{n}}{2^{n}}
$$

Критерий простоты числа следующий: значение выражения
$$
(-1)^{n+1}\frac{(1+\sqrt{11} i)^{n}+(1-\sqrt{11} i)^{n}-2^{n}}{n 2^{n}}
$$

является целым для простых чисел $n$.

Данный тест трудоемок для вычисления. А также не является эффективным, так как выдает положительный результат теста для чисел, кратных 3. Если предварительно убрать числа, кратные 3, то результат значительно улучшается (из проверки чисел от $2$ до $50000$ ошибочно определены как простые $2$ составных числа $731$ и $1541$).

\item[8)] При $a=2, b=3$

\begin{math}
ng(n)=[2,2,-10,14,2,-46,86,-34,-190,482,-394,-658,2498,-3022,\\
-1450,11966,-19582,3266,52214,-114226,\ldots]
\end{math}

Это знакочередующаяся последовательность целых чисел \href{http://oeis.org/A057682}{A057682} \cite{oeis}, откуда получаем:
$$
ng(n)=(-1)^{n+1}\frac{(3+\sqrt{3} i)^{n}+(3-\sqrt{3} i)^{n}}{2^{n}}
$$

Критерий простоты числа следующий: значение выражения
$$
(-1)^{n+1}\frac{(2+\sqrt{8} i)^{n}+(2-\sqrt{8} i)^{n}-4^{n}}{n 2^{n}}
$$

является целым для простых чисел $n$.

Данный тест трудоемок для вычисления. А также не является эффективным, так как выдает положительный результат теста для чисел, кратных 3, и для некоторых четных чисел. Если предварительно убрать числа, кратные 2 и 3, то результат улучшается (из проверки чисел от $2$ до $50000$ ошибочно определены как простые $3$ составных числа $961$, $1105$ и $3481$).

\item[9)] При $a=3, b=3$

\begin{math}
ng(n)=[3,-3,0,9,-27,54,-81,81,0,-243,729,-1458,2187,-2187,0,\\
6561,-19683,39366,-59049,59049,\ldots]
\end{math}

Это знакочередующаяся последовательность целых чисел \href{http://oeis.org/A057682}{A057682} \cite{oeis}, откуда получаем:
$$
ng(n)=(-1)^{n+1}\frac{(3+\sqrt{3} i)^{n}+(3-\sqrt{3} i)^{n}}{2^{n}}
$$

Критерий простоты числа следующий: значение выражения
$$
(-1)^{n+1}\frac{(3+\sqrt{3} i)^{n}+(3-\sqrt{3} i)^{n}-6^{n}}{n 2^{n}}
$$

является целым для простых чисел $n$.

Данный тест трудоемок для вычисления. А также не является эффективным, так как выдает положительный результат теста для чисел, кратных 3. Если предварительно убрать числа, кратные 3, то результат улучшается (из проверки чисел от $2$ до $50000$ ошибочно определены как простые $26$ составных чисел).

\item[10)] При произвольных значениях $a, b$ можно привести критерий простоты числа в общем виде: при любых целых $a, b$ значение выражения
$$
(-1)^{n+1}\frac{(a+\sqrt{4b-a^2} i)^{n}+(a-\sqrt{4b-a^2} i)^{n}-(2a)^{n}}{n 2^{n}}
$$

является целым для простых чисел $n$.

\end{itemize}

\subsection{Производящая функция $F(x)=\frac{bx}{1 - ax}$}
Рассмотрим в качестве внутренней функции композиции $G(x)=R(F(x))$ следующую производящую функцию:
$$
F(x)=\frac{bx}{1 - ax}
$$

Композита для данной производящей функции \cite{KruchininVV}:
$$
F^\Delta(n,k,a,b)=a^{n - k}b^{k}{n - 1 \choose k - 1}
$$

Функция коэффициентов $g(n)$ композиции $G(x)=R(F(x))$:
$$
g(n)=\sum\limits _{k=1}^{n}{F^\Delta(n,k,a,b)r(k)}=\sum\limits _{k=1}^{n}{a^{n - k}b^{k}{n - 1 \choose k - 1}\frac{(-1)^{k-1}}{k}}
$$

Рассмотрим частные случаи параметров $a$ и $b$:
\begin{itemize}
\item[1)] При $a=1, b=1$

\begin{math}
ng(n)=[1,1,1,1,1,1,1,1,1,1,1,1,1,1,1,1,1,1,1,1,\ldots]
\end{math}

Это последовательность из повторяющихся единиц, откуда получаем:
$$
ng(n)= 1
$$

Критерий простоты числа следующий: значение выражения
$$
\frac{1-(-1)^{n+1}}{n}
$$

является целым для простых чисел $n$.

Данный тест не является эффективным, так как определяет в качестве составных чисел только четные числа.

\item[2)] При $a=2, b=1$

\begin{math}
ng(n)=[1,3,7,15,31,63,127,255,511,1023,2047,4095,8191,16383,\\
32767,65535,131071,262143,524287,1048575,\ldots]
\end{math}

Это последовательность целых чисел \href{http://oeis.org/A000225}{A000225} \cite{oeis}, откуда получаем:
$$
ng(n)=2^n - 1
$$

Критерий простоты числа следующий: значение выражения
$$
\frac{2^n - 1 - (-1)^{n + 1}}{n}
$$

является целым для простых чисел $n$.

Данный тест не является эффективным, так как выдает положительный результат теста для большого числа составных чисел (из проверки чисел от $2$ до $50000$ ошибочно определены как простые $69$ составных чисел).

\item[3)] При $a = 3, b = 1$

\begin{math}
ng(n)=[1,5,19,65,211,665,2059,6305,19171,58025,175099,527345,\\
1586131,4766585, 14316139,42981185,129009091,387158345,1161737179,\\
3485735825,\ldots]
\end{math}

Это последовательность целых чисел \href{http://oeis.org/A001047}{A001047} \cite{oeis}, откуда получаем:
$$
ng(n)=3^n - 2^n
$$

Критерий простоты числа следующий: значение выражения
$$
\frac{3^n - 2^n - (-1)^{n + 1}}{n}
$$

является целым для простых чисел $n$.

Данный тест не является эффективным, так как выдает положительный результат теста для большого числа составных чисел (из проверки чисел от $2$ до $50000$ ошибочно определено как простое $71$ составнoe числo). Стоит отметить, что эффективность данного теста повышается, если к нему добавить проверку делимости чисел на $5$, $7$ и $11$, после которой на том же интервале ошибочно определяется всего $8$ чисел. 

\item[4)] При $a = 1, b = 2$

\begin{math}
ng(n)=[2,0,2,0,2,0,2,0,2,0,2,0,2,0,2,0,2,0,2,0,\ldots]
\end{math}

Это последовательность чередующихся двоек и нулей, откуда получаем:
$$
ng(n)=1 - (-1)^{n + 1}
$$

Критерий простоты числа следующий: значение выражения
$$
\frac{1 - (-1)^{n + 1} - (-1)^{n + 1}}{n}
$$

является целым для простых чисел $n$.

Данный тест демонстрирует набор тех же ошибок, что и тест 2, и также не является эффективным.

\item[5)] При $a=2, b=2$

\begin{math}
ng(n)=[2,4,8,16,32,64,128,256,512,1024,2048,4096,8192,16384,\\
32768,65536,131072,262144,524288,1048576,\ldots]
\end{math}

Это последовательность целых чисел \href{http://oeis.org/A000079}{A000079} \cite{oeis}, откуда получаем:
$$
ng(n)=2^n
$$

Критерий простоты числа следующий: значение выражения
$$
\frac{2^n - 2^n(-1)^{n + 1}}{n}
$$

является целым для простых чисел $n$.

Данный тест не является эффективным, так как выдает положительный результат теста для большого числа составных чисел (из проверки чисел от $2$ до $1000$ ошибочно определены как простые $340$ составных чисел).

\item[6)] При $a=3, b=2$

\begin{math}
ng(n)=[2,8,26,80,242,728,2186,6560,19682,59048,177146,531440,\\
1594322,4782968,14348906,43046720,129140162,387420488,1162261466,\\
3486784400,\ldots]
\end{math}

Это последовательность целых чисел \href{http://oeis.org/A024023}{A024023} \cite{oeis}, откуда получаем:
$$
ng(n)=3^n - 1
$$

Критерий простоты числа следующий: значение выражения
$$
\frac{3^n - 1 - 2^n(-1)^{n + 1}}{n}
$$

является целым для простых чисел $n$.

Данный тест не является эффективным, так как выдает положительный результат теста для большого числа составных чисел (из проверки чисел от $2$ до $100000$ ошибочно определены как простые $199$ составных чисел). Стоит отметить, что эффективность данного теста повышается, если к нему добавить проверку делимости чисел на $2$, $3$, $5$ и $7$, после которой на том же интервале ошибочно определяется всего $12$ чисел.

\item[7)] При $a=2, b=3$

\begin{math}
3,3,9,15,33,63,129,255,513,1023,2049,4095,8193,16383,32769,\\
65535,131073,262143,524289,1048575,\ldots]
\end{math}

Это последовательность целых чисел \href{http://oeis.org/A062510}{A062510} \cite{oeis}, откуда получаем:
$$
ng(n)=2^n - (-1)^n
$$

Критерий простоты числа следующий: значение выражения
$$
\frac{2^n - (-1)^n - 3^n(-1)^{n + 1}}{n}
$$

является целым для простых чисел $n$.

Результаты данного теста совпадают с результатами теста 6.

\item[8)] При $a=3, b=3$

\begin{math}
ng(n)=[3,9,27,81,243,729,2187,6561,19683,59049,177147,531441,\\
1594323,4782969, 14348907,43046721,129140163,387420489,1162261467,\\
3486784401,\ldots]
\end{math}

Это последовательность целых чисел \href{http://oeis.org/A000244}{A000244} \cite{oeis}, откуда получаем:
$$
ng(n)=3^n
$$

Критерий простоты числа следующий: значение выражения
$$
\frac{3^n - 3^n(-1)^{n + 1}}{n}
$$

является целым для простых чисел $n$.

Данный тест не является эффективным, так как выдает положительный результат теста для большого числа составных чисел (из проверки чисел от $2$ до $1000$ ошибочно определены как простые $337$ составных чисел).

\item[9)] При $a=5, b=2$

\begin{math}
ng(n)=[2,16,98,544,2882,14896,75938,384064,1933442,9706576,\\
48650978,243609184,1219108802,6098732656,30503229218,152544843904,\\
762810312962,3814309845136,19072324066658,95363944856224,\ldots]
\end{math}

Это последовательность целых чисел \href{http://oeis.org/A005058}{A005058} \cite{oeis}, откуда получаем:
$$
ng(n)=5^n - 3^n
$$

Критерий простоты числа следующий: значение выражения
$$
\frac{5^n - 3^n - 2^n(-1)^{n + 1}}{n}
$$

является целым для простых чисел $n$.

Данный тест не является эффективным, так как выдает положительный результат теста для большого числа составных чисел (из проверки чисел от $2$ до $10000$ ошибочно определены как простые $134$ составных числа). Стоит отметить, что эффективность данного теста повышается, если к нему добавить проверку делимости чисел на $2$, $3$, $5$ и $7$, после которой на том же интервале ошибочно определяется всего $5$ чисел.

\item[10)] При $a=5, b=4$

\begin{math}
ng(n)=[4,24,124,624,3124,15624,78124,390624,1953124,9765624,\\
48828124,244140624,1220703124,6103515624,30517578124,152587890624,\\
762939453124,3814697265624,19073486328124,95367431640624,\ldots]
\end{math}

Это последовательность целых чисел \href{http://oeis.org/A005058}{A005058} \cite{oeis}, откуда получаем:
$$
ng(n)=5^n - 1
$$

Критерий простоты числа следующий: значение выражения
$$
\frac{5^n - 1 - 4^n(-1)^{n + 1}}{n}
$$

является целым для простых чисел $n$.

Данный тест не является эффективным, так как выдает положительный результат теста для большого числа составных чисел (из проверки чисел от $2$ до $10000$ ошибочно определены как простые $89$ составных чисел). Стоит отметить, что эффективность данного теста повышается, если к нему добавить проверку делимости чисел на $2$, $3$, $5$ и $7$, после которой на том же интервале ошибочно определяется всего $3$ числа.


\end{itemize}

\subsection{Производящая функция $F(x)=\frac{x}{(1-x)^m}$}
Рассмотрим в качестве внутренней функции композиции $G(x)=R(F(x))$ следующую производящую функцию:
$$
F(x)=\frac{x}{(1-x)^m}
$$

Композита для данной производящей функции \cite{KruchininVV}:
$$
F^\Delta(n,k,m)={n+k(m-1)-1 \choose km-1}
$$

Функция коэффициентов $g(n)$ композиции $G(x)=R(F(x))$:
$$
g(n)=\sum^{n}_{k=1}{F^\Delta(n,k,m)r(k)}=\sum^{n}_{k=1}{{n+k(m-1)-1 \choose km-1}\frac{(-1)^{k-1}}{k}}
$$

Рассмотрим частные случаи параметров $m$:
\begin{itemize}
\item[1)] При $m=2$

\begin{math}
ng(n)=[1,3,4,3,1,0,1,3,4,3,1,0,1,3,4,3,1,0,1,3,\ldots]
\end{math}

Это последовательность целых чисел \href{http://oeis.org/A078070}{A078070} \cite{oeis}, откуда получаем:
$$
ng(n)=2-2\mathrm{cos}(\frac{\pi n}{3})
$$

Критерий простоты числа следующий: значение выражения
$$
\frac{2-2\mathrm{cos}(\frac{\pi n}{3})+(-1)^n}{n}
$$

является целым для простых чисел $n$.

Данный тест не является эффективным, так как выдает положительный результат теста для большого числа составных чисел.

Для последующих значений $m$ на данный момент неизвестна закрытая формула функции коэффициентов $ng(n)$, зависящая только от одного параметра $n$. Поэтому в общем виде можно использовать следующий общий критерий простоты числа:

При любых целых значениях $m$, значение выражения
$$
\sum^{n-1}_{k=1}{{n+k(m-1)-1 \choose km-1}\frac{(-1)^{n-1}}{n}}
$$

является целым для простых чисел $n$.

\end{itemize}

\subsection{Производящая функция $F(x)=\frac{x^m}{1-x}$}
Рассмотрим в качестве внутренней функции композиции $G(x)=R(F(x))$ следующую производящую функцию:
$$
F(x)=\frac{x^m}{1 - x}
$$

Композита для данной производящей функции \cite{KruchininVV}:
$$
F^\Delta(n,k,m)={n + k(m-1) - 1 \choose k - 1}
$$

Функция коэффициентов $g(n)$ композиции $G(x)=R(F(x))$:
$$
g(n)=\sum\limits _{k=1}^{n}{F^\Delta(n,k,m)r(k)}=\sum\limits _{k=1}^{n}{n - k(m-1) - 1 \choose k - 1}\frac{(-1)^{k-1}}{k}
$$

Для различных значений $m$ закрытая формула коэффициентов $ng(n)$ производной композиции $G(x)=R(F(x))$ трудно вычисляема, ввиду чего проверка количества ошибок данных тестов является вычислительно трудной задачей. В общем виде можно использовать следующий общий критерий простоты:

Для любых целых значений $m$ значение выражения
$$
\sum\limits _{k=1}^{n}{n + k(m-1) - 1 \choose k - 1} \frac{(-1)^{k-1}}{k}
$$

\subsection{Производящая функция $F(x)=\frac{x}{1 - a x - b x^2}$}
Рассмотрим в качестве внутренней функции композиции $G(x)=R(F(x))$ следующую производящую функцию:
$$
F(x)=\frac{x}{1 - ax - bx^2}
$$

Композита для данной производящей функции \cite{KruchininVV}:
$$
F^\Delta(n,k,a,b)=\sum\limits _{m=0}^{n-k} a^{-n + k + 2m}b^{n - k - m}{m \choose n-k-m}{k+m-1 \choose k-1}
$$

Функция коэффициентов $g(n)$ композиции $G(x)=R(F(x))$:
$$
g(n)=\sum\limits _{k=1}^{n}{F^\Delta(n,k,a,b)r(k)}=
$$
$$
=\sum\limits _{k=1}^{n}\frac{(-1)^{k-1}}{k}  \sum\limits _{m=0}^{n-k} a^{-n + k + 2m}b^{n - k - m}{m \choose n-k-m}{k+m-1 \choose k-1}
$$

Рассмотрим частные случаи параметров $a$ и $b$:
\begin{itemize}

\item[1)] При $a=1$ и $b=1$

\begin{math}
ng(n)=[1,1,4,5,11,16,29,45,76,121,199,320,521,841,1364,2205,\\
3571,5776,9349,15125,\ldots]
\end{math}

Это последовательность \href{http://oeis.org/A001350}{A001350} \cite{oeis}, представляющая собой последовательность ассоциированных чисел Мерсенна. Простой формулы для данной последовательности, зависящей только от $n$, на данный момент не найдено.

\item[2)] При $a=1$ и $b=2$

\begin{math}
ng(n)=[1,1,7,9,31,49,127,225,511,961,2047,3969,8191,16129,\\
32767,65025,131071,261121,524287,1046529,\ldots]
\end{math}

Это последовательность \href{http://oeis.org/A085903}{A085903} \cite{oeis}, откуда известно, что если принять $a(n)=ng(n)$, то
$$
a(2n) = {\mathit{M_n}}^2,
$$
где $\mathit{M_n}$ -- числа Мерсенна. Простой формулы для данной последовательности, зависящей только от $n$, на данный момент не найдено.
\end{itemize}

Для различных значений $a$ и $b$ закрытая формула коэффициентов $ng(n)$ производной композиции $G(x)=R(F(x))$ трудно вычисляема, ввиду чего проверка количества ошибок данных тестов является вычислительно трудной задачей. В общем виде можно использовать следующий общий критерий простоты:

Для любых целых значений $a$ и $b$ значение выражения
$$
\sum\limits _{k=1}^{n-1}\frac{(-1)^{k-1}}{k}  \sum\limits _{m=0}^{n-k} a^{-n + k + 2m}b^{n - k - m}{m \choose n-k-m}{k+m-1 \choose k-1}
$$

является целым для простых чисел $n$.

\subsection{Производящая функция $F(x)=\frac{x}{1 - a x - d x^4}$}
Рассмотрим в качестве внутренней функции композиции $G(x)=R(F(x))$ следующую производящую функцию:
$$
F(x)=\frac{x}{1-a\,x-d\,x^4}
$$

Композита для данной производящей функции \cite{KruchininVV}:
$$
F^\Delta(n,k,a,d)=\sum_{j=0}^{\frac{n-k}{3}}{a^{n-k-4 j} d^{j} {n-3 j-1 \choose k-1} {n-k-3 j \choose j}}
$$

Функция коэффициентов $g(n)$ композиции $G(x)=R(F(x))$:
$$
g(n)=\sum_{k=1}^{n}{F^\Delta(n,k,a,d) r(k)}=
$$
$$
=\sum_{k=1}^{n}{\sum_{j=0}^{\frac{n-k}{3}}{a^{n-k-4 j} d^{j} {n-3 j-1 \choose k-1} {n-k-3 j \choose j} \frac{(-1)^{k-1}}{k}}}
$$

Для различных значений $a, d$ на данный момент неизвестна закрытая формула функции коэффициентов $ng(n)$, зависящая только от одного параметра $n$, вследствие чего данная функция трудно вычисляема. Но при этом для небольших значений $n$ критерий показывает хороший результат. Поэтому в общем виде можно использовать следующий общий критерий простоты числа:

При любых целых значениях $a, d$, значение выражения
$$
\sum_{k=1}^{n-1}{\sum_{j=0}^{\frac{n-k}{3}}{a^{n-k-4 j} d^{j} {n-3 j-1 \choose k-1} {n-k-3 j \choose j} \frac{(-1)^{k-1}}{k}}}
$$

является целым для простых чисел $n$.

\subsection{Производящая функция $F(x)=\frac{x}{1- a x - e x^5}$}
Рассмотрим в качестве внутренней функции композиции $G(x)=R(F(x))$ следующую производящую функцию:
$$
F(x)=\frac{x}{1 - ax - ex^5}
$$

Композита для данной производящей функции \cite{KruchininVV}:
$$
F^\Delta(n,k,a,e)=\sum\limits _{j=0}^{\frac{n - k}{4}} a^{n - k - 5j} e^j {n - 4j - 1 \choose k - 1} {n - k - 4j \choose j}
$$

Функция коэффициентов $g(n)$ композиции $G(x)=R(F(x))$:
$$
g(n)=\sum\limits _{k=1}^{n}{F^\Delta(n,k,a,e)r(k)}=
$$
$$
=\sum\limits _{k=1}^{n} \frac{(-1)^{k-1}}{k} \sum\limits _{j=0}^{\frac{n - k}{4}} {a^{n - k - 5j} e^j {n - 4j - 1 \choose k - 1} {n - k - 4j \choose j}}
$$

Для различных значений $a$ и $e$ закрытая формула коэффициентов $ng(n)$ производной композиции $G(x)=R(F(x))$ трудно вычисляема, ввиду чего проверка количества ошибок данных тестов является вычислительно трудной задачей. В общем виде можно использовать следующий общий критерий простоты:

Для любых целых значений $a$ и $e$ значение выражения
$$
\sum\limits _{k=1}^{n-1} \frac{(-1)^{k-1}}{k} \sum\limits _{j=0}^{\frac{n - k}{4}} {a^{n - k - 5j} e^j {n - 4j - 1 \choose k - 1} {n - k - 4j \choose j}}
$$

является целым для простых чисел $n$.

\subsection{Производящая функция $F(x)=\frac{x}{1- a x - p x^m}$}
Рассмотрим в качестве внутренней функции композиции $G(x)=R(F(x))$ следующую производящую функцию:
$$
F(x)=\frac{x}{1 - a x - p x^m}
$$

Данная функция является обобщением для частных случаев, рассмотренных в пунктах 5.5-5.7.

Композита для данной производящей функции \cite{KruchininVV}:
$$
F^\Delta(n,k,a,p,m)=\sum_{j=0}^{\frac{n-k}{m-1}}{a^{n-k-m j} p^{j} {n-(m-1)j-1 \choose k-1} {n-k-(m-1)j \choose j}}
$$

Функция коэффициентов $g(n)$ композиции $G(x)=R(F(x))$:
$$
g(n)=\sum_{k=1}^{n}{F^\Delta(n,k,a,p,m) r(k)}=
$$
$$
=\sum_{k=1}^{n}{\sum_{j=0}^{\frac{n-k}{m-1}}{a^{n-k-m j} p^{j} {n-(m-1)j-1 \choose k-1} {n-k-(m-1)j \choose j} \frac{(-1)^{k-1}}{k}}}
$$

Для различных значений $a, p, m$ на данный момент неизвестна закрытая формула функции коэффициентов $ng(n)$, зависящая только от одного параметра $n$, вследствие чего данная функция трудно вычисляема. Поэтому в общем виде можно использовать следующий общий критерий простоты числа:

При любых целых значениях $a, p, m$, значение выражения
$$
\sum_{k=1}^{n-1}{\sum_{j=0}^{\frac{n-k}{m-1}}{a^{n-k-m j} p^{j} {n-(m-1)j-1 \choose k-1} {n-k-(m-1)j \choose j} \frac{(-1)^{k-1}}{k}}}
$$

является целым для простых чисел $n$.

\subsection{Производящая функция $F(x)=\frac{x}{1-x}-x^m$}
Рассмотрим в качестве внутренней функции композиции $G(x)=R(F(x))$ следующую производящую функцию:
$$
F(x)=\frac{x}{1 - x} - x^m
$$

Композита для данной производящей функции \cite{KruchininVV}:
$$
F^\Delta(n,k,m)=\sum\limits _{j=\left\lceil\frac{-n + km + 1}{m}\right\rceil}^{k}(-1)^{k - j} {k \choose j}{n - (k - j)m - 1 \choose j - 1} + (-1)^k \delta_{\left(km\right),n}
$$

Функция коэффициентов $g(n)$ композиции $G(x)=R(F(x))$:
$$
g(n)=\sum\limits _{k=1}^{n}{F^\Delta(n,k,m)r(k)}=
$$
$$
=\sum\limits _{k=1}^{n} \frac{(-1)^{k-1}}{k}  \sum\limits _{j=\left\lceil\frac{-n + km + 1}{m}\right\rceil}^{k} (-1)^{k - j} {k \choose j}{n - (k - j)m - 1 \choose j - 1} + (-1)^k \delta_{\left(km\right),n} 
$$

Для различных значений $m$ закрытая формула коэффициентов $ng(n)$ производной композиции $G(x)=R(F(x))$ трудно вычисляема, ввиду чего проверка количества ошибок данных тестов является вычислительно трудной задачей. В общем виде можно использовать следующий общий критерий простоты:

Для любых целых значений $m$ значение выражения
$$
\sum\limits _{k=1}^{n-1} \frac{(-1)^{k-1}}{k} \sum\limits _{j=\left\lceil\frac{-n + km + 1}{m}\right\rceil}^{k} (-1)^{k - j}{k \choose j}{n - (k - j)m - 1 \choose j - 1} + (-1)^k \delta_{\left(km\right),n}
$$

является целым для простых чисел $n$.

\subsection{Производящая функция $F(x)=\frac{x+x^2}{(1-x)^3}$}
Рассмотрим в качестве внутренней функции композиции $G(x)=R(F(x))$ следующую производящую функцию:
$$
F(x)=\frac{x+x^2}{(1-x)^3}
$$

Композита для данной производящей функции \cite{KruchininVV}:
$$
F^\Delta(n,k)=\sum_{i=k}^{n}{{k \choose i-k} {n+3k-i-1 \choose 3k-1}}
$$

Функция коэффициентов $g(n)$ композиции $G(x)=R(F(x))$:
$$
g(n)=\sum^{n}_{k=1}{F^\Delta(n,k)r(k)}=
\sum_{k=1}^{n}{\sum_{i=k}^{n}{{k \choose i-k} {n+3k-i-1 \choose 3k-1}} \frac{(-1)^{k-1}}{k}}
$$

\begin{math}
ng(n)=[1,3,4,3,1,0,1,3,4,3,1,0,1,3,4,3,1,0,1,3,\ldots]
\end{math}

Закрытая формула функции коэффициентов $ng(n)$, зависящая только от одного параметра $n$ на данный момент неизвестна, вследствие чего данная функция трудно вычисляема. Поэтому в общем виде можно использовать следующий общий критерий простоты числа:

значение выражения
$$
\sum_{k=1}^{n-1}{\sum_{i=k}^{n}{{k \choose i-k} {n+3k-i-1 \choose 3k-1}} \frac{(-1)^{k-1}}{k}}
$$

является целым для простых чисел $n$.

\subsection{Производящая функция $F(x)=\frac{x+x^2}{1-x-x^2}$}
Рассмотрим в качестве внутренней функции композиции $G(x)=R(F(x))$ следующую производящую функцию:
$$
F(x)=\frac{x+x^2}{1-x-x^2}
$$

Композита для данной производящей функции \cite{KruchininVV}:
$$
F^\Delta(n,k)=\sum\limits _{i=k}^{n} {i-1 \choose k-1}{i \choose n-i}
$$

Функция коэффициентов $g(n)$ композиции $G(x)=R(F(x))$:
$$
g(n)=\sum\limits _{k=1}^{n}{F^\Delta(n,k)r(k)}=\sum\limits _{k=1}^{n}\frac{(-1)^{k-1}}{k} 
\sum\limits _{i=k}^{n} {i-1 \choose k-1}{i \choose n-i}
$$

Для различных значений $m$ закрытая формула коэффициентов $ng(n)$ производной композиции $G(x)=R(F(x))$ трудно вычисляема, ввиду чего проверка количества ошибок данных тестов является вычислительно трудной задачей. В общем виде можно использовать следующий общий критерий простоты:

Для любых целых значений $m$ значение выражения
$$
\sum\limits _{k=1}^{n-1}\frac{(-1)^{k-1}}{k} 
\sum\limits _{i=k}^{n} {i-1 \choose k-1}{i \choose n-i}
$$

является целым для простых чисел $n$.

\subsection{Производящая функция $F(x)=\frac{x-x^2}{1-x+x^2}$}
Рассмотрим в качестве внутренней функции композиции $G(x)=R(F(x))$ следующую производящую функцию:
$$
F(x)=\frac{x-x^2}{1-x+x^2}
$$

Композита для данной производящей функции \cite{KruchininVV}:
$$
F^\Delta(n,k)=\sum_{i=k}^{n}{{i-1 \choose k-1} {i \choose n-i} (-1)^{n-i}}
$$

Функция коэффициентов $g(n)$ композиции $G(x)=R(F(x))$:
$$
g(n)=\sum^{n}_{k=1}{F^\Delta(n,k)r(k)}=\sum^{n}_{k=1}{\sum_{i=k}^{n}{{i-1 \choose k-1} {i \choose n-i} (-1)^{n-i}} \frac{(-1)^{k-1}}{k}}
$$

\begin{math}
ng(n)=[1,-1,-2,-1,1,2,1,-1,-2,-1,1,2,1,-1,-2,-1,1,2,1,-1,\ldots]
\end{math}

Это последовательность целых чисел \href{http://oeis.org/A087204}{A087204} \cite{oeis}, откуда получаем:
$$
ng(n)=-\frac{1}{6} ( (n)(mod 6) + 2 (n+1)(mod 6) + (n+2)(mod 6) -
$$
$$- (n+3)(mod 6) - 2 (n+4)(mod 6) - (n+5)(mod 6))
$$

Критерий простоты числа следующий: значение выражения
$$
\frac{-\frac{1}{6} ( (n)(mod 6) + 2 (n+1)(mod 6) + (n+2)(mod 6)}{n} +
$$
$$+ \frac{-(n+3)(mod 6) - 2 (n+4)(mod 6) - (n+5)(mod 6)) - (-1)^n}{n}
$$

является целым для простых чисел $n$.

Данный тест не является эффективным, так как выдает положительный результат теста для огромного количества составных чисел.

\subsection{Производящая функция $F(x)=\frac{1-\sqrt{1-4abx}}{2a}$}
Рассмотрим в качестве внутренней функции композиции $G(x)=R(F(x))$ следующую производящую функцию:
$$
F(x)=\frac{1-\sqrt{1-4abx}}{2a}
$$

Композита для данной производящей функции \cite{KruchininVV}:
$$
F^\Delta(n,k,a,b)=\frac{a^{n-k}b^nk{2n-k-1 \choose n-1}}{n}
$$

Функция коэффициентов $g(n)$ композиции $G(x)=R(F(x))$:
$$
g(n)=\sum\limits _{k=1}^{n}{F^\Delta(n,k,a,b)r(k)}=\sum\limits _{k=1}^{n}\frac{(-1)^{k-1}}{k} 
\frac{a^{n-k}b^nk{2n-k-1 \choose n-1}}{n}
$$

Для различных значений $a$ и $b$ закрытая формула коэффициентов $ng(n)$ производной композиции $G(x)=R(F(x))$ трудно вычисляема, ввиду чего проверка количества ошибок данных тестов является вычислительно трудной задачей. В общем виде можно использовать следующий общий критерий простоты:

Для любых целых значений $a$ и $b$ значение выражения
$$
\sum\limits _{k=1}^{n-1} (-1)^{k-1} 
\frac{a^{n-k}b^n{2n-k-1 \choose n-1}}{n}
$$

является целым для простых чисел $n$.

\subsection{Производящая функция $F(x)=\frac{1-2x-\sqrt{1-4x}}{2x}$}
Рассмотрим в качестве внутренней функции композиции $G(x)=R(F(x))$ следующую производящую функцию:
$$
F(x)=\frac{1-2x-\sqrt{1-4x}}{2x}
$$

Композита для данной производящей функции \cite{KruchininVV}:
$$
F^\Delta(n,k)=\frac{k}{n} {2n \choose n-k}
$$

Функция коэффициентов $g(n)$ композиции $G(x)=R(F(x))$:
$$
g(n)=\sum^{n}_{k=1}{F^\Delta(n,k)r(k)}=\sum^{n}_{k=1}{\frac{k}{n} {2n \choose n-k} \frac{(-1)^{k-1}}{k}}
$$

\begin{math}
ng(n)=[1,3,10,35,126,462,1716,6435,24310,92378,352716,1352078,\\
5200300,20058300,77558760,300540195,1166803110,4537567650,17672631900,\\
68923264410,\ldots]
\end{math}

Закрытая формула функции коэффициентов $ng(n)$, зависящая только от одного параметра $n$ на данный момент неизвестна, вследствие чего данная функция трудно вычисляема. Поэтому в общем виде можно использовать следующий общий критерий простоты числа:

значение выражения
$$
\sum^{n-1}_{k=1}{\frac{(-1)^{k-1}}{n} {2n \choose n-k}}
$$

является целым для простых чисел $n$.

\subsection{Производящая функция $F(x)=\frac{1-\sqrt{1-4x}}{2(1-x)}$}
Рассмотрим в качестве внутренней функции композиции $G(x)=R(F(x))$ следующую производящую функцию:
$$
F(x)=\frac{1-\sqrt{1-4x}}{2(1-x)}
$$

Композита для данной производящей функции \cite{KruchininVV}:
$$
F^\Delta(n,k)=k \sum^{n-k}_{i=0}{\frac{{k+i-1 \choose k-1} {2(n-i)-k-1 \choose n-i-1}}{n-i}}
$$

Функция коэффициентов $g(n)$ композиции $G(x)=R(F(x))$:
$$
g(n)=\sum^{n}_{k=1}{F^\Delta(n,k)r(k)}=\sum^{n}_{k=1}{k \sum^{n-k}_{i=0}{\frac{{k+i-1 \choose k-1} {2(n-i)-k-1 \choose n-i-1}}{n-i}} \frac{(-1)^{k-1}}{k}}
$$

\begin{math}
ng(n)=[1,3,7,19,61,213,771,2843,10609,39933,151317,576445,2205581,\\
8469737,32625547,126007739,487793921,1892133357,7352607157,28616740029,\ldots]
\end{math}

Закрытая формула функции коэффициентов $ng(n)$, зависящая только от одного параметра $n$ на данный момент неизвестна, вследствие чего данная функция трудно вычисляема. Поэтому в общем виде можно использовать следующий общий критерий простоты числа:

значение выражения
$$
\sum^{n-1}_{k=1}{\sum^{n-k}_{i=0}{\frac{{k+i-1 \choose k-1} {2(n-i)-k-1 \choose n-i-1}}{n-i}} (-1)^{k-1}}
$$

является целым для простых чисел $n$.

\subsection{Производящая функция $F(x)=\frac{-1-x+\sqrt{1+2x+5x^2}}{2x}$}
Рассмотрим в качестве внутренней функции композиции $G(x)=R(F(x))$ следующую производящую функцию:
$$
F(x)=\frac{-1-x+\sqrt{1+2x+5x^2}}{2x}
$$

Композита для данной производящей функции \cite{KruchininVV}:
$$
F^\Delta(n,k)=\frac{k}{n} \sum\limits _{j=0}^n{j \choose -n-k+2j}(-1)^{j-k}{n \choose j}
$$

Функция коэффициентов $g(n)$ композиции $G(x)=R(F(x))$:
$$
g(n)=\sum\limits _{k=1}^{n}{F^\Delta(n,k)r(k)}=\sum\limits _{k=1}^{n}\frac{(-1)^{k-1}}{k} 
\frac{k}{n} \sum\limits _{j=0}^n{j \choose -n-k+2j}(-1)^{j-k}{n \choose j}
$$

Закрытая формула коэффициентов $ng(n)$ производной композиции $G(x)=R(F(x))$ трудно вычисляема, ввиду чего проверка количества ошибок данных тестов является вычислительно трудной задачей. В общем виде можно использовать следующий общий критерий простоты:

Значение выражения
$$
\sum\limits _{k=1}^{n-1}(-1)^{k-1}\frac{a^{n-k}b^n{2n-k-1 \choose n-1}}{n}
$$

является целым для простых чисел $n$.

\subsection{Производящая функция $F(x)=\frac{1+x-\sqrt{1-6x+x^2}}{2x}$}
Рассмотрим в качестве внутренней функции композиции $G(x)=R(F(x))$ следующую производящую функцию:
$$
F(x)=\frac{1+x-\sqrt{1-6x+x^2}}{2x}
$$

Композита для данной производящей функции \cite{KruchininVV}:
$$
F^\Delta(n,k)=\frac{k}{2^kn} \sum\limits _{j=0}^{n-k}(-1)^j2^{n-j}{n \choose j}{2n-k-j-1 \choose n-1}
$$

Функция коэффициентов $g(n)$ композиции $G(x)=R(F(x))$:
$$
g(n)=\sum\limits _{k=1}^{n}{F^\Delta(n,k)r(k)}=\sum\limits _{k=1}^{n}\frac{(-1)^{k-1}}{k} 
\frac{k}{2^kn} \sum\limits _{j=0}^{n-k}(-1)^j2^{n-j}{n \choose j}{2n-k-j-1 \choose n-1}
$$

Закрытая формула коэффициентов $ng(n)$ производной композиции $G(x)=R(F(x))$ трудно вычисляема, ввиду чего проверка количества ошибок данных тестов является вычислительно трудной задачей. В общем виде можно использовать следующий общий критерий простоты:

Значение выражения
$$
\sum\limits _{k=1}^{n-1}\frac{(-1)^{k-1}}{2^kn}\sum\limits _{j=0}^{n-k}(-1)^j2^{n-j}{n \choose j}{2n-k-j-1 \choose n-1}
$$

является целым для простых чисел $n$.

\subsection{Производящая функция $F(x)=\frac{1-x-\sqrt{1-6x+x^2}}{2x}$}
Рассмотрим в качестве внутренней функции композиции $G(x)=R(F(x))$ следующую производящую функцию:
$$
F(x)=\frac{1-x-\sqrt{1-6x+x^2}}{2x}
$$

Композита для данной производящей функции \cite{KruchininVV}:
$$
F^\Delta(n,k)=\frac{k \sum_{i=0}^{n-k}{{n \choose i} {2n-k-i-1 \choose n-1}}}{n}
$$

Функция коэффициентов $g(n)$ композиции $G(x)=R(F(x))$:
$$
g(n)=\sum_{k=1}^{n}{F^\Delta(n,k) r(k)}=\sum_{k=1}^{n}{\frac{k \sum_{i=0}^{n-k}{{n \choose i} {2n-k-i-1 \choose n-1}}}{n} \frac{(-1)^{k-1}}{k}}
$$

\begin{math}
1,3,13,63,321,1683,8989,48639,265729,1462563,8097453,45046719,\\
251595969,1409933619,7923848253,44642381823,252055236609,\\
1425834724419,8079317057869,45849429914943,\ldots]
\end{math}

Это последовательность целых чисел \href{http://oeis.org/A001850}{A001850} \cite{oeis}

Закрытая формула функции коэффициентов $ng(n)$, зависящая только от одного параметра $n$ на данный момент неизвестна, вследствие чего данная функция трудно вычисляема. Поэтому в общем виде можно использовать следующий общий критерий простоты числа:

значение выражения
$$
\sum_{k=1}^{n-1}{\sum_{i=0}^{n-k}{{n \choose i} {2n-k-i-1 \choose n-1}} \frac{(-1)^{k-1}}{n}}
$$

является целым для простых чисел $n$.

\subsection{Производящая функция $F(x)=\frac{-1-x+\sqrt{1-2x-3x^2}}{2x^2}$}
Рассмотрим в качестве внутренней функции композиции $G(x)=R(F(x))$ следующую производящую функцию:
$$
F(x)=\frac{-1-x+\sqrt{1-2x-3x^2}}{2x^2}
$$

Композита для данной производящей функции \cite{KruchininVV}:
$$
F^\Delta(n,k)=\frac{k}{n} \sum_{j=0}^{n}{{j \choose -n-k+2j} {n \choose j}}
$$

Функция коэффициентов $g(n)$ композиции $G(x)=R(F(x))$:
$$
g(n)=\sum_{k=1}^{n}{F^\Delta(n,k) r(k)}=\sum_{k=1}^{n}{\frac{k}{n} \sum_{j=0}^{n}{{j \choose -n-k+2j} {n \choose j}} \frac{(-1)^{k-1}}{k}}
$$

\begin{math}
1,1,4,9,26,70,197,553,1570,4476,12827,36894,106471,308113,893804,\\
2598313,7567466,22076404,64498427,188689684,\ldots]
\end{math}

Закрытая формула функции коэффициентов $ng(n)$, зависящая только от одного параметра $n$ на данный момент неизвестна, вследствие чего данная функция трудно вычисляема. Поэтому в общем виде можно использовать следующий общий критерий простоты числа:

значение выражения
$$
\sum_{k=1}^{n-1}{\frac{(-1)^{k-1}}{n} \sum_{j=0}^{n}{a^{n-k-m j} {{j \choose -n-k+2j} {n \choose j}}}}
$$

является целым для простых чисел $n$.

\subsection{Производящая функция $F(x)=\frac{1+x-\sqrt{1-2x-3x^2}}{2(1+x)}$}
Рассмотрим в качестве внутренней функции композиции $G(x)=R(F(x))$ следующую производящую функцию:
$$
F(x)=\frac{1+x-\sqrt{1-2x-3x^2}}{2(1+x)}
$$

Композита для данной производящей функции \cite{KruchininVV}:
$$
F^\Delta(n,k)=\frac{k}{n} \sum\limits _{j=k}^n(-1)^{n-j}{n \choose j}{-k+2j-1 \choose j-1}
$$

Функция коэффициентов $g(n)$ композиции $G(x)=R(F(x))$:
$$
g(n)=\sum\limits _{k=1}^{n}{F^\Delta(n,k)r(k)}=\sum\limits _{k=1}^{n}\frac{(-1)^{k-1}}{k} 
\frac{k}{n} \sum\limits _{j=k}^n(-1)^{n-j}{n \choose j}{-k+2j-1 \choose j-1}
$$

Закрытая формула коэффициентов $ng(n)$ производной композиции $G(x)=R(F(x))$ трудно вычисляема, ввиду чего проверка количества ошибок данных тестов является вычислительно трудной задачей. В общем виде можно использовать следующий общий критерий простоты:

Значение выражения
$$
\sum\limits _{k=1}^{n-1}\frac{(-1)^{k-1}}{n} \sum\limits _{j=k}^n(-1)^{n-j}{n \choose j}{-k+2j-1 \choose j-1}
$$

является целым для простых чисел $n$.

\newpage
\section{Заключение}
В результате проделанной работы были достигнуты следующие результаты:
\begin{enumerate}
\item[1)] изучена теоретическая часть о производящих функциях, необходимая для построения критериев простоты числа; 
\item[2)] изучена методика построения новых критериев простоты числа;
\item[3)] построены и проанализированы новые критерии простоты числа;
\item[4)] освоена система компьютерной алгебры, с помощью которой проводились построение и анализ критериев простоты.
\item[5)] написан отчет в формате программной среде LaTeX, предназначенной для качественного представления математической документации.
\end{enumerate}

В дальнейшей работе планируется:
\begin{itemize}
\item продолжить построение новых критериев с использованием других внутренних и внешних производящих функций;
\item попытаться сгенерировать собственный тест на простоту числа.
\end{itemize}

\newpage
% - Переименование списка литературы
\renewcommand{\refname}{Список использованных источников}
\bibliography{books}

\ESKDappendix{Обязательное}{\normalfont Компакт-диск}
Компакт-диск содержит: 
\begin{itemize}
\item электронную версию пояснительной записки в форматах *.tex и *.pdf;
\item индивидуальные ежемесячные отчеты студентов;
\item групповые ежемесячные отчёты.
\end{itemize}

\end{document}