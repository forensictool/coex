\documentclass[russian,utf8]{eskdtext}
\usepackage[numbertop, numbercenter]{eskdplain}

% - Полуторный интервал
\renewcommand{\baselinestretch}{1.50}
% - Отступ красной строки
\setlength{\parindent}{1.25cm}	
% - Шрифт Times New Roman
\renewcommand{\rmdefault}{ftm}

% - Убирает точку в списке литературы
\makeatletter
\def\@biblabel#1{#1 }

% - ГОСТ списка литературы
\bibliographystyle{gost2008my}

% - Верикальные отступы заголовков 
\ESKDsectSkip{section}{1em}{1em}
\ESKDsectSkip{subsection}{1em}{1em}
\ESKDsectSkip{subsubsection}{1em}{1em}

% - Изменение заголовков
\usepackage{titlesec}
\titleformat{\section}{\centering\normalfont\normalsize}{\thesection}{1.0em}{}
\titleformat{\subsection}{\centering\normalfont\normalsize}{\thesubsection}{1.0em}{}
\titleformat{\subsubsection}{\centering\normalfont\normalsize}{\thesubsubsection}{1.0em}{}
\titleformat{\paragraph}{\normalsize}{\theparagraph}{1.0em}{}

% - Для больших таблиц
\usepackage{longtable}
\usepackage{tabularx}
\renewcommand{\thetable}{\thesection.\arabic{table}}

% - Используем графику в документе
\usepackage{graphicx}
\graphicspath{{images/}}
\renewcommand{\thefigure}{\thesection.\arabic{figure}}

% - Счётчики
\usepackage{eskdtotal}

% - Для переопределения списков
\makeatletter
\renewcommand{\theenumi}{\arabic{enumi}}
\renewcommand{\labelenumi}{\theenumi)}

\sloppy

\begin{document}

% Титульная страница
\newpage
\ESKDthisStyle{empty}

\begin{center}
Министерство образования и науки Российской Федерации\\
Федеральное государственное бюджетное образовательное учреждение высшего профессионального образования\\
<<ТОМСКИЙ ГОСУДАРСТВЕННЫЙ УНИВЕРСИТЕТ СИСТЕМ УПРАВЛЕНИЯ И РАДИОЭЛЕКТРОНИКИ>> (ТУСУР)\\
Кафедра комплексной информационной безопасности электронно-вычислительных систем (КИБЭВС)\\
\end{center}

\begin{tabbing}
XXXXXXXXXXXXXXXXXXXXXXXXXXX \=
XXXXXXXXXXXXXXXX\kill
\> УТВЕРЖДАЮ\\
\> заведующий каф.КИБЭВС\\
\> \underline{\ \ \ \ \ \ \ \ \ \ \ \ \ \ \ \ \ \ \ \ } А.А. Шелупанов\\
\> <<\underline{\ \ \ \ \ }>>\underline{\ \ \ \ \ \ \ \ \ \ \ \ \ \ \ \ \ \ \ \ } 2014г.\\
\end{tabbing}

\begin{center}
КОМПЬЮТЕРНАЯ ЭКСПЕРТИЗА\\
Отчет по групповому проектному обучению\\
Группа КИБЭВС-1208\\
\end{center}

\begin{tabbing}
XXXXXXXXXXXXXXXXXXXXXXXXXXX \=
XXXXXXXXXXXXXXXX\kill
\> Ответственный исполнитель\\
\> Студент гр. 720-1\\
\> \underline{\ \ \ \ \ \ \ \ \ \ \ \ \ \ \ \ \ \ \ \ } Никифоров Д. С.\\
\> <<\underline{\ \ \ \ \ }>>\underline{\ \ \ \ \ \ \ \ \ \ \ \ \ \ \ \ \ \ \ \ } 2014г.\\
\ \\
\> Научный руководитель\\
\> Аспирант каф.КИБЭВС\\
\> \underline{\ \ \ \ \ \ \ \ \ \ \ \ \ \ \ \ \ \ \ \ } Гуляев А. И.\\
\> <<\underline{\ \ \ \ \ }>>\underline{\ \ \ \ \ \ \ \ \ \ \ \ \ \ \ \ \ \ \ \ } 2014г.\\
\end{tabbing}
\vfill
\begin{center}
Томск -- 2014
\end{center}

% Реферат
\newpage
\ESKDthisStyle{empty}
\paragraph{\hfill РЕФЕРАТ \hfill}
Курсовая работа содержит \ESKDtotal{page} страниц, \ESKDtotal{figure} рисунка, \ESKDtotal{table} таблицы, \ESKDtotal{bibitem} источников, \ESKDtotal{appendix} приложение.

%допилить ключевые слова
КОМПЬЮТЕРНАЯ ЭКСПЕРТИЗА, ФОРЕНЗИКА, ЛОГИ, QT, XML, GIT, LATEX, MOZILLA THUNDERBIRD, MS OUTLOOK, WINDOWS, PST, MSG, RTF, HTML, БИБЛИОТЕКИ, РЕПОЗИТОРИЙ, ПОЧТОВЫЙ КЛИЕНТ, SQLLITE, РЕЕСТР, МЕТА-ДАННЫЕ, READPST, JPEG, PNG, ID3V1, JFIF, RIFF, CHUNK, DBX, C++.

Цель работы --- создание программного комплекса, предназначенного для проведения компьютерной экспертизы.

Задачей, поставленной на данный семестр, стало написание программного комплекса, имеющего следующие возможности: 
\begin{enumerate}
\item \textbf{ПРАВИТЬ}
\end{enumerate}

Результаты работы в данном семестре:

\begin{itemize}
\item \textbf{ПРАВИТЬ}
\end{itemize}
5 Заключение
В этом семестре было проделано:
Исследование юнит-тестирования в инструментарии Qt;
Имплементация юнит-тестирования для пробной программы;
Имплементация юнит-тестирования для модуля, сканирующего медиа-файлы;

Пояснительная записка выполнена при помощи системы компьютерной вёрстки \LaTeX.


\ESKDstyle{plain}

% Латех
\input{tex}

% Список литературы
\newpage
\renewcommand{\refname}{Список использованных источников}
\bibliography{lit}

\end{document}   
