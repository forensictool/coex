\documentclass[russian,utf8]{eskdtext}
\usepackage[numbertop, numbercenter]{eskdplain}

% - Полуторный интервал
\renewcommand{\baselinestretch}{1.50}
% - Отступ красной строки
\setlength{\parindent}{1.25cm}	
% - Шрифт Times New Roman
\renewcommand{\rmdefault}{ftm}

% - Убирает точку в списке литературы
\makeatletter
\def\@biblabel#1{#1 }

% - ГОСТ списка литературы
\bibliographystyle{gost2008my}

% - Верикальные отступы заголовков 
\ESKDsectSkip{section}{1em}{1em}
\ESKDsectSkip{subsection}{1em}{1em}
\ESKDsectSkip{subsubsection}{1em}{1em}

% - Изменение заголовков
\usepackage{titlesec}
\titleformat{\section}{\centering\normalfont\normalsize}{\thesection}{1.0em}{}
\titleformat{\subsection}{\centering\normalfont\normalsize}{\thesubsection}{1.0em}{}
\titleformat{\subsubsection}{\centering\normalfont\normalsize}{\thesubsubsection}{1.0em}{}
\titleformat{\paragraph}{\normalsize}{\theparagraph}{1.0em}{}

% - Для больших таблиц
\usepackage{longtable}
\usepackage{tabularx}
\renewcommand{\thetable}{\thesection.\arabic{table}}

% - Используем графику в документе
\usepackage{graphicx}
\graphicspath{{images/}}
\renewcommand{\thefigure}{\thesection.\arabic{figure}}

% - Счётчики
\usepackage{eskdtotal}

% - Для переопределения списков
\makeatletter
\renewcommand{\theenumi}{\arabic{enumi}}
\renewcommand{\labelenumi}{\theenumi)}

\sloppy

\begin{document}

% Титульная страница
\newpage
\ESKDthisStyle{empty}

\begin{center}
Министерство образования и науки Российской Федерации\\
Федеральное государственное бюджетное образовательное учреждение высшего профессионального образования\\
<<ТОМСКИЙ ГОСУДАРСТВЕННЫЙ УНИВЕРСИТЕТ СИСТЕМ УПРАВЛЕНИЯ И РАДИОЭЛЕКТРОНИКИ>> (ТУСУР)\\
Кафедра комплексной информационной безопасности электронно-вычислительных систем (КИБЭВС)\\
\end{center}

\begin{tabbing}
XXXXXXXXXXXXXXXXXXXXXXXXXXX \=
XXXXXXXXXXXXXXXX\kill
\> УТВЕРЖДАЮ\\
\> заведующий каф.КИБЭВС\\
\> \underline{\ \ \ \ \ \ \ \ \ \ \ \ \ \ \ \ \ \ \ \ } А.А. Шелупанов\\
\> <<\underline{\ \ \ \ \ }>>\underline{\ \ \ \ \ \ \ \ \ \ \ \ \ \ \ \ \ \ \ \ } 2014г.\\
\end{tabbing}

\begin{center}
КОМПЬЮТЕРНАЯ ЭКСПЕРТИЗА\\
Отчет по групповому проектному обучению\\
Группа КИБЭВС-1208\\
\end{center}

\begin{tabbing}
XXXXXXXXXXXXXXXXXXXXXXXXXXX \=
XXXXXXXXXXXXXXXX\kill
\> Ответственный исполнитель\\
\> Студент гр. 720-1\\
\> \underline{\ \ \ \ \ \ \ \ \ \ \ \ \ \ \ \ \ \ \ \ } Никифоров Д. С.\\
\> <<\underline{\ \ \ \ \ }>>\underline{\ \ \ \ \ \ \ \ \ \ \ \ \ \ \ \ \ \ \ \ } 2014г.\\
\ \\
\> Научный руководитель\\
\> Аспирант каф.КИБЭВС\\
\> \underline{\ \ \ \ \ \ \ \ \ \ \ \ \ \ \ \ \ \ \ \ } Гуляев А. И.\\
\> <<\underline{\ \ \ \ \ }>>\underline{\ \ \ \ \ \ \ \ \ \ \ \ \ \ \ \ \ \ \ \ } 2014г.\\
\end{tabbing}
\vfill
\begin{center}
Томск -- 2014
\end{center}

% Реферат
\newpage
\ESKDthisStyle{empty}
\paragraph{\hfill РЕФЕРАТ \textbf{ПРАВИТЬ!!!} \hfill}
Курсовая работа содержит \ESKDtotal{page} страниц, \ESKDtotal{figure} рисунка, \ESKDtotal{table} таблицы, \ESKDtotal{bibitem} источников, \ESKDtotal{appendix} приложение.

%допилить ключевые слова
КОМПЬЮТЕРНАЯ ЭКСПЕРТИЗА, ФОРЕНЗИКА, ЛОГИ, QT, XML, GIT, LATEX, ICQ, MS OUTLOOK, WINDOWS, PST, MSG, RTF, HTML, БИБЛИОТЕКИ, РЕПОЗИТОРИЙ, МЕССЕНДЖЕР, ПОЧТОВЫЙ КЛИЕНТ, SQLLITE, РЕЕСТР, ИЗОБРАЖЕНИЯ, READPST, JPEG, PNG.

Цель работы --- создание программного комплекса, предназначенного для проведения компьютерной экспертизы.

Задачей, поставленной на данный семестр, стало написание программного комплекса, имеющего следующие возможности: 
\begin{enumerate}
\item сбор и анализ информации из реестра;
\item сбор и анализ информации из журналов истории браузеров;
\item сбор и анализ информации из мессенджеров;
\item сбор и анализ информации из почтовых приложений;
\item идентификации файлов изображений по внутреннему содержимому и их проверка;
\item сбора информации об установленном ПО по остаточным файлам.
\end{enumerate}

Результаты работы в данном семестре:

\begin{itemize}
\item реализован алгоритм извлечения строковых переменных из реестра Windows;
\item реализован алгоритм побитового считывания файла формата PST;
\item реализован импорт истории (посещений, поисковых запросов, загруженных файлов), закладок и 
другой информации (версия приложения, логин аккаунта google) из приложения Google Chrome;
\item реализован алгоритм парсинга контактного листа пользователя, сохраняемого приложением ICQ;
\item реализована проверка конца файла для форматов JPEG и PNG (для идентификации файлов изображений) и проверка заголовков 5 форматов изображений;
\end{itemize}

Пояснительная записка выполнена при помощи системы компьютерной вёрстки \LaTeX.


\ESKDstyle{plain}

% Латех
\TeX\ --- это созданная американским математиком и программистом Дональдом Кнутом система для вёрстки текстов. Сам по себе \TeX\ представляет собой специализированный язык программирования.Каждая издательская система представляет собой пакет макроопределений этого языка.

\LaTeX\ --- это созданная Лэсли Лэмпортом издательская система на базе \TeX'а \cite{lvovskyi} \LaTeX\ позволяет пользователю сконцентрировать свои услия на содержании и структуре текста, не заботясь о деталях его оформления.

Для подготовки отчётной и иной документации нами был выбран \LaTeX\, так как совместно с системой контроля версий Git он предоставляет возможность совместного создания и редактирования документов. Огромным достоинством системы \LaTeX\ то, что создаваемые с её помощью файлы обладают высокой степенью переносимости. \cite{latexrus}

Совместно с \LaTeX\ часто используется Bib\TeX\ --- программное обеспечение для создания форматированных списков библиографии. Оно входит в состав дистрибутива \LaTeX\ и позволяет создавать удобную, универсальную и долговечную библиографию. Bib\TeX\ стал одной из причин, по которой нами был выбран \LaTeX\ для создания документации.


% Список литературы
\newpage
\renewcommand{\refname}{Список использованных источников}
\bibliography{lit}

\end{document}   
